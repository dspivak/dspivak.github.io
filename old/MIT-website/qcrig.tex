\documentclass{amsart} % Nicer than default article style: less
                        %flashy headings, etc.

\usepackage{amsmath,amsthm}     % Handy math stuff, theorem environments.
\usepackage{amssymb,empheq}            % Fancy math symbols.
\usepackage{euscript}           % Nice script font.
\usepackage{graphicx,enumerate,calc,lscape,color}
\definecolor{red}{rgb}{1,0,0}
\usepackage[matrix,arrow,curve,frame]{xy}    % XY-pic diagram pac
\usepackage{hyperref}

\xymatrixcolsep{1.9pc}                          % Adjust size of diagrams.
\xymatrixrowsep{1.9pc}
\newdir{ >}{{}*!/-5pt/\dir{>}}                  % Make better tailed arrows

% Don't force the bottoms of the pages to be at the same spot:
\raggedbottom

\def\longbfib{\DOTSB\twoheadleftarrow\joinrel\relbar}
%\def\longbcofib{\DOTSB\relbar\joinrel\leftarrowtail}
\def\longbcofib{\leftarrowtail}
\def\longfib{\DOTSB\relbar\joinrel\twoheadrightarrow}

% Allow worse line breaks.  We'll get fewer ``underfull hbox'' messages.
\tolerance=1500

% Put sections, but not subsections, into the table of contents:
\setcounter{tocdepth}{1}
% Number sections, but not subsections.
\setcounter{secnumdepth}{2}

% For temporary questions.  For example, \margnote{This is something
% I'm confused about.} puts that message in the margin.
\newcommand{\margnote}[1]{\mbox{}\marginpar{\hspace{0pt}#1}}

% Some theorem-like environments, all numbered together starting at 1
% in each section.

% The default \theoremstyle is bold headings and italic body text.
\newtheorem{thm}[subsection]{Theorem}
\newtheorem{defn}[subsection]{Definition}
\newtheorem{prop}[subsection]{Proposition}
\newtheorem{claim}[subsection]{Claim}
\newtheorem{cor}[subsection]{Corollary}
\newtheorem{lemma}[subsection]{Lemma}
\newtheorem{conj}[subsection]{Conjecture}
\newtheorem{ques}[subsection]{Question}
\newtheorem{problem}[subsection]{Problem}

\theoremstyle{definition}  % Bold headings and Roman body text.
\newtheorem{ex}[subsection]{Example}
\newtheorem{exercise}[subsection]{Exercise}
\newtheorem{convention}[subsection]{Convention}
\newtheorem{note}[subsection]{Note}
\newtheorem{warning}[subsection]{Warning}
\newtheorem{remark}[subsection]{Remark}
\newtheorem{discussion}[subsection]{Discussion}

\newcommand{\entrylabel}[1]{\mbox{\textsf{#1}}\hfil}

\newenvironment{entry}
  {\begin{list}{}%
        {\renewcommand{\makelabel}{\entrylabel}%
         \setlength{\labelwidth}{70pt}%
         \setlength{\leftmargin}{\labelwidth+\labelsep}%
        }%
  }%
  {\end{list}}

\newcommand{\dfn}{\textbf} % Make defined words bold.
\newcommand{\future}[1]{{\textbf{#1}}}
\newcommand{\mdfn}[1]{\dfn{\mathversion{bold}#1}} % Even make math bold

% Various commands that are useful.  Please add your own.

% Topologists smash /\ and wedge \/.
\newcommand{\Smash}             {\wedge}
\newcommand{\bigSmash}          {\bigwedge}
\newcommand{\Wedge}             {\vee}
\DeclareRobustCommand{\bigWedge}{\bigvee}
\newcommand{\tens}              {\otimes}               %tensor
\newcommand{\iso}               {\cong}
\newcommand{\eqhsma}[1]{\underset              %Equivariant half-smash:
                           {\Sigma_{#1}}       %\eqhsma{n} produces
                           {\rtimes}}          %`half-smash over Sigma_n'


\newcommand{\cat}{\EuScript}    % Use \EuScript to name a category.
\newcommand{\cA}{{\cat A}}      % Only seems to work for caps, and only gets
\newcommand{\cB}{{\cat B}}      % first letter.
\newcommand{\cC}{{\cat C}}
\newcommand{\cD}{{\cat D}}
\newcommand{\cE}{{\cat E}}
\newcommand{\cF}{{\cat F}}
\newcommand{\cG}{{\cat G}}
\newcommand{\cH}{{\cat H}}
\newcommand{\cI}{{\cat I}}
\newcommand{\cJ}{{\cat J}}
\newcommand{\cL}{{\cat L}}
\newcommand{\cM}{{\cat M}}
\newcommand{\cN}{{\cat N}}
\newcommand{\cO}{{\cat O}}
\newcommand{\cP}{{\cat P}}
\newcommand{\cQ}{{\cat Q}}
\newcommand{\cR}{{\cat R}}
\newcommand{\cS}{{\cat S}}
\newcommand{\cT}{{\cat T}}
\newcommand{\cV}{{\cat V}}
\newcommand{\cW}{{\cat W}}
\newcommand{\cY}{{\cat Y}}
\newcommand{\Top}{{\cat Top}}
\newcommand{\kG}{{\mathcal G}}
\newcommand{\Gtop}{G\Top}
\newcommand{\Spectra}{{\cat Spectra}}
\newcommand{\Motspectra}{{\cat MotSpectra}}
\DeclareMathOperator{\tr}{tr}
\DeclareMathOperator{\topr}{\psi}
\newcommand{\Set}{{\cat Set}}
\newcommand{\cCat}{{\cat Cat}}
\newcommand{\sSet}{{s{\cat Set}}}
\newcommand{\Sch}{{\cat Sch}}
\newcommand{\Man}{{\cat Man}}
\newcommand{\Ab}{{\cat Ab}}
\newcommand{\Mod}{{\cat Mod}}
\newcommand{\Grph}{{\cat Grph}}
\newcommand{\Grpd}{{\cat Grpd}}
\newcommand{\Cat}{{\cat Cat}}
\newcommand{\sPre}{sPre}
\newcommand{\Pre}{Pre}
\newcommand{\Shv}{Shv}
\newcommand{\Ho}{\text{Ho}\,}
\newcommand{\ho}{\text{Ho}\,}

\newcommand{\sSetJ}{\sSet_J}
\newcommand{\univ}{\EuScript}   % Use \EuScript to name a universe.
\newcommand{\uD}{{\univ D}}
\newcommand{\uU}{{\univ U}}
\newcommand{\uV}{{\univ V}}
\newcommand{\uW}{{\univ W}}
\newcommand{\uA}{{\univ A}}
\newcommand{\uB}{{\univ B}}
\newcommand{\uS}{{\univ S}}

\newcommand{\kK}{{\mathcal K}}
\newcommand{\kE}{{\mathcal E}}

\newcommand{\field}[1]  {\mathbb #1} % Use blackboard bold for these sets
\newcommand{\A}         {\field A}
\newcommand{\R}         {\field R}
\newcommand{\LL}        {\field L}
\newcommand{\N}         {\field N}
\newcommand{\HH}        {\field H}
\newcommand{\Z}         {\field Z}
\newcommand{\C}         {\field C}
\newcommand{\M}         {\field M}
\newcommand{\Q}         {\field Q}
\renewcommand{\P}         {\field P}

\newcommand{\Si}{\Sigma^{\infty}}
\newcommand{\Oi}{\Omega^{\infty}}

\DeclareMathOperator*{\mor}{mor}
\DeclareMathOperator*{\invlim}{lim}
\DeclareMathOperator*{\colim}{colim}
\DeclareMathOperator*{\thocolim}{thocolim}
\DeclareMathOperator*{\hocolim}{hocolim}
\DeclareMathOperator*{\holim}{holim}
\DeclareMathOperator*{\badhocolim}{badhocolim}
\DeclareMathOperator{\spec}{Spec}
\DeclareMathOperator{\Spec}{Spec}
\DeclareMathOperator{\Hom}{Hom}
\DeclareMathOperator{\Et}{Et}
\DeclareMathOperator{\dgn}{dgn}
\DeclareMathOperator{\Tot}{Tot}
\DeclareMathOperator{\homp}{hom_+}
\DeclareMathOperator*{\shom}{hom}
\DeclareMathOperator{\Map}{Map}
\DeclareMathOperator{\hMap}{hMap}
\DeclareMathOperator{\mapp}{map_*}
\DeclareMathOperator{\End}{End}
\DeclareMathOperator{\LIE}{LIE}
\DeclareMathOperator{\ob}{ob}
\DeclareMathOperator{\Mor}{Mor}
\DeclareMathOperator{\LLP}{LLP}
\DeclareMathOperator{\RLP}{RLP}
\DeclareMathOperator{\sk}{sk}
\DeclareMathOperator{\cosk}{cosk}
\DeclareMathOperator{\coskp}{cosk}
\DeclareMathOperator{\coeq}{coeq}
\DeclareMathOperator{\eq}{eq}
\DeclareMathOperator{\diag}{diag}
\DeclareMathOperator{\Singt}{Sing_t}
\DeclareMathOperator{\Vect}{Vect}
\DeclareMathOperator{\Fact}{Fact}
\DeclareMathOperator{\Extn}{Extn}
\DeclareMathOperator{\sRes}{sRes}
\DeclareMathOperator{\coRes}{coRes}
\DeclareMathOperator{\Ex}{Ex}
\DeclareMathOperator{\hgt}{ht\,}
\DeclareMathOperator{\Th}{Th}
\DeclareMathOperator{\Tor}{Tor}
\DeclareMathOperator{\Cotor}{Cotor}
\DeclareMathOperator{\Ext}{Ext}
\DeclareMathOperator{\Cell}{Cell}
\DeclareMathOperator{\id}{id}
\DeclareMathOperator{\Aut}{Aut}
\DeclareMathOperator{\ev}{ev}

%\newcommand{\Sq}[1]{\mathsf{Sq}^{#1}}

\newcommand{\dexp}[3]{\shom\nolimits^{#2}(#1,#3)}
\newcommand{\hexp}[2]{\shom\nolimits^{\scop}(#1,#2)}
\newcommand{\sexp}[2]{\shom(#1,#2)}

\newcommand{\hocolimd}[1]{\hocolim\nolimits^{#1}}
\newcommand{\hocolimc}{\corhocolim}

\newcommand{\ra}{\rightarrow}                   % right arrow
\newcommand{\lra}{\longrightarrow}              % long right arrow
\newcommand{\la}{\leftarrow}                    % left arrow
\newcommand{\lla}{\longleftarrow}               % long left arrow
\newcommand{\llra}[1]{\stackrel{#1}{\lra}}      % labeled long right
                            % arrow
\newcommand{\llla}[1]{\stackrel{#1}{\lla}}      % labeled long right
                            % arrow
\newcommand{\lequiv}[1]{\underset{\scriptstyle #1}{\sim}}
\newcommand{\wwe}[1]{\underset{\scriptstyle #1}{\simeq}}
\newcommand{\mapiso}{\llra{\cong}}
\newcommand{\we}{\llra{\sim}}                   % weak equivalence
\newcommand{\bwe}{\llla{\sim}}
\newcommand{\cof}{\rightarrowtail}              % cofibration
\newcommand{\fib}{\twoheadrightarrow}           % fibration
\newcommand{\bfib}{\twoheadleftarrow}           % fibration
\newcommand{\trfib}{\stackrel{\sim}{\longfib}}
\newcommand{\trcof}{\stackrel{\sim}{\cof}}
\newcommand{\btrfib}{\stackrel{{\sim}}{\longbfib}}
\newcommand{\btrcof}{\stackrel{{\sim}}{\longbcofib}}
\newcommand{\inc}{\hookrightarrow}              % inclusion
\newcommand{\dbra}{\rightrightarrows}           % double arrow for eqlizer
\newcommand{\ldbra}{\vect{\underrightarrow{\phantom\lra}}}
\newcommand{\eqra}{\llra{\sim}}                 % equivalence/isomorphism

\newcommand{\blank}{-}                          % A hyphen, as in
                            % (-)xV
\newcommand{\hyphen}{-}                          % A hyphen, as in (-)xV
\newcommand{\Id}{Id}                            % The identity functor
\newcommand{\und}{\underline}
\newcommand{\norm}[1]{\mid \! #1 \! \mid}       %\norm{x} gives |x|
\newcommand{\mM}{\underline{\cM}}
\newcommand{\msM}{\underline{s\cM}}
\newcommand{\mMC}{\underline{M}^\cC}
\newcommand{\msSet}{\underline{\sSet}}
\newcommand{\sPreF}{\underline{\sPre}}
\newcommand{\mMap}{\underline{\Map}}
\newcommand{\TopF}{\underline{\ToP}}
\newcommand{\sTopF}{\underline{s\Top}}


% These commands are for the period and comma in the lower right entry of
% a diagram.  They put the punctuation 2 pts to the right, but make
% TeX (and hence the diagram package) unaware of the extra width
% of that entry.
\newcommand{\period}    {{\makebox[0pt][l]{\hspace{2pt} .}}}
\newcommand{\comma}     {{\makebox[0pt][l]{\hspace{2pt} ,}}}
\newcommand{\semicolon} {{\makebox[0pt][l]{\hspace{2pt} ;}}}

\newcommand{\ech}{\text{\Cech}}
\newcommand{\scat}{\Delta}
\newcommand{\scop}{{\Delta^{op}}}
\newcommand{\re}{Re}
\newcommand{\Rea}{\re\,}
\newcommand{\core}{CoRe}
\newcommand{\sing}{Sing}
\newcommand{\Sing}{\sing\,}
\newcommand{\cosing}{CoSing}
\newcommand{\assign}{\mapsto}
\newcommand{\copr}{\,\amalg\,}
\newcommand{\ovcat}{\downarrow}
\newcommand{\bdot}{\bullet}
\newcommand{\restr}[1]{\!\mid_{#1}}
\newcommand{\wig}{\sim}
\newcommand{\fprod}[1]{\underset{#1}{\times}}
\newcommand{\disj}{\bigsqcup}
\newcommand{\bd}[1]{\partial\Delta^{#1}}
\newcommand{\dunion}{{\stackrel{\cdot}{\cup}}}
\newcommand{\constant}{c_*}
\newcommand{\Ucheck}{\check{U}_{\bullet}}
\newcommand{\Ncheck}{\check{N}}
\newcommand{\adjoint}{\rightleftarrows}
\newcommand{\horns}[2]{\Lambda^{#1,#2}}
\newcommand{\bdd}[1]{\partial\Delta^{#1}}
\newcommand{\Lamb}[1]{\Lambda^{#1}}
\newcommand{\del}[1]{\Delta^{#1}}
\newcommand{\delt}[1]{\Delta_{t}^{#1}}
\newcommand{\dels}[1]{\Delta_{s}^{#1}}
\newcommand{\he}{\simeq}
\newcommand{\RR}[1]{{\mathbf{R}}#1}
\newcommand{\pt}{pt}

\newcommand{\hocolims}{\hocolimd{*}}

\newcommand{\Schk}{Sch/k}
\newcommand{\Smk}{Sm_k}
\newcommand{\Sm}{Sm}
\newcommand{\SchC}{Sch/\C}
\newcommand{\SmC}{Sm/\C}
\newcommand{\sshv}{s\text{Shv}}
\newcommand{\MV}{{\cat M}{\cat V}}
\newcommand{\stabMV}{\Spectra(\MV)}

\newcommand{\anglb}[1]{{{<}#1{>}}}

\newcommand{\cons}{c_*}
\newcommand{\wrt}{\text{wrt }}

\newcommand{\fix}{\mbox{}\par\noindent}

\newcommand{\dcoprod}[1]{{\displaystyle\coprod_{#1}}}

\newcommand{\rea}[1]{|{#1}|}             %geometric realization of #1
\newcommand{\map}{\rightarrow}
\newcommand{\ol}{\overline}
\newcommand{\ceck}[1]{\Cech(#1)}         %Cech complex for #1
\newcommand{\oceck}[1]{\Cech^{o}(#1)}    %Ordered Cech complex for #1
\newcommand{\oreal}[1]{\rea{\oceck{U}}}  %Realization of ordered Cech cplex
\newcommand{\creal}[1]{\rea{\ceck{U}}}   %Realization of the Cech complex
\newcommand{\Spc}{\text{Spc}}            %Spc(k); i.e. ``Spaces over k''
\newcommand{\fcup}{\cup}
\newcommand{\Gal}{\text{Gal}}
\newcommand{\stn}{s_{\leq n}}

\newcommand{\Cech}{\check{C}}
\newcommand{\CCech}{\v{C}ech\ }
\newcommand{\mCech}{\text{\CCech}}

\newcommand{\fibr}[1]{{\field H}#1}
\newcommand{\piH}{\pi HC}
\newcommand{\comment}[1]{\textbf{[#1]}}

\newcommand{\CW}{CW}
\DeclareMathOperator{\Gr}{Gr}
\newcommand{\V}[2]{V_{#1}(\A^{#2})}
\newcommand{\F}{\mathbb{F}}
\newcommand{\vectr}[1]{\mathbf{#1}}
\newcommand{\Fl}{\text{Fl}}
\newcommand{\GL}{GL}
\newcommand{\power}[2]{\und{#1}^{#2}}

\newcommand{\Sp}{\Spectra}
\newcommand{\MotSp}{\Motspectra}
\newcommand{\tre}{t_{re}}
\newcommand{\cl}{\mathrm{cl}}
\numberwithin{equation}{subsection}

% The following code corrects the problem with equation numbering.

\renewcommand{\theequation}{\thesubsection}

\newenvironment{myequation}
  {\addtocounter{subsection}{1}\begin{eqnarray}}
  {\end{eqnarray}$\!\!$}

\newcommand{\PS}{P\cS}
\newcommand{\HC}{HC}
\newcommand{\ch}{\check{\cC}}
\renewcommand{\S}{\mathbb{S}}
\DeclareMathOperator{\Sq}{Sq}

\newcommand{\cyl}{cyl}
\newcommand{\Ccyl}{C_{cyl}}
\newcommand{\jC}{\mathfrak{C}}
\newcommand{\jD}{\mathfrak{D}}
\newcommand{\jCn}{\mathfrak{C}^{nec}}
\newcommand{\jCh}{\mathfrak{C}^{hoc}}
\newcommand{\swedge}{\bar{\wedge}}
\newcommand{\Nec}{{\cat{N}ec}}
\newcommand{\PNec}{{\cat{P}Nec}}
\newcommand{\Necf}{{\cat{N}ec^{f}}}
\DeclareMathOperator{\Spi}{Spi}
\newcommand{\Dao}{\Delta_{\alpha,\omega}}
\newcommand{\Spao}{\cat{S}pi_{\alpha,\omega}}
\DeclareMathOperator{\Wcofib}{Wcofib}
\DeclareMathOperator{\Wfib}{Wfib}
\DeclareMathOperator{\W}{W}

\newcommand{\longnote}[2][4.9in]{\noindent\fcolorbox{black}{yellow}{\parbox{#1}{\color{black} #2}}}
\newcommand{\shortnote}[1]{\fcolorbox{black}{yellow}{\color{black} #1}}
\newcommand{\nosee}[1]{}
\newcommand{\raj}{//}
\def\tn{\textnormal}
\def\im{\tn{im}}
\def\Ob{\tn{ob}}
\def\jCL{\jC^\tn{Lur}}
\def\id{\tn{id}}
\def\to{\ra}
\def\To{\xrightarrow}
\def\too{\longrightarrow}
\def\fromm{\longleftarrow}
\def\from{\leftarrow}
\def\From{\xleftarrow}
\def\vect{\overrightarrow}
\def\taking{\colon}
\def\sCat{{s\Cat}}
\def\poleq{\preceq}
\def\inj{\hookrightarrow}
\def\cross{\times}
\def\wt{\widetilde}
\def\bD{{\bf \Delta}}
\newcommand{\Adjoint}[4]{\xymatrix@1{#2 \ar@<.5ex>[r]^-{#1} & #3 \ar@<.5ex>[l]^-{#4}}}
\newcommand{\bpSet}{\sSet_{*,*}}
\def\ss{\subseteq}

\begin{document}

\title{Rigidification of quasi-categories}

\author{Daniel Dugger}
\author{David I. Spivak}

\address{Department of Mathematics\\ University of Oregon\\ Eugene, OR
97403}

\address{Department of Mathematics\\ University of Oregon\\ Eugene, OR
97403}

\email{ddugger@uoregon.edu}

\email{dspivak@uoregon.edu}


\begin{abstract}
We give a new construction for rigidifying a quasi-category into a
simplicial category, and prove that it is weakly equivalent to the
rigidification given by Lurie.  Our construction comes from the use of
necklaces, which are simplicial sets obtained by stringing simplices
together.  As an application of these methods, we use our model to
reprove some basic facts from \cite{L} about the rigidification
process.
\end{abstract}



\maketitle

\tableofcontents

\section{Introduction}
Quasi-categories are a certain generalization of categories, where in addition to the usual $1$-morphisms one  has $n$-morphisms for every natural
number $n$.  They are special among higher categories in that they
have the property that for $n>1$ the $n$-morphisms are all invertible
in an appropriate sense.   Quasi-categories have been extensively
studied by Cordier and Porter \cite{CP}, by Joyal \cite{J1},
\cite{J2}, and by Lurie \cite{L}.  If $K$ is a quasi-category and $x$
and $y$ are two objects of $K$, then one may associate a ``mapping
space'' $K(x,y)$ which is a simplicial set.  There are many different
constructions for these mapping spaces, but in \cite{L} one particular
model is given for which there are composition maps $K(y,z)\times
K(x,y)\ra K(x,z)$ giving rise to a simplicial category.  This
simplicial category is denoted $\jC(K)$, and it may be thought of as a
{\it rigidification\/} of the quasi-category $K$.  It is proven in
\cite{L} that the homotopy theories of quasi-categories and simplicial
categories are equivalent via this functor.



In this paper we introduce some new models for the mapping spaces
$K(x,y)$, which are particularly easy to describe and particularly
easy to use---in fact they are just the nerves of ordinary categories
(i.e., 1-categories).  Like Lurie's model, our models admit
composition maps giving rise to a simplicial category; so we are
giving a new method for rigidifying quasi-categories.  We prove that
our construction is homotopy equivalent (as a simplicial category) to
Lurie's $\jC(K)$.  Moreover, because our mapping spaces are
nerves of categories there are many standard tools available for
analyzing their homotopy types.  We demonstrate the effectiveness of
this by giving new proofs of some basic facts about the functor
$\jC(\blank)$.

One payoff of this approach is that it is possible to give a streamlined proof
of Lurie's Quillen equivalence between the homotopy theory of
quasi-categories and simplicial categories.  This requires, however, a
more detailed study of the model category structure on
quasi-categories.  We will take this up in a sequel \cite{DS} and prove the
Quillen equivalence there.


\medskip

\subsection{Mapping spaces via simplicial categories}\label{subsec:mapping in sCat}
Now we describe our results in more detail.  A quasi-category
is a simplicial set that has the right-lifting-property 
with respect to inner horn inclusions $\Lambda^n_i\to\Delta^n,\;\;\;
0<i<n$.  It turns out that there is a unique model structure on
$\sSet$ where the cofibrations are the monomorphisms and the fibrant
objects are the quasi-categories; this will be called the {\it Joyal
model structure\/} and denoted $\sSet_J$.  The weak equivalences in
$\sSet_J$ will here be called {\it Joyal equivalences\/} (they are
called
``weak categorical equivalences'' in \cite{J2}).  The
existence of the Joyal model structure will not be needed in this
paper, although it provides some useful context.  The notions of
quasi-categories and Joyal equivalences, however, will be used in
several places.  See 
Section~\ref{subsec:quasi-categories} for additional background.


There is a functor, constructed in \cite{L}, which sends any
simplicial set $K$ to a corresponding simplicial category $\jC(K)\in\sCat$.  
This is the left adjoint in a Quillen pair
\[ \jC \colon \sSet_J \adjoint s\Cat\colon N, \] where $N$ is called
the \dfn{coherent nerve}.  
To avoid later confusion, we note that if a
simplicial category $\cD$ has discrete mapping spaces---that is, if it
is just an ordinary category---then $N\cD$ coincides with the
classical nerve construction.
The functor $N$ can be described quite
explicitly (see Section~\ref{se:background}), but the functor $\jC$ is
in comparison a little mysterious.  In \cite{L} each $\jC(K)$ is
defined as a certain colimit in the category $s\Cat$, but colimits in
$\sCat$ are notoriously difficult to understand.  


Our main goal in this paper is to give a different model for the
functor $\jC$. Define a \dfn{necklace} (which we picture as
``unfastened'') to be a simplicial set of the form
\[ \Delta^{n_0}\Wedge\Delta^{n_1}\Wedge \cdots \Wedge\Delta^{n_k}\]
where each $n_i\geq 0$ and where in each wedge the final vertex of
$\Delta^{n_i}$ has been glued to the initial vertex of
$\Delta^{n_{i+1}}$.  Necklaces were used by Baues in his study of loop spaces \cite{Ba}.

The first and last vertex in any necklace $T$ are denoted $\alpha_T$
and $\omega_T$, respectively (or just $\alpha$ and $\omega$ if $T$ is
obvious from context).  If $S$ and $T$ are two necklaces, then by
$S\Wedge T$ we mean the necklace obtained in the evident way, by
gluing the final vertex $\omega_S$ of $S$ to the initial vertex
$\alpha_T$ of $T$.  Write $\Nec$ for the category whose objects are
necklaces and where a morphism is a map of simplicial sets which preserves the
initial and final vertices.  

Let $S\in \sSet$ and let $a,b\in S_0$.  If $T$ is a necklace, we use
the notation $$T\to S_{a,b}$$ to indicate a morphism of simplicial
sets $T\ra S$ which sends $\alpha_T$ to $a$ and $\omega_T$ to $b$.  Let
$(\Nec\ovcat S_{a,b})$ denote the evident category whose objects are
pairs $[T,T\ra S_{a,b}]$ where $T$ is a necklace.  Note that for
$a,b,c\in S_0$ there is a functor
\begin{align}\label{dia:composition} (\Nec\ovcat S_{b,c})\times (\Nec\ovcat S_{a,b}) \lra (\Nec\ovcat
S_{a,c})
\end{align}
which sends the pair $([T_2,T_2\ra S_{b,c}],[T_1,T_1\ra S_{a,b}])$ to
$[T_1\Wedge T_2,T_1\Wedge T_2\ra S_{a,c}]$.

Let $\jCn(S)$ be the function which assigns to any $a,b\in S_0$ the
simplicial set $\jCn(S)(a,b)=N(\Nec\ovcat S_{a,b})$ (the classical nerve
of the $1$-category $(\Nec\ovcat S_{a,b})$).  The above pairings of
categories induces pairings on the nerves, which makes $\jCn(S)$ into
a simplicial category with object set $S_0$.  

\begin{thm}
\label{th:main1}
There is a natural zig-zag of weak equivalences of simplicial
categories between $\jCn(S)$
and $\jC(S)$, for all simplicial sets $S$.
\end{thm}

In the above result, the weak equivalences for simplicial categories
are the so-called ``DK-equivalences'' used by Bergner in
\cite{Bergner} and first defined in \cite{DK}. 
 See Section~\ref{se:background} for this notion.


In this paper we also give an explicit description of the mapping
spaces in the simplicial category $\jC(S)$.  A rough statement is
given below, but see Section~\ref{sec:jC} for more details.

\begin{thm}
Let $S$ be a simplicial set and let $a,b\in S$.  Then the mapping
space $X=\jC(S)(a,b)$ is the simplicial set whose $n$-simplices
are triples subject to a certain equivalence relations.  The triples
consist of a necklace $T$, a map $T\ra S_{a,b}$, and a flag
$\vect{T}=\{ T^0\subseteq\cdots\subseteq T^n \}$ of vertices in $T$.  
For the equivalence relation, see Corollary~\ref{co:description1}.
The face maps and
degeneracy maps are obtained by removing or repeating elements $T^i$
in the flag.

\par\noindent
The pairing
\[ \jC(S)(b,c) \times \jC(S)(a,b) \lra \jC(S)(a,c) 
\]
sends the pair of $n$-simplices $([T\ra S; \vect{T^i}],[U\ra S,\vect{U^i}])$ to 
$[U\Wedge T\ra S,\vect{U^i\cup T^i}]$.
\end{thm}


Theorem~\ref{th:main1} turns out to be very useful in the study of the
functor $\jC$.  There are many tools in classical homotopy theory for
understanding the homotopy types of nerves of $1$-categories, and via
Theorem~\ref{th:main1} these tools can be applied to understand mapping
spaces in $\jC(S)$.  We demonstrate this technique in
Section~\ref{se:properties} by proving, in a new way, the following
two properties of $\jC$ found in \cite{L}.

\begin{thm} Let $X$ and $Y$ be simplicial sets.
\begin{enumerate}[(a)]
\item The natural map
$\jC(X\times Y) \ra \jC(X)\times \jC(Y)$ is a weak equivalence of
simplicial categories;
\item If $X\ra Y$ is a Joyal equivalence  then $\jC(X)\ra
\jC(Y)$ is a weak equivalence.
\end{enumerate}
\end{thm}


%\subsection{Relation with the work of Lurie}
%In this paper we take as a given the Joyal model structure on $\sSet$,
%and from there we develop the properties of mapping spaces and
%categorification.  In Lurie's book \cite{L} he takes a different
%approach, where he starts by developing the properties of mapping
%spaces and categorification and then proves the existence of the Joyal
%model structure as a consequence of this work.  His methods involve a
%detailed and lengthy study of what he calls ``straightening and
%unstraightening'' functors, and it was a vague disatisfaction with
%this material---together with the hope of avoiding it---that first
%led us to the work in the present paper.
%
%Due to the inherent differences in the two approaches, it is slightly
%awkward for us to quote results from \cite{L} without creating
%confusions and possible circularities.  Because of this, there are a
%few minor results whose proofs we end up repeating or
%redoing in a slightly different way.  The result is that the present
%paper can be read independently of \cite{L}---although this should not
%be taken as a denial of the  intellectual debt we owe to that
%work.

\subsection{Notation and Terminology}
We will sometimes use $\sSet_K$ to refer to the usual model structure
on simplicial sets, which we'll term the {\it Kan model structure\/}.
The fibrations are the Kan fibrations, the weak equivalences (called
Kan equivalences from now on) are the maps which induce homotopy
equivalences on geometric realizations, and the cofibrations are the
monomorphisms.

We will often be working with the category $\bpSet=(\bdd{1}\ovcat\sSet)$.
Note that $\Nec$ is a
full subcategory of $\bpSet$.

An object of $\bpSet$ is a simplicial set $X$ with two distinguished
points $a$ and $b$.  We sometimes (but not always) write $X_{a,b}$ for $X$, to remind
us that things are taking place in $\bpSet$ instead of $\sSet$.  

If $\cC$ is a (simplicial) category containing objects $X$ and $Y$, we
write $\cC(X,Y)$ for the (simplicial) set of morphisms from $X$ to $Y$.

\subsection{Acknowledgements}
The first author was supported by NSF grant  DMS-0604354.  The second
author was supported by ONR grant N000140910466.  
We are grateful to Emily Riehl for
comments on an earlier version of this paper.

%%%%%%%%%%%%%%%%%%%%%%%%%%%%%%%%%%%%%%%%%%%%%%%%%%%%%%%%%%%%%%%%%%%%%
\section{Background on quasi-categories}
\label{se:background}

In this section we give the background on quasi-categories and
simplicial categories needed in the rest of the paper.

\subsection{Simplicial categories}
A simplicial category is a category enriched over simplicial sets; it
can also be thought of as a simplicial object of $\Cat$ in which the
simplicial operators are all equal to the identity on the object sets.
We use $\sCat$ to denote the category of simplicial categories.  Given
a simplicial category $\cD$, let $\pi_0\cD$ denote the ordinary
category having the same object set, and where the set of morphisms from $x$ to $y$ is $\pi_0\bigl ( \cD(x,y) \bigr )$; as this construction is functorial in $\cD$, we regard $\pi_0\taking\sCat\to\Cat$ as a functor.


A cofibrantly-generated
model structure on
$\sCat$ was developed in \cite{Bergner}.  In this structure a map of simplicial
categories $F\colon\cC \ra \cD$ is a weak equivalence (sometimes called a
{\it DK-equivalence\/}) if
\begin{enumerate}[(1)]
\item For all $a,b\in \ob\cC$, the map $\cC(a,b) \ra \cD(Fa,Fb)$ is a
Kan equivalence of simplicial sets; 
\item The induced functor of ordinary categories $\pi_0 F\colon \pi_0
\cC \ra \pi_0\cD$ is surjective on isomorphism classes of objects.  
\end{enumerate}  
Likewise, the map $F$ is a fibration if
\begin{enumerate}[(1)]
\item For all $a,b\in \ob\cC$, the map $\cC(a,b) \ra \cD(Fa,Fb)$ is a
Kan fibration of simplicial sets;
\item For all $a \in \ob\cC$ and $b\in \ob\cD$, if $e\colon Fa \ra b$
is a map in $\cD$ which becomes an isomorphism in $\pi_0\cD$, then
there is an object $b'\in \cC$ and a map $e'\colon a\ra b'$ such that
$F(e')=e$ and $e'$ becomes an isomorphism in $\pi_0\cC$.
\end{enumerate}  
The cofibrations are the maps which have the
left lifting property with respect to the acyclic fibrations.

\begin{remark}
The second part of the fibration condition seems a little awkward at
first.  In this paper we will actually have no need to think about fibrations
of simplicial categories, but have included the definition for
completeness.  

Bergner writes down sets of generating cofibrations and acyclic
cofibrations in \cite{Bergner}. 
\end{remark}

\subsection{Quasi-categories and Joyal equivalences}\label{subsec:quasi-categories}
As mentioned in the introduction, 
there is a unique model structure on
$\sSet$ with the properties that
\begin{enumerate}[(i)]
\item The cofibrations are the monomorphisms;
\item The fibrant objects are the quasi-categories.
\end{enumerate}
%
It is easy to see that there is at most one such structure.  To do
this, for any set $S$ let $\cE(S)$ be the groupoid with object set $S$
having the property that for any two
objects $s_1,s_2\in S$ there is a unique map $s_1\to s_2$.  Let $E^1$ be the nerve of the groupoid $\cE(\{0,1\})$.  The
$k$-simplices of $E^1$ may be identified with $(k+1)$-tuples
consisting of $0$s and $1$s, where the face and degeneracy maps are
the usual deletion and repetition of entries.  Note that the geometric
realization of $E^1$ is essentially the standard model for $S^\infty$.
One may also describe $E^1$ as the $0$-coskeleton---see \cite{AM}, for
instance---of the set $\{0,1\}$.  If $X$ is any simplicial set then
maps $X\ra E^1$ are in bijective correspondence with set maps $X_0\ra
\{0,1\}$.

Assume that a model structure on $\sSet$ exists having properties (i)
and (ii) above.
The map
$E^1\ra *$ has the right lifting property with respect to all
monomorphisms, and so it will be an acyclic fibration in this
structure.  Therefore $X\times E^1\ra X$ is also an acyclic fibration
for any $X$, and hence $X\times E^1$ will be a cylinder object for
$X$.  Since every object is cofibrant, a map $A\ra B$ will be a weak
equivalence if and only if it induces bijections $[B,Z]_{E^1} \ra
[A,Z]_{E^1}$ for every quasi-category $Z$, where $[A,Z]_{E^1}$ means
the coequalizer of $\sSet(A\times E^1,Z)\dbra \sSet(A,Z)$.  Therefore
the weak equivalences are determined by properties (i)--(ii), and
since the cofibrations and weak equivalences are determined so are the
fibrations.  Thus, such a model category structure will be unique.

Motivated by the above discussion, we define a map of simplicial sets $A\ra B$ to be a
\dfn{Joyal equivalence} if it induces bijections
$[B,Z]_{E^1}\to[A,Z]_{E^1}$ for every quasi-category $Z$.

That there actually exists a model stucture satisfying (i) and (ii) is
not so clear, but it was established by Joyal.  See \cite{J1} or
\cite{J2}, \cite{L} for another proof (particularly Theorems 2.2.5.1
and 2.4.6.1), or the appendix of \cite{DS}
for a compact presentation of Joyal's arguments.  We will call this
the \dfn{Joyal model structure} and denote it by $\sSet_J$.  The weak
equivalences are defined differently in both \cite{J2} and \cite{L},
but of course turn out to be equivalent to the definition we have
adopted here.

In the rest of the paper we will never use the Joyal model structure,
only the notion of Joyal equivalence.

\subsection{Background on $\jC$ and $N$}

Given a simplicial category $S$, one can
construct a simplicial set called the {\it coherent
nerve\/} of $S$ \cite{C}, \cite[1.1.5]{L}.  We will now describe this construction.

Recall the adjoint functors $F\colon \Grph \adjoint \Cat\colon U$.
Here $\Cat$ is the category of (small) $1$-categories, and $\Grph$ is the
category of directed graphs: a graph consists of a vertex set $X_0$, an edge
set $X_1$, and two maps $X_1\dbra X_0$.  The functor $U$ is the evident 
forgetful functor that takes a category $C$ to $\Mor(C)\dbra \ob(C)$,
and $F$ is a free functor that adds in formal compositions.  See  
\cite[Chapter II.7]{ML}.    

Given any category $\cC$ we may then
consider the comonad resolution $(FU)_\bullet(\cC)$ given by
$[n]\mapsto (FU)^{n+1}(\cC)$.  This is a simplicial category.
There is a functor of simplicial categories $(FU)_\bullet(\cC) \ra
\cC$ (where the latter is considered a discrete simplicial category).
This functor induces a weak equivalence on all mapping spaces, a fact
which can be seen by applying $U$, at which point the comonad
resolution picks up a contracting homotopy.  Note that this means that
the simplicial mapping spaces in $(FU)_\bullet(\cC)$ are all homotopy
discrete.

Recall that $[n]$ denotes the category $0\ra 1 \ra \cdots \ra n$, where
there is a unique map from $i$ to $j$ whenever $i\leq j$.  
We let $\jC(\Delta^n)$ denote the simplicial category
$(FU)_{\bullet}([n])$.  The mapping spaces in this simplicial category
can be analyzed completely, and are as follows.  For each $i$ and $j$,
let $P_{i,j}$ denote the poset of all subsets of $\{i,i+1,\ldots,j\}$
containing $i$ and $j$ (ordered by inclusion).  Note that the nerve of
$P_{i,j}$ is isomorphic to the cube $(\Delta^1)^{j-i-1}$ if $j>i$,
$\Delta^0$ if $j=i$, and the emptyset if $j<i$.  The nerves of the $P_{i,j}$'s
naturally form the mapping spaces of a simplicial category with object
set $\{0,1,\ldots,n\}$, using the pairings $P_{j,k}\times P_{i,j} \ra
P_{i,k}$ given by union of sets.  

\begin{lemma}
\label{le:C(Delta^n)}
There is an isomorphism of simplicial categories $\jC(\Delta^n)\iso
NP$.  
\end{lemma}


\begin{remark}
The proof of the above lemma is a bit of an aside from the main thrust
of the paper, so it is given in Appendix~\ref{se:appendix}.  In fact
we could have {\it defined\/} $\jC(\Delta^n)$ to be $NP$, which is
what Lurie does in \cite{L}, and avoided the lemma entirely; the
construction $(FU)_\bullet([n])$ will never again be used in this
paper.  Nevertheless, the identification of $NP$ with
$(FU)_{\bullet}([n])$ demonstrates that $NP$ arises naturally in this setting.
\end{remark}

For any simplicial category $\cD$, the \dfn{coherent nerve} of $\cD$
is the simplicial set $N\cD$ given by
\[ [n]\mapsto s\Cat(\jC(\Delta^n),\cD).\]
It was proven by Lurie \cite{L} that if every mapping space in $\cD$ is a Kan complex, then $N\cD$ is a quasi-category; see also Lemma~\ref{lemma:N preserves fibrants} below.

The functor $N$ has a left adjoint.  If we temporarily denote this by
$\jC'$, note that there are canonical bijections
\[ s\Cat(\jC'(\Delta^n),\cD)\iso \sSet(\Delta^n,N\cD)=
s\Cat(\jC(\Delta^n),\cD).\]
It follows from this that $\jC'(\Delta^n)\iso \jC(\Delta^n)$.  It
therefore makes sense to denote the left adjoint of $N$ as just
$\jC\colon \sSet\ra s\Cat$.

Any simplicial set $K$ may be written as a colimit of
simplices via the formula
\[ K\iso \colim_{\Delta^n\ra K} \Delta^n, 
\]
and consequently one has
\begin{align}\label{dia:jC colim} \jC(K)\iso \colim_{\Delta^n\ra K} \jC(\Delta^n)
\end{align}
where the colimit takes place in $s\Cat$.
This formula is a bit unwieldy, however, in the sense that it
does not give much concrete information about the mapping
spaces in $\jC(K)$.  The
point of the next three sections is to obtain such concrete
information, via the use of necklaces.
%%%%%%%%%%%%%%%%%%%%%%%%%%%%%%%%%%%%%%%%%%%%%%%%%%%%%%%%%%%%%%%%%%%%%%

\section{Necklaces}
\label{se:necklaces}

A necklace is a simplicial set obtained by stringing simplices
together in succession.  In this section we establish some basic facts
about them, as well as facts about the more general category of
ordered simplicial sets.  When $T$ is a necklace we are able to give a
complete description of the mapping spaces in $\jC(T)$ as nerves of
certain posets, generalizing what was said for $\jC(\Delta^n)$ in the
last section.  See Proposition~\ref{prop:mapping in necklaces}.  

\medskip

As briefly discussed in the introduction, a \dfn{necklace} is defined to be a simplicial set of the form
\[ \Delta^{n_0}\Wedge\Delta^{n_1}\Wedge \cdots \Wedge\Delta^{n_k}\]
where each $n_i\geq 0$ and where in each wedge the final vertex of
$\Delta^{n_i}$ has been glued to the initial vertex of
$\Delta^{n_{i+1}}$.  We say that the necklace is in \dfn{preferred
form} if either $k=0$ or each $n_i\geq 1$.

Let $T=\Delta^{n_0}\Wedge\Delta^{n_1}\Wedge \cdots \Wedge\Delta^{n_k}$
be in preferred form.  Each $\Delta^{n_i}$ is called a \dfn{bead} of
the necklace.  A \dfn{joint} of the necklace is either an initial or a
final vertex in some bead.  Thus, every necklace has at least one
vertex, one bead, and one joint; $\Delta^0$ is not a bead in any
necklace except in the necklace $\Delta^0$ itself.

Given a necklace $T$, write $V_T$ and $J_T$ for the sets of vertices
and joints of $T$.  Note that $V_T=T_0$ and $J_T\subseteq V_T$.  Both $V_T$ and
$J_T$ are totally ordered, by saying $a\leq b$ if there is a directed
path in $T$ from $a$ to $b$.  The initial and final vertices of $T$ are denoted $\alpha_T$ and $\omega_T$ (and we sometimes drop the subscript); note that $\alpha_T,\omega_T\in J_T$.

Every necklace $T$ comes with a particular map $\bdd{1}\ra T$ which
sends $0$ to the initial vertex of the necklace, and $1$ to the final
vertex.  If $S$ and $T$ are two necklaces, then by $S\Wedge T$ we mean
the necklace obtained in the evident way, by gluing the final vertex
of $S$ to the initial vertex of $T$.  Let $\Nec$ denote the full
subcategory of $\bpSet=(\bdd{1}\ovcat\sSet)$ whose objects are
necklaces $\bdd{1}\to T$.  By a \dfn{map of necklaces} we mean a map in this
category.  We will sometimes talk about $\Nec$ as though it
is a subcategory of $\sSet$.

A simplex is a necklace with one bead.  A \dfn{spine} is a necklace in
which every bead is a $\Delta^1$.  Every necklace $T$ has an
associated simplex and spine, which we now define.  Let $\Delta[T]$ be
the simplex whose vertex set is the same as the (ordered) vertex set
of $T$.  Likewise, let $\Spi[T]$ be the longest spine inside of $T$.
Note that there are inclusions $\Spi[T]\inc T\inc \Delta[T]$.  The
assignment $T\ra \Delta[T]$ is a functor, but $T\ra \Spi[T]$ is not
(for instance, the unique map of necklaces $\Delta^1\to\Delta^2$ does
not induce a map on spines).


\subsection{Ordered simplicial sets}
If $T\ra T'$ is a map of necklaces, then the image of $T$ is also a
necklace.  To prove this, as well as for several other reasons
scattered thoughout the paper, it turns out to be very convenient to
work in somewhat greater generality.  

If $X$ is a simplicial set, define a relation on its $0$-simplices by
saying that $x\poleq y$ if there exists a spine $T$ and a map $T\ra
X$ sending $\alpha_T\mapsto x$ and $\omega_T\mapsto y$.  In other words,
$x\poleq y$ if there is a directed path from $x$ to $y$ inside of
$X$.  Note that this relation is clearly reflexive and transitive, but
not necessarily antisymmetric: that is, if $x\poleq y$ and $y\poleq x$
it need not be true that $x=y$.  In cases where we are considering
different simplicial sets $X$ and $Y$, we will write $\poleq_X$ and
$\poleq_Y$ to distinguish the relations.

\begin{defn}
\label{de:ordered}
A simplicial set $X$ is \dfn{ordered} if
\begin{enumerate}[(i)]
\item The relation $\poleq$ defined on $X_0$ is antisymmetric, and
\item A simplex $x\in X_n$ is determined by its
sequence of vertices $x(0)\poleq\cdots\poleq x(n)$; i.e. no two
distinct $n$-simplices have identical vertex sequences.
\end{enumerate} 

\end{defn}

Note the role of degenerate simplices in condition (ii).  For example,
notice that $\Delta^1/\bdd{1}$ is not an ordered simplicial set.


\begin{lemma}\label{lemma:facts on ordered}
Let $X$ and $Y$ denote ordered simplicial sets
and let $f\colon X\ra Y$ be a map.
\begin{enumerate}
\item The category of ordered simplicial sets is closed under taking
finite limits.
\item Every necklace is an
ordered simplicial set.
\item If $X'\subseteq X$ is a simplicial subset, then $X'$ is also
ordered.
\item The map $f$ is completely determined by the map $f_0\taking X_0\to
Y_0$ on vertices.
\item If $f_0$ is injective then so is $f$.
\item The image of an $n$-simplex $x\taking\Delta^n\to X$ is of the
form $\Delta^k\inj X$ for some $k\leq n$.
\item If $T$ is a necklace and $y\colon T\to X$ is a map, then its
image is a necklace.

\end{enumerate}

\end{lemma}

\begin{proof}
For (1), the terminal object is a point with its unique
ordering.  Given a diagram of the form
$$X\too Z\fromm Y,$$ 
let $A=X\cross_Z Y$.  It is clear  that if $(x,y)\poleq_A(x',y')$ then both $x\poleq_Xx'$
and $y\poleq_Yy'$ hold, and so antisymmetry of $\poleq_A$ follows from
that of $\poleq_X$ and $\poleq_Y$.  Condition (ii) from
Definition~\ref{de:ordered} is easy to check.

Parts (2)--(5) are easy, and left to the reader.

For (6), the sequence $x(0),\ldots,x(n)\in X_0$ may have
duplicates; let $d\taking \Delta^k\to\Delta^n$ denote any face such
that $x\circ d$ contains all vertices $x(j)$ and has no duplicates.
Note that $x\circ d$ is an injection by (5).  A certain degeneracy of $x\circ
d$ has the same vertex sequence as $x$.  Since $X$ is ordered, $x$ is this
degeneracy of $x\circ d$.  Hence, $x\circ d\taking\Delta^k\inj X$ is
the image of $x$.

Claim (7) follows from (6).  


\end{proof}

The following notion is also useful:

\begin{defn}
Let $A$ and $X$ be ordered simplicial sets.  A map $A\ra X$ is called
a \dfn{simple inclusion} if it has the right lifting property with
respect to the canonical inclusions $\bd{1}\inc T$ for all necklaces
$T$.  (Note that such a map really is an inclusion using Lemma \ref{lemma:facts on ordered}(5), because it has the lifting property for $\bd{1}\ra \Delta^0$).
\end{defn}

It is sometimes useful to think of a map $T\ra X$, where $T$ is a
necklace, as being a ``generalized path'' in $X$.  The notion of simple
inclusion says if there is such a ``generalized path'' in $X$ whose endpoints
are both in $A$, then it must lie entirely within $A$.  As a simple exercise,
the reader might check that four out of the five inclusions
$\Delta^1\inj\Delta^1\cross\Delta^1$ are simple inclusions.

\begin{lemma}
\label{le:simple}
A simple inclusion $A\inc X$ of ordered simplicial sets has the right lifting property with
respect to the maps $\bd{k}\inc \Delta^k$ for all $k\geq 1$.  
\end{lemma}

\begin{proof}
Suppose given a square
\[ \xymatrix{ \bd{k}\ar[r]\ar[d] & A \ar[d] \\
 \Delta^k \ar[r] & X.
}
\]
By restricting the map $\bd{k}\ra A$ to $\bd{1}\inc \bd{k}$ (given by
the initial and final vertices of $\bd{k}$), we get a corresponding
lifting square with $\bd{1}\inc \del{k}$.  Since $A\ra X$ is a simple
inclusion, this new square has a lift $l\colon \Delta^k \ra A$.  It is
not immediately clear that $l$ restricted to $\bd{k}$ equals our
original map, but the two maps are equal after composing with $A\ra X$;
since $A\ra X$ is a monomorphism, the two maps are themselves equal.    
\end{proof}

\begin{lemma}\label{lemma:pushout of simple inclusion}

Suppose that $X\from A\to Y$ is a diagram of ordered simplicial sets, and both $A\to X$
and $A\to Y$ are simple inclusions.   Then the pushout
$B=X\amalg_AY$ is an ordered simplicial set, and the inclusions $X\inc B$ and $Y\inc
B$ are both simple.

\end{lemma}

\begin{proof}

We first show that the maps $X\inc B$ and $Y\inc B$ have the
right-lifting-property with respect to $\bd{1}\inc T$ for all
necklaces $T$.   To see this, suppose that $u,v\in X$ are vertices,
$T$ is a necklace, and $f\taking T\ra B_{u,v}$ is a map; we want to
show that $f$ factors through $X$.  Note that any simplex $\Delta^k\to
B$ either factors through $X$ or through $Y$.  Suppose that $f$ does
not factor through $X$.  From the set of
beads of $T$ which do not factor through $X$, take 
any maximal subset $T'$ in which all the beads are
adjacent.  Then we have a necklace $T'\ss T$ such that $f(T')\ss
Y$. 
If there exists a bead in $T$ prior to $\alpha_{T'}$, then it must map
into $X$ since $T'$ was maximal; therefore 
$f(\alpha_{T'})$ would lie in  $X\cap Y=A$.  Likewise, if
there is no bead prior to $\alpha_{T'}$ then $f(\alpha_{T'})=u$ and so
again $f(\alpha_{T'})$ lies in $X\cap Y=A$.  Similar remarks apply to
show that $f(\omega_{T'})$ lies in $A$.  At this point the fact that
$A\inc Y$ is a simple inclusion implies that 
$f(T')\ss A\ss X$, which is a contradiction.  So in fact $f$ factored
through $X$.


We have shown that $X\inc B$ (and dually $Y\inc B$) has the
right-lifting-property with respect to maps $\bd{1}\inc T$, for $T$ a
necklace.  Now we show that $B$ is ordered, and this will complete the
proof.  So suppose $u,v\in B$ are such that $u\poleq v$ and $v\poleq
u$.  There there are spines $T$ and $U$ and maps $T\ra B_{u,v}$, $U\ra
B_{v,u}$.  Consider the composite spine $T\Wedge U \ra B_{u,u}$.  If
$u\in X$, then by the proven right-lifting-property for $X\inc B$ 
it follows that
the image of $T\Wedge U$ maps entirely into $X$; so $u\poleq_X v$ and
$v\poleq_X u$, which means $u=v$ because $X$ is ordered.  The same
argument works if $u\in Y$, so this verifies antisymmetry of
$\poleq_B$.

To verify condition (ii) of Definition~\ref{de:ordered}, suppose
$p,q\taking\Delta^k\to B$ are $k$-simplices with the same sequence of
vertices; we wish to show $p=q$.
We know that $p$ factors through $X$ or $Y$, and so does $q$; if both
factor through $Y$, then the fact that $Y$ is ordered implies that
$p=q$ (similarly for $X$).  So we may assume $p$ factors through $X$
and $q$ factors through $Y$.  By induction on $k$, the restrictions of
$p$ and $q$ to any proper face of $\Delta^k$ are equal; therefore
$p|_{\bdd{k}}$ and $q|_{\bdd{k}}$ are equal, hence they factor through $A$.  By
Lemma~\ref{le:simple} applied to $A\inc X$, the map $p$ factors
through $A$.  Therefore it also factors through $Y$, and now we are
done because $q$ also factors through $Y$ and $Y$ is ordered.
\end{proof}

\subsection{Rigidification of necklaces}

Let $T$ be a necklace.  Our next goal is to give a complete
description of the simplicial category $\jC(T)$.  The object set of
this category is precisely $T_0$. 

For vertices $a,b\in T_0$, let $V_T(a,b)$ denote the set of vertices
in $T$ between $a$ and $b$, inclusive (with respect to the relation
$\poleq$).  Let $J_T(a,b)$ denote the union of $\{a,b\}$ with the set
of joints between $a$ and $b$.  There is a unique subnecklace of $T$ with joints
$J_T(a,b)$ and vertices $V_T(a,b)$; let
$\wt{B}_0,\wt{B}_1,\ldots\wt{B}_k$ denote its beads.  There are
canonical inclusions of each $\wt{B}_i$ to $T$.  Hence, there is a
natural map
\[
\jC(\wt{B}_{k})(j_k,b)
\times \jC(\wt{B}_{k-1})(j_{k-1},j_k) \times \cdots \times 
\jC(\wt{B}_1)(j_1,j_2) \times \jC(\wt{B}_0)(a,j_1)
\ra \jC(T)(a,b)
\]
obtained by first including the $\wt{B}_i$'s into $T$ and then using
the composition in $\jC(T)$ (where $j_i$ and $j_{i+1}$ are the joints
of $\wt{B}_i$).  We will see that this map is an isomorphism.  Note that
each of the sets $\jC(\wt{B}_i)(\blank,\blank)$ has an easy
description, as in Lemma \ref{le:C(Delta^n)}; from this one may
extrapolate a corresponding description for $\jC(T)(\blank,\blank)$,
to be explained next.



Let $C_T(a,b)$ denote the poset 
whose elements are the subsets of $V_T(a,b)$
which contain $J_T(a,b)$, and whose ordering is inclusion.  
There is a pairing of categories
\[ C_T(b,c)\times C_T(a,b) \ra C_T(a,c) \]
given by union of subsets.

Applying the nerve functor, we obtain a simplicial category $NC_T$
with object set $T_0$.  For $a,b\in T_0$, an $n$-simplex in
$NC_T(a,b)$ can be seen as a flag of sets $\vect{T}=T^0\subseteq
T^1\subseteq\cdots\subseteq T^n$, where $J_T\subseteq T^0$ and
$T^n\subseteq V_T$.


\begin{prop}\label{prop:mapping in necklaces}
Let $T$ be a necklace.  There is a natural isomorphism of simplicial
categories
between $\jC(T)$ and $NC_T$.
\end{prop}

\begin{proof}
Write $T=B_1\Wedge B_2\Wedge \cdots \Wedge B_k$, where the $B_i$'s are
the beads of $T$.
Then 
\begin{myequation}
\label{eq:CT}
 \jC(T)=\jC(B_1)\amalg_{\jC(*)} \jC(B_2)\amalg_{\jC(*)}\cdots
\amalg_{\jC(*)} \jC(B_k)
\end{myequation}
since $\jC$ preserves colimits.  Note that $\jC(*)=\jC(\Delta^0)=*$, 
the category with one object and a single morphism (the identity).  

Note that we have isomorphisms $\jC(B_i)\iso NC_{B_i}$ by Lemma
\ref{le:C(Delta^n)}.  We therefore get maps of categories
$\jC(B_i)\ra NC_{B_i} \ra NC_T$, and it is readily checked these
extend to a map $f\colon \jC(T)\ra NC_T$.  To see that this functor is
an isomorphism, it suffices to show that it is fully faithful (as it
is clearly a bijection on objects).

For any $a,b\in T_0$ we will construct an inverse to the map
$f\colon \jC(T)(a,b) \ra NC_T(a,b)$, when $b>a$ (the case $b\leq a$ being obvious).  Let $B_r$ and $B_s$ be the beads containing $a$ and $b$, respectively (if $a$ (resp. $b$) is a joint, let $B_r$ (resp. $B_s$) be the latter (resp. former) of the two beads which contain it).  Let
$j_r,j_{r+1}\ldots,j_{s+1}$ denote the ordered elements of $J_T(a,b)$, indexed so that
$j_i$ and $j_{i+1}$ lie in the bead $B_i$; note that $j_r=a$ and $j_{s+1}=b$.  

Any simplex  $x\in NC_T(a,b)_n$ can be uniquely
written as the composite of $n$-simplices $x_s\circ\cdots\circ
x_{r}$, where $x_i\in NC_T(j_{i},j_{i+1})_n$.  Now $j_{i}$ and
$j_{i+1}$ are vertices within the same bead $B_{i}$ of $T$, therefore
$x_i$ may be regarded as an $n$-simplex in
$\jC(B_i)(j_i,j_{i+1})$.
We then get associated $n$-simplices in $\jC(T)(j_i,j_{i+1})$, and taking
their composite gives an $n$-simplex $\tilde{x}\in\jC(T)(a,b)$.  We
define a map $g\colon NC_T(a,b)\ra \jC(T)(a,b)$ by sending $x$ to
$\tilde{x}$.
One readily checks that this is well-defined
and compatible with the simplicial operators, and it is also clear
that $f\circ g=\id$.  

To see that $f$ is an isomorphism it suffices to now show that $g$ is
surjective.  From the expression (\ref{eq:CT}) for $\jC(T)$ as
a colimit of the categories $\jC(B_i)$, it follows at once that every
map in $\jC(T)(a,b)$ can be written as a composite of maps
from the $\jC(B_i)$'s.  It is an immediate consequence that $g$ is surjective.
\end{proof}

\begin{cor}
\label{cor:mapping space is cube}
Let $T=B_0\Wedge B_1\Wedge \cdots \Wedge B_k$ be a necklace.  Let
$a,b\in T_0$ be such that $a<b$.  Let $j_r,j_{r+1},\ldots,j_{s+1}$ be
the elements of $J_T(a,b)$ (in order), and let $B_i$ denote the bead
containing $j_i$ and $j_{i+1}$, for $r\leq i\leq s$.  Then the map
\[ 
\jC(B_s)(j_s,j_{s+1})\times \cdots \times 
\jC(B_r)(j_r,j_{r+1}) 
\ra \jC(T)(a,b)
\]
is an isomorphism.  Therefore $\jC(T)(a,b)\iso (\Delta^1)^N$ where
$N=|V_T(a,b)-J_T(a,b)|$.  In particular, $\jC(T)(a,b)$ is contractible if $a\leq b$ and empty otherwise.
\end{cor}

\begin{proof}
Follows at once from Proposition \ref{prop:mapping in necklaces}.
\end{proof}

\begin{remark}\label{rem:heuristic}
Given a necklace $T$, there is a heuristic way to understand faces
(both codimension one and higher) in
the cubes $\jC(T)(a,b)$ in terms of ``paths" from $a$ to $b$ in $T$.
To choose a face in $\jC(T)(a,b)$, one chooses three subsets
$Y,N,M\subset V_T(a,b)$ which cover the set $V_T(a,b)$ and are mutually
disjoint.  The set $Y$ is the set of vertices which we require our
path to go through -- it must contain $J_T(a,b)$; the set $N$ is the
set of vertices which we require our path to not go through; and the
set $M$ is the set of vertices for which we leave the question open.
Such choices determine a unique face in $\jC(T)(a,b)$.  The dimension
of this face is precisely the number of vertices in $M$.
\end{remark}

%%%%%%%%%%%%%%%%%%%%%%%%%%%%%%%%%%%%%%%%%%%%%%%%%%%%%%%%%%%%%%%%%%%%%%

\section{The rigidification functor}\label{sec:jC}

Recall that we fully understand $\jC(\Delta^n)$ as a simplicial
category, and that $\jC\colon \sSet \ra s\Cat$ is defined for $S\in\sSet$ by the formula
\[ \jC(S)=\colim_{\Delta^n\ra S} \jC(\Delta^n).\]
The trouble with this formula is that given a diagram $X\colon I\to
s\Cat$ of simplicial categories, it is generally quite difficult to
understand the mapping spaces in the colimit.  For the above colimit, however,
something special happens because the simplicial categories
$\jC(\Delta^n)$ are ``directed" in a certain sense.  It turns out by
making use of necklaces one can write down a precise description of
the mapping spaces for $\jC(S)$; this is the goal of the present
section.

\medskip



Fix a simplicial set $S$ and elements $a,b\in S_0$.  
For any necklace $T$ and map $T\ra S_{a,b}$, there is an induced map
$\jC(T)(\alpha,\omega) \ra \jC(S)(a,b)$.  Let $(\Nec\ovcat S_{a,b})$
denote the category whose objects are pairs $[T,T\ra S_{a,b}]$ and
whose morphisms are maps of necklaces $T\ra T'$ giving commutative triangles over
$S$.  Then we obtain a  map
\begin{myequation}
\label{eq:colim-map}
 \colim_{T\ra S \in (\Nec\ovcat S_{a,b})} \Bigl [
\jC(T)(\alpha,\omega)\Bigr ] \lra \jC(S)(a,b).
\end{myequation}
Let us write $E_S(a,b)$ for the domain of this map.  Note that there
are composition maps
\begin{myequation}
\label{eq:composite}
 E_S(b,c) \times E_S(a,b) \lra E_S(a,c) 
\end{myequation}
induced in the following way.  Given $T\ra S_{a,b}$ and $U\ra S_{b,c}$
where $T$ and $U$ are necklaces, one obtains $T\Wedge U\ra S_{a,c}$ in
the evident manner.  The composite
\[\xymatrix{ 
\jC(U)(\alpha_U,\omega_U)\times \jC(T)(\alpha_T,\omega_T) \ar[r] &
\jC(T\Wedge U)(\omega_T,\omega_U)\times \jC(T\Wedge
U)(\alpha_T,\omega_T) \ar[d]^\mu \\
& \jC(T\Wedge U)(\alpha_T,\omega_U),
}
\]
induces the pairing of (\ref{eq:composite}), where the first map in
the composite can be understood using Proposition
\ref{prop:mapping in necklaces}  and we have used
$\omega_T=\alpha_U$.
One readily checks that $E_S$ is a simplicial category with object set
$S_0$, and (\ref{eq:colim-map}) yields a map of simplicial categories
$E_S \ra \jC(S)$.  Moreover, the construction $E_S$ is clearly functorial in $S$.

Here is our first result:

\begin{prop}
\label{pr:jC-main}
For every simplicial set $S$, the map $E_S\ra \jC(S)$ is an
isomorphism of simplicial categories.
\end{prop}

\begin{proof}
First note that if $S$ is itself a necklace then the identity map
$S\ra S$ is a terminal object in $(\Nec\ovcat S_{a,b})$.  It follows
at once that $E_S(a,b)\ra \jC(S)(a,b)$ is an isomorphism for all $a$
and $b$.

Now let $S$ be an arbitrary simplicial set, and 
choose vertices $a,b\in S_0$.  We will show
that $E_S(a,b)\to\jC(S)(a,b)$ is an isomorphism.
Consider the commutative diagram of simplicial sets
$$\xymatrix{\Bigl( \colim_{\Delta^k\to
S}E_{\Delta^k}\Bigr) (a,b)\ar[r]^-t\ar[d]_{\iso}&E_S(a,b)\ar[d]\\
\Bigl(\colim_{\Delta^k\to
S}\jC(\Delta^k)\Bigr)(a,b)\ar@{=}[r]&\jC(S)(a,b).}$$ The bottom equality is the definition of $\jC$.  The left-hand map is an isomorphism by our remarks in the first
paragraph.  It follows that the top map $t$ is injective.  To complete the proof it
therefore suffices to show that $t$ is surjective.

Choose an $n$-simplex $x\in E_S(a,b)_n$; it is represented by a
necklace $T$, a map $f\taking T\to S_{a,b}$, and an element
$\tilde{x}\in \jC(T)(\alpha,\omega)$.
We have a commutative diagram
\[\xymatrix{
\left(
\colim_{\Delta^k\to T}\jC(\Delta^k)\right)(\alpha,\omega)\ar[r] &
\jC(T)(\alpha,\omega) \\
\left(\colim_{\Delta^k\to T}E_{\Delta^k}\right)(\alpha,\omega)\ar[r]\ar[u]
\ar[d]_{f}&E_T(\alpha,\omega)\ar[u]\ar[d]^{E_f}\\
\left(\colim_{\Delta^k\to
S}E_{\Delta^k}\right)(a,b)\ar[r]^-t&E_S(a,b).} 
\]
The $n$-simplex in $E_T(\alpha,\omega)$ represented by
$[T,\id_T\taking T\to T;\tilde{x}]$ is sent to $x$ under $E_f$.  It
suffices to show that the middle horizontal map is surjective, for
then $x$ will be in the image of $t$.  But the top map is
an isomorphism, and the vertical arrows in the top row are
isomorphisms by the remarks from the first paragraph.  Thus, we are done.
\end{proof}

\begin{cor}
\label{co:description1}
For any simplicial set $S$ and elements $a,b\in S_0$, the simplicial
set $\jC(S)(a,b)$ admits the following description. 
An $n$-simplex in $\jC(S)(a,b)$
consists of an equivalence class of triples $[T,T\ra S,\vect{T}]$, where
\begin{itemize}
\item$T$ is a necklace; 
\item $T\ra S$ is a map of
simplicial sets which sends $\alpha_T$ to $a$ and $\omega_T$ to $b$;
and 
\item $\vect{T}$ is a flag of sets $T^0\subseteq
T^1\subseteq\cdots\subseteq T^n$ such that $T^0$ contains the joints
of $T$ and $T^n$ is contained in the set of vertices of $T$.
\end{itemize}
The equivalence
relation is generated by considering $[T,T\ra S; \vect{T}]$ and $[U,U\ra
S;\vect{U}]$ to be equivalent if there exists a map of necklaces $f\colon
T\ra U$ over $S$ with $\vect{U}=f_*(\vect{T})$.


The $i$th face (resp. degeneracy) map omits (resp. repeats) the set
$T^i$ in the flag.
That is, if
$x=[T,T\ra S;T^0\subseteq\cdots\subseteq T^n]$ represents an $n$-simplex of
$\jC(S)(a,b)$ and $0\leq i\leq n$, then
$$s_i(x)=[T,T\ra S; T^0\subseteq\cdots\subseteq
T^i\subseteq T^i\subseteq\cdots\subseteq T^n]$$
and
\[
d_i(x)=[T\ra S; T^0\subseteq\cdots\subseteq
T^{i-1}\subseteq T^{i+1}\subseteq\cdots\subseteq T^n].
\]
\end{cor}

\begin{proof}
This is a straightforward interpretation of the colimit  appearing in
the definition of $E_S$ from (\ref{eq:colim-map}).  Recall that  every
colimit can be written as a coequalizer
\[ 
 \colim_{T\ra S \in (\Nec\ovcat S_{a,b})} \Bigl [
\jC(T)(\alpha,\omega)\Bigr ] \iso \coeq \Bigl[
\coprod_{T_1\ra T_2\ra S} \jC(T_1)(\alpha,\omega)
\dbra \coprod_{T\ra S} \jC(T)(\alpha,\omega)\Bigr],
\]
and that simplices of $\jC(T)(\alpha,\omega)$ are identified with flags of subsets of
$V_T$, containing $J_T$, by Proposition~\ref{prop:mapping in necklaces}. The
simplices of $\coprod_{T\ra S} \jC(T)(\alpha,\omega)$ therefore
correspond to triples
$[T,T\ra S,\vect{T}]$, and the simplices of the coequalizer correspond
to equivalence classes of such triples.  The relation given in the
statement of the corollary is precisely the one coming from the above coequalizer.
\end{proof}

Our next goal is to simplify the equivalence relation appearing in
Corollary~\ref{co:description1} somewhat.  This analysis is somewhat
cumbersome, but culminates in the important Proposition~\ref{pr:Fn-hodiscrete}.

Let us begin by introducing some terminology.  A \dfn{flagged
necklace} is a pair $[T,\vect{T}]$ where $T$ is a necklace and
$\vect{T}$ is a flag of subsets of $V_T$ which all contain $J_T$.  
The \dfn{length of the flag} is the number of subset symbols, or one less
than the number of subsets.  
A morphism of flagged necklaces $[T,\vect{T}]\ra [U,\vect{U}]$ exists
only if the flags have the same length, in which case it is a map of necklaces
$f\colon T\ra U$   such that $f(T^i)=U^i$ for all $i$.
Finally, a flag $\vect{T}=(T^0\subseteq \cdots \subseteq T^n)$ is
called \dfn{flanked} if $T^0=J_T$ and $T^n=V_T$.  Note that if
$[T,\vect{T}]$ and $[U,\vect{U}]$ are both flanked, then every
morphism $[T,\vect{T}]\ra [U,\vect{U}]$ is surjective (because its
image will be a subnecklace of $U$ having the same joints and vertices
as $U$, hence it must be all of $U$).

\begin{lemma}
\label{le:flankify}
Under the equivalence relation of Corollary~\ref{co:description1},
each of the triples $[T,T\ra S,\vect{T}]$ is equivalent to one in which the
flag is flanked.  Moreover, two flanked triples are equivalent (in the
sense of Corollary~\ref{co:description1}) if and only if they can be
connected by a zig-zag of morphisms of flagged necklaces in which
every triple of the zig-zag is flanked.  
\end{lemma}

\begin{proof}
Suppose given a flagged necklace $[T,T^0\subseteq \cdots \subseteq
T^n]$.  There is a unique subnecklace $T'\inc T$ whose set of joints
is $T^0$ and whose vertex set is $T^n$.  Then the pair
$(T',T^0\subseteq \cdots \subseteq T^n)$ is flanked.  This assignment,
which we call {\bf flankification}, is actually functorial: a morphism
of flagged necklaces $f\colon[T,\vect{T}]\ra [U,\vect{U}]$ must map
$T'$ into $U'$ and therefore gives a morphism $[T',\vect{T}]\ra
[U',\vect{U}]$.

Using the equivalence relation of Corollary~\ref{co:description1},
each triple $[T,T\ra S,\vect{T}]$ will be equivalent to the flanked
triple $[T',T'\ra T\ra S,\vect{T}]$ via the map $T'\ra T$.  If the
flanked triple $[U,U\ra S,\vect{U}]$ is equivalent to the flanked
triple $[V,V\ra S,\vect{V}]$ then there is a zig-zag of maps between
triples which starts at the first and ends at the second, by Corollary \ref{co:description1}.  Applying
the flankification functor gives a
corresponding zig-zag in which every object is flanked.
\end{proof}

\begin{remark}\label{rem:flanked and outers}
By the previous lemma, we can alter our model for $\jC(S)(a,b)$ so
that the $n$-simplices are equivalence classes of triples $[T,T\ra
S,\vect{T}]$ in which the flag is flanked, and the equivalence
relation is given by maps (which are necessarily surjections) of
flanked triples.  Under this model the degeneracies and inner faces
are given by the same description as before: repeating or omitting one
of the subsets in the flag.  The outer faces $d_0$ and $d_n$ are now
more complicated, however, because omitting the first or last subset
in the flag may produce one which is no longer flanked; one must first
remove the subset and then apply the flankification functor from
Lemma~\ref{le:flankify}.   This model for $\jC(S)(a,b)$ was originally
shown to us by Jacob Lurie; it will play only a very minor role in 
what follows.
\end{remark}

Our next task will be to analyze surjections of flagged triples.  Let
$T$ be a necklace and $S$ a simplicial set.  Say that a map $T\ra S$
is \dfn{totally nondegenerate} if the image of each bead of $T$ is a
nondegenerate simplex of $S$.  Note a totally nondegenerate map need
not be an injection: for example, let $S=\Delta^1/\bd{1}$ and consider
the nondegenerate $1$-simplex $\Delta^1 \ra S$.

As a prelude to what we are about to do, recall that the $n$-simplices
of a simplicial set $S$ correspond to maps $\Delta^n\ra S$ in $\sSet$.
Every map $\sigma\colon\Delta^k\ra \Delta^n$ represents a simplicial
operator, in the sense that if $s\colon \Delta^n\ra S$ is an
$n$-simplex of $S$ then $\sigma\circ s$ gives a $k$-simplex of $S$.  Said
differently, there is an evident map of categories $\Delta \ra \sSet$ (the
Yoneda embedding) and the image is the full subcategory of $\sSet$
whose objects are the $\Delta^n$'s: so every map $\Delta^k\ra
\Delta^n$ corresponds to a map in $\Delta$, i.e., a simplicial
operator.  Under this correspondence, surjections $\Delta^k\ra \Delta^n$
correspond to degeneracy operators and injections correspond to face
operators.  



In a simplicial set $S$, if $z\in S$ is a
degenerate simplex then there is a unique nondegenerate simplex $z'$
and a unique degeneracy operator $s_\sigma=s_{i_1}s_{i_2}\cdots
s_{i_k}$ such that $z=s_\sigma(z')$; see \cite[Lemma 15.8.4]{H}.
In other words, if $\Delta^n\ra S$ is degenerate then there is a
nondegenerate simplex $\Delta^k\ra S$ and a unique surjection
$\Delta^n\ra\Delta^k$ making the evident triangle commute.  Applying
this one bead at a time,
one 
finds that for any map $T\ra S$ there is
a necklace $\overline{T}$, a map $\overline{T}\ra S$ which is totally
nondegenerate, and a surjection of necklaces $T\ra
\overline{T}$ making the evident triangle commute; moreover, these
three things are unique up to isomorphism.

\begin{prop}
\label{pr:lift}
Let $S$ be a simplicial set and let $a,b\in S_0$.  
\begin{enumerate}[(a)]
\item Suppose that $T$ and $U$ are necklaces, $U\To{u} S$ and $T\To{t}
S$ are two maps, and that $t$ is totally nondegenerate.  Then
there is at most one surjection $f\colon U\fib T$
such that $u=t\circ f$.
\item
Suppose that one
has a diagram
\[ \xymatrix{ U \ar@{->>}[d]_g\ar@{->>}[r]^f & T \ar[d] \\
V\ar[r] & S
}
\]
where $T$, $U$, and $V$ are flagged necklaces, $T\ra S$ is totally
nondegenerate, and $f$ and $g$ are surjections.  Then there
exists a unique map of flagged necklaces $V\ra T$ making the diagram
commute.
\end{enumerate}
\end{prop}

\begin{proof}
It is easy to see that if $A\ra B$ is a map of
necklaces then every joint of $B$ is the image of some joint of
$A$.  Moreover, if $A\ra B$ is surjective then one checks using the ordering of $A$ and $B$ that every joint of
$A$ must map to a joint of $B$.  It follows readily that if
$A\ra B$ is a surjection of
necklaces and $B\neq *$ then every bead of $B$ is surjected on by a unique bead of
$A$.  Also, each bead of $A$ is either collapsed onto a joint of $B$
or else mapped surjectively onto a bead of $B$.

For (a), note that we may assume $T\neq *$ (or else the claim is
trivial).  Assume there are two distinct surjections $f,f'\colon U\ra T$ such
that $tf=tf'=u$.  Let $B$ be the first bead of $U$ on which $f$ and
$f'$ disagree.  Let $j$ denote the
initial vertex of $B$, and let $C$ be the bead of $T$ whose initial
vertex is $f(j)=f'(j)$.

Suppose that $f$ collapses $B$ to a point, in which case $u$ must also map $B$
to a point.  Then $f'$ cannot surject $B$ onto $C$, 
for otherwise $C\ra S$ would factor
through the point $u(B)$ and this would contradict $T\ra S$ being totally
nondegenerate.  So $f'$ also collapses $B$ to point, contradicting the
assumption that $f$ and $f'$ disagree on $B$.
Therefore $f$ (and by symmetry $f'$) cannot collapse $B$ to a point,
and hence must surject $B$ onto $C$.  
This identifies the simplex $B\ra U\ra S$ with a degeneracy of
the nondegenerate simplex $C\ra S$.  Then by uniqueness of
degeneracies we have that $f$ and $f'$ must coincide on $B$, which is
a contradiction.


Next we turn to part (b).  Note that the map $V\ra T$ will necessarily
be surjective, so the uniqueness part is guaranteed by (a); we need
only show existence.  

Observe that if $B$ is a bead in $U$ which maps to a point in $V$
then it maps to a point in $T$, by the reasoning above.  It now follows that
there exists a necklace $U'$, obtained by collapsing every bead of $U$
that maps to a point in $V$, and a commutative diagram
$$\xymatrix@=14pt{U\ar@{->>}[rr]^f\ar@{->>}[rd]\ar@{->>}[dd]_g&&T\ar[dd]\\
&U'\ar@{->>}[ru]^(.4){f'}\ar@{->>}[dl]_(.4){g'}\\V\ar[rr]&&S}$$
Replacing $U, f,$ and $g$ by $U',f',$ and $g'$, and dropping the
primes, we can now assume that $g$ induces a one-to-one correspondence
between beads of $U$ and beads of $V$.  Let $B_1,\ldots,B_m$ denote
the beads of $U$, and let $C_1,\ldots,C_m$ denote the beads of $V$.

Assume that we have constructed the lift $l\colon V\ra T$ on the beads
$C_1,\ldots,C_{i-1}$.  If the bead $B_i$ is mapped by $f$ to a point,
then evidently we can define $l$ to map $C_i$ to this same point and
the diagram will commute.  Otherwise $f$ maps $B_i$ surjectively onto
a certain bead $D$ inside of $T$.  We have the diagram
\[\xymatrix{B_i \ar@{->>}[r]^f \ar@{->>}[d]_{g} 
& D\ar[d]^{t}\\
C_i \ar[r]_{v} & S
}
\] 
where here $f$ and $g$ are surjections between simplices and therefore
represent degeneracy operators $s_f$ and $s_g$.  We have that
$s_f(t)=s_g(v)$.  But the simplex $t$ of $S$ is nondegenerate by
assumption, therefore by \cite[Lemma 15.8.4]{H} we must have
$v=s_h(t)$ for some degeneracy operator $s_h$ such that $s_f=s_gs_h$.
The operator $s_h$ corresponds (as explained prior to
Proposition~\ref{pr:lift}) to a surjection of simplices $C_i \ra D$
making the above square commute, and we define $l$ on $C_i$ to
coincide with this map.  Continuing by induction, this produces the
desired lift $l$.  It is easy to see that $l$ is a map of flagged
necklaces, as $l(V^i)=l(g(U^i))=f(U^i)=T^i$.
\end{proof}


\begin{cor}
\label{co:unique-triple}
Let $S$ be a simplicial set and $a,b\in S_0$.  
Under the equivalence relation from Corollary~\ref{co:description1},
every triple $[T,T\ra S_{a,b},\vect{T}]$ is equivalent to a unique triple
$[U,U\ra S_{a,b},\vect{U}]$ which is both flanked and totally nondegenerate.
\end{cor}

\begin{proof}
Let $t=[T,T\ra S_{a,b},\vect{T}]$.  Then $t$ is clearly equivalent to at
least one flanked, totally nondegenerate triple because we can replace
$t$ with $[T',T'\ra S_{a,b},\vect{T}]$ (flankification) and then with
$[\overline{T'},\overline{T'}\ra S_{a,b},\vect{T}']$ (defined above Proposition \ref{pr:lift}).

Now suppose that $[U,U\ra S_{a,b},\vect{U}]$ and $[V,V\ra S_{a,b},\vect{V}]$ are
both flanked, totally nondegenerate, and equivalent in
$\jC(S)(a,b)_n$.  Then by Lemma~\ref{le:flankify} there is a zig-zag
of maps between flanked necklaces (over $S$) connecting $U$ to $V$:
\[ \xymatrixcolsep{1.3pc}
\xymatrix{ &W_1\ar@{->>}[dl]\ar@{->>}[dr] && W_2\ar@{->>}[dl]\ar@{->>}[dr] && \cdots\ar@{->>}[dl]\ar@{->>}[dr] &&
W_k\ar@{->>}[dl]\ar@{->>}[dr] \\
U=U_1 && U_2 && U_3 & \cdots & U_{k}& &\,U_{k+1}=V
}
\]
Using Proposition~\ref{pr:lift}, we inductively construct surjections of
flanked necklaces $U_i\ra U$ over $S$.  This produces a surjection $V\ra U$
over $S$.  Similarly, we obtain a surjection $U\to V$ over $S$.  By
Proposition~\ref{pr:lift}(a) these maps must be inverses of each other; that
is, they are isomorphisms.
\end{proof}

\begin{remark}
Again, as in Remark~\ref{rem:flanked and outers} the above corollary
shows that we can describe $\jC(S)(a,b)$ as the simplicial set whose
$n$-simplices are triples $[T,T\ra S,\vect{T}]$ which are both flanked
and totally nondegenerate.  The degeneracies and inner faces are again
easy to describe---they are repetition or omission of a set in the
flag---but for the outer faces one must first omit a set and then
modify the triple appropriately.  The usefulness of this description
is 
limited because of these complications with the outer faces, but it does
make a brief appearance in
Corollary~\ref{cor:maps in ordered} below.
\end{remark}

The following result is the culmination of our work in this section,
and will turn out to be a key step in the proof of our main theorems.
Fix a simplicial set $S$ and vertices $a,b\in S_0$, and let $F_n$
denote the category of flagged triples over $S_{a,b}$ that have length
$n$.  That is, the objects of $F_n$ are triples $[T,T\ra
S_{a,b},T^0\subseteq \cdots
\subseteq T^n]$ and morphisms are maps of necklaces $f\colon T\ra T'$
over $S$ such that $f(T^i)=(T')^i$ for all $i$.

\begin{prop} 
\label{pr:Fn-hodiscrete}
For each $n\geq 0$, the nerve of $F_n$ is homotopy discrete
in $\sSet_K$.  
\end{prop}

\begin{proof}

Recall from Lemma~\ref{le:flankify} that there is a functor $\phi\colon F_n
\ra F_n$ which sends any triple to its `flankification'.  The
flankification is a subnecklace of the original necklace, and
therefore the inclusion gives a
natural transformation from $\phi$ to the identity. If $F_n'$ denotes
the full subcategory of $F_n$ consisting of flanked triples, we thus find
that 
$F_n'\inc F_n$ induces a Kan equivalence upon
taking nerves.  To prove the proposition it
will therefore suffice to prove that the nerve of $F_n'$ is homotopy discrete.  

Recall from Corollary~\ref{co:unique-triple} that every component of
$F_n'$ contains a unique triple $t$ which is both flanked and totally
nondegenerate.  Moreover, following the proof of that corollary
one sees that every triple in the same component as $t$
admits a unique map to $t$---uniqueness follows from
Proposition~\ref{pr:lift}(a), using that a map of flanked triples is
necessarily surjective.  We have therefore shown that
$t$ is a final object
for its component, hence its component is contractible.  This
completes the proof.

\end{proof}


\subsection{The functor $\jC$ applied to ordered simplicial sets}~

Note that even if a simplicial set $S$ is small---say, in the sense
that it has finitely many nondegenerate simplices---the space
$\jC(S)(a,b)$ may be quite large.  This is due to the fact that there
are infinitely many necklaces mapping to $S$ (if $S$ is nonempty).  For certain simplicial sets $S$,
however, it is possible to restrict to necklaces which map {\it injectively\/}
into $S$; this cuts down the possibilities.  The following results
and subsequent
example demonstrate this.  Recall the definition 
of ordered simplicial sets from Definition \ref{de:ordered}.


\begin{lemma}
\label{le:jCL-necklace}
Let $D$ be an ordered simplicial set and let $a,b\in D_0$.  Then every
$n$-simplex in $\jC(D)(a,b)$ is represented by a unique triple
$[T,T\ra D,\vect{T}]$ in which $T$ is a necklace, $\vect{T}$ is a flanked flag of length $n$, and the map $T\ra D$
is injective.
\end{lemma}

\begin{proof}
By Corollary \ref{co:unique-triple}, every $n$-simplex in
$\jC(D)(a,b)$ is represented by a unique triple $[T,T\to D,\vect{T}]$
which is both flanked and totally non-degenerate.  It suffices to show
that if $D$ is ordered, then any totally non-degenerate map $T\to D$
is injective.  This follows from Lemma \ref{lemma:facts on ordered}(6).
\end{proof}

\begin{cor}\label{cor:maps in ordered}

Let $D$ be an ordered simplicial set, and $a,b\in D_0$.  Let
$M_D(a,b)$ denote the simplicial set for which $M_D(a,b)_n$ is the set
of triples $[T,T\To{f} D_{a,b},\vect{T}]$, where $f$ is injective and
$\vect{T}$ is a flanked flag of length $n$; face and boundary maps are
as in Remark \ref{rem:flanked and outers}.  Then there is a natural
isomorphism \[\jC(D)(a,b)\To{\iso} M_D(a,b).\] 
\end{cor}

\begin{proof}

This follows immediately from Lemma \ref{le:jCL-necklace}.

\end{proof}


\begin{ex}
\label{ex:computation}
Consider the simplicial set $S=\Delta^2\amalg_{\Delta^1}\Delta^2$
depicted as follows:

\begin{center}\begin{picture}(40,40)\put(-4,-4){1}\put(31,-3){3}\put(-4,30){0}\put(30,30){2}\put(0,0){$\bullet$}\put(26,0){$\bullet$}\put(0,26){$\bullet$}\put(26,26){$\bullet$}\put(3,2){\vector(1,0){25}}\put(2,28){\vector(0,-1){24}}\put(28,28){\vector(0,-1){24}}\put(2,28){\vector(1,0){25}}\put(3,2){\vector(1,1){25}}\end{picture}\end{center}

We will describe the mapping space $X=\jC(S)(0,3)$ by giving its
non-degenerate simplices and face maps.  Note that $S$ is ordered.

By Lemma \ref{le:jCL-necklace}, it suffices to consider flanked
necklaces that inject into $S$.  There are only five such injections
that have endpoints $0$ and $3$.  These are
$T=\Delta^1\vee\Delta^1$, which maps to $S$ in two different ways
$f,g$; and $U=\Delta^1\vee\Delta^1\vee\Delta^1$,
$V=\Delta^1\vee\Delta^2$, and $W=\Delta^2\vee\Delta^1$, each of which
maps uniquely into $S_{0,3}$.  The image of $T_0$ under $f$ is
$\{0,1,3\}$ and under $g$ is $\{0,2,3\}$.  The images of $U_0, V_0,$
and $W_0$ are all $\{0,1,2,3\}$.

We find that $X_0$ consists of three elements $[T;\{0,1,3\}]$,
$[T;\{0,2,3\}]$ and $[U;\{0,1,2,3\}]$.  There are two nondegenerate
$1$-simplices, $[V;\{0,1,3\}\subset\{0,1,2,3\}]$ and
$[W;\{0,2,3\}\subset\{0,1,2,3\}]$.  These connect the three
$0$-simplices in the obvious way, resulting in two $1$-simplices with
a common final vertex.  There are no higher non-degenerate simplices.
Thus $\jC(S)(0,3)$ looks like
$$\xymatrix@1{\bullet\ar[r]&\bullet&\bullet\ar[l]}.$$
\end{ex}

%%%%%%%%%%%%%%%%%%%%%%%%%%%%%%%%%%%%%%%%%%%%%%%%%%%%%%%%%%%%%%%%%%%%%%

\section{Homotopical models for rigidification}

In the last section we gave a very explicit description of the mapping
spaces $\jC(S)(a,b)$, for arbitrary simplicial sets $S$ and $a,b\in
S_0$.  While this description was explicit, in some ways it is not
very useful from a homotopical standpoint---in practice it is hard to use this
description to identify the homotopy type of $\jC(S)(a,b)$. 

In this section we will discuss a functor $\jCn\colon\sSet\ra s\Cat$
that has a simpler description than $\jC$ and which is more
homotopical.  We prove that for any simplicial set $S$ there is a
natural zigzag of weak equivalences between $\jC(S)$ and $\jCn(S)$.
Variants of this construction are also introduced, leading to a
collection of functors $\sSet\to\sCat$ all of which are weakly equivalent to $\jC$.
  
\medskip

Let $S\in \sSet$.  A choice of $a,b\in S_0$ will be
regarded as a map $\bd{1}\ra S$.  Let $(\Nec\ovcat S_{a,b})$ be the
overcategory for the inclusion functor  $\Nec\inc (\bd{1}\ovcat S)$.  
Finally, define
\[ \jCn(S)(a,b)=N(\Nec\ovcat S_{a,b}).
\]
Then $\jC(S)$ is a simplicial category in an evident way; see (\ref{dia:composition}).

\begin{remark}
Both the functor $\jC$ and the functor $\jCn$ have distinct advantages
and disadvantages.  The main advantage to $\jC$ is that it is left
adjoint to the coherent nerve functor $N$ (in fact it is a left
Quillen functor $\sSet_J\ra s\Cat$); as such, it preserves
colimits.  However, as mentioned above, the functor $\jC$ can be
difficult to use in practice because the mapping spaces have an
awkward description.

It is at this point that our functor $\jCn$ becomes useful, because
the mapping spaces are given as 
nerves of 1-categories.  Many tools are available for determining
when a morphism between nerves is a Kan equivalence. 
This will be an important point in
\cite{DS}, where we show the $\jC$ functor gives a Quillen equivalence
between $\sSet_J$ and $\sCat$.  See also Section~\ref{se:properties} below.
\end{remark}



Our main theorem is that there is a simple zigzag of weak equivalences
between $\jC(S)$ and $\jCn(S)$; that is, there is a functor
$\jCh\taking\sSet\to\sCat$ and natural weak equivalences $\jC\from
\jCh\to\jCn$.  We begin by describing the functor $\jCh$. 

Fix a simplicial set $S$.   Define $\jCh(S)$ to
have object set $S_0$, and for every $a,b\in S_0$
\[ \jCh(S)(a,b)=\hocolim_{T\in (\Nec\ovcat S_{a,b})}
\jC(T)(\alpha,\omega).\]
Note the similarities to Proposition \ref{pr:jC-main}, where it was shown
that $\jC(S)(a,b)$ has a similar description in which the hocolim is
replaced by the colim.  In our definition of $\jCh(S)(a,b)$ we mean to
use a particular model for the homotopy colimit: namely, recall that
if $X\colon I\ra \sSet$ then the Bousfield-Kan model
\cite[XII.5.2]{BK} 
for $\hocolim_I X$ is the
diagonal of the bisimplicial set (the simplicial replacement of $X$) which in level $k$ is
\[\coprod_{i_0\ra \cdots \ra i_k} X_{i_0}.
\]
Applying this in our case, $\jCh(S)(a,b)$ is 
the diagonal
of the bisimplicial set whose $(k,l)$-simplices are pairs 
\begin{myequation}
\label{eq:bisimp}
 (F\taking
[k]\to(\Nec\ovcat S_{a,b})\ ,\  x\in \jC(F(0))(\alpha,\omega)_l),
\end{myequation}
where $F(0)$ denotes the necklace obtained by 
applying $F$ to $0\in [k]$ and then applying the forgetful functor
$(\Nec\downarrow S_{a,b})\to\Nec$.  The composition law for $\jCh$ is
defined just as for the $E_S$ construction from Section~\ref{sec:jC}.

If one pictures the bisimplicial set of (\ref{eq:bisimp}) with the
index $k$ varying horizontally and $l$ varying vertically, note that
the $l$th horizontal row is the nerve $NF_l$ of the category of
flagged necklaces mapping to $S$ (where the flags have length $l$).


We proceed to establish
natural transformations $\jCh\to\jCn$ and $\jCh\to\jC$.
Note that the simplicial set
$\jCn(S)(a,b)$ is the homotopy colimit of the constant
functor $\{*\}\taking(\Nec\ovcat S_{a,b})\to\sSet$ which sends
everything to a point (using \cite[Proposition 18.1.6]{H}, for example).  
The map $\jCh(S)(a,b) \ra
\jCn(S)(a,b)$ is the map of homotopy colimits induced by the evident
map of diagrams.
 Since the spaces $\jC(T)(\alpha,\omega)$ are all
contractible simplicial sets (see Corollary~\ref{cor:mapping space is
cube}), the induced map $\jCh(S)(a,b)\ra \jCn(S)(a,b)$ is a Kan
equivalence.  We thus obtain a natural weak equivalence of simplicial
categories $\jCh(S)\to\jCn(S)$.

For any diagram in a model category there is a canonical natural
transformation from the homotopy colimit to the colimit of that
diagram.  Hence there is a morphism
\[ \jCh(S)(a,b)\to\colim_{T\in(\Nec\downarrow
S_{a,b})}\jC(T)(\alpha,\omega) \iso \jC(S)(a,b).
\] 
(For the isomorphism we are using Proposition~\ref{pr:jC-main}.)
As this is natural in $a,b\in S_0$ and
natural in $S$, we have a natural transformation $\jCh\to\jC$.

\begin{thm}\label{th:main}
For every simplicial set $S$, the maps 
$\jC(S)\from \jCh(S)\to\jCn(S)$ are weak equivalences of simplicial
categories.  Equivalently, the maps induce Kan equivalences on all
mapping spaces.
\end{thm}

\begin{proof}
We have already established the result for $\jCh\to\jCn$,
so it suffices to show that for each
simplicial set $S$ and objects $a,b\in S_0$ the natural map
$\jCh(S)(a,b)\to\jC(S)(a,b)$ is a Kan equivalence. 

Recall 
that $\jCh(S)(a,b)$ is the diagonal of the bisimplicial set
(\ref{eq:bisimp}), and that the
$l$th `horizontal' row is the nerve $NF_l$ of the category of flagged
necklaces mapping to $S$, where the flags have length $l$.  
Also recall from Corollary~\ref{co:description1} that $\jC(S)(a,b)$ is the simplicial set which
in level $l$ is $\pi_0(NF_l)$.  But
Proposition~\ref{pr:Fn-hodiscrete} says that $NF_l \ra \pi_0(NF_l)$ is
a Kan equivalence, for every $l$.  It follows that $\jCh(S)(a,b)\ra
\jC(S)(a,b)$ is also a Kan equivalence.
\end{proof}

\subsection{Other models for rigidification}
\label{se:gadgets}

One can imagine variations of our basic construction in which one
replaces necklaces with other convenient simplicial sets---which we
might term ``gadgets,'' for lack of a better word.  We will see in
Section~\ref{sec:properties of rigidification}, for instance, that
using {\it products\/} of necklaces leads to a nice theorem about the
rigidification of a product.  In \cite{DS}
several key arguments will hinge on a clever choice of what gadgets to
use.  In the material below we give some basic requirements of the
``gadgets'' which will ensure they give a model equivalent to that of
necklaces.


Suppose $\cP$ is a subcategory of $\bpSet=(\bd{1}\ovcat\sSet)$ containing
the terminal object.  For any simplicial set $S$ and vertices $a,b\in
S_0$, let $(\cP\ovcat S_{a,b})$ denote the overcategory whose objects
are pairs $[P,P\to S]$, where $P\in\cP$ and the map $P\to S$ sends
$\alpha_P\mapsto a$ and $\omega_P\mapsto b$.  Define
\[\jC^\cP(S)(a,b)=N(\cP\ovcat S_{a,b}).\] The object $\jC^\cP$ is
simply an assignment which takes a simplicial set $S$ with two
distinguished vertices and produces a ``$\cP$-mapping space."  However, if
$\cP$ is closed under the wedge operation (i.e. for any
$P_1,P_2\in\cP$ one has $P_1\vee P_2\in\cP$), then $\jC^\cP$ may be
given the structure of a functor $\sSet\to\sCat$ in the evident way.

\begin{defn}\label{de:gadgets}

We call a subcategory $\kG\subseteq\bpSet$ a \dfn{category of gadgets} if it satisfies the following properties:
\begin{enumerate}[(1)]
\item $\kG$ contains the category $\Nec$, 
\item For every object $X\in \kG$ and every necklace $T$, all maps
$T\ra X$ are contained in $\kG$, and 
\item For any $X\in \kG$, the simplicial set $\jC(X)(\alpha,\omega)$
is contractible.
\end{enumerate}

The category $\kG$
is said to be \dfn{closed under wedges} if it is also true that
\begin{enumerate}[(4)]
\item For any $X,Y\in \kG$, the wedge $X\Wedge Y$ also belongs to $\kG$.
\end{enumerate}

\end{defn}

\begin{prop}
\label{pr:gadget}
Let $\kG$ be a category of gadgets.  Then for any simplicial set $S$
and any $a,b\in S_0$,
the natural map
\[ \jCn(S)(a,b) \lra \jC^{\kG}(S)(a,b) \]
(induced by the inclusion $\Nec\inc \kG$)
is a Kan equivalence.  If $\kG$ is closed under wedges then the map of
simplicial categories $\jCn(S) \ra \jC^{\kG}(S)$ is a weak equivalence.
\end{prop}

\begin{proof}
Let $j\colon (\Nec\ovcat S_{a,b}) \ra (\kG \ovcat S_{a,b})$ be the
functor induced by the inclusion map $\Nec \inc \kG$.  The map in the
statement of the proposition is just the nerve of $j$.  To verify that
it is a Kan equivalence, it is enough by Quillen's Theorem A \cite{Q}
to verify that all the overcategories of $j$ are contractible.  So fix
an object $[X,X\ra S]$ in $(\kG\ovcat S_{a,b})$.  The overcategory
$(j\ovcat [X,X\ra S])$ is precisely the category $(\Nec\ovcat
X_{\alpha,\omega})$, the nerve of which is $\jCn(X)(\alpha,\omega)$.
By Theorem~\ref{th:main} this nerve is weakly equivalent to
$\jC(X)(\alpha,\omega)$, which is contractible by our assumptions
about $\kG$.

The second statement of the result is a direct consequence of the first.
\end{proof}



%%%%%%%%%%%%%%%%%%%%%%%%%%

\section{Properties of rigidification}\label{sec:properties of
rigidification}
\label{se:properties}

In this section we establish two main properties of the
rigidification functor $\jC$.  First, we prove that there is a
natural weak equivalence $\jC(X\times Y) \he \jC(X)\times \jC(Y)$.
Second, we prove that whenever $S\ra S'$ is a Joyal equivalence
it follows that $\jC(S)\ra \jC(S')$ is a weak equivalence in
$\sCat$.  These properties are also proven in \cite{L}, but the proofs
we give here are of a different nature and make central use of the
$\jCn$ functor.

\medskip


If $T_1,\ldots,T_n$ are necklaces then they are, in particular,
ordered simplicial sets in the sense of Definition~\ref{de:ordered}.
So $T_1\times \cdots \times T_n$ is also ordered, by
Lemma~\ref{lemma:facts on ordered}.  Let $\kG$ be the full subcategory
of $\bpSet=(\bd{1}\ovcat \sSet)$ whose objects are products of necklaces
with a map $f\taking\bd{1}\ra T_1\times\cdots \times T_n$ that has
$f(0)\poleq f(1)$.  

\begin{prop}
\label{pr:products=gadgets}
The category $\kG$ is a category of gadgets in the sense of
Definition~\ref{de:gadgets}.
\end{prop}

For the proof of this one needs to verify that $\jC(T_1\times\cdots
\times T_n)(\alpha,\omega)\he *$.  This is not difficult, but is a bit
of a distraction; we prove it later as Proposition \ref{prop:product of necklaces}.  

\begin{prop}
\label{pr:product}
For any simplicial sets $X$ and $Y$, both $\jC(X\times Y)$ and
$\jC(X)\times \jC(Y)$ are simplicial categories with object set
$X_0\times Y_0$.  For any $a_0,b_0\in X$ and $a_1,b_1\in Y$, the
natural map
\[ \jC(X\times Y)(a_0a_1,b_0b_1) \ra \jC(X)(a_0,b_0)\times
\jC(Y)(a_1,b_1)
\]
induced by $\jC(X\times Y)\ra \jC(X)$ and $\jC(X\times Y)\ra \jC(Y)$
is a Kan equivalence.
Consequently, the map of simplicial categories
\[ \jC(X\times Y) \ra \jC(X)\times \jC(Y)\]
is a weak equivalence in $\sCat$.
\end{prop}

\begin{proof}

Let $\kG$ denote the above category of gadgets, in which the objects
are products of necklaces.  By Theorem~\ref{th:main} and
Proposition~\ref{pr:gadget} 
it suffices to prove the result for $\jC^\kG$ in place of $\jC$.

Consider the functors 
\[\Adjoint{\phi}{(\kG\ovcat (X\cross
Y)_{a_0a_1,b_0b_1})}{(\kG\ovcat X_{a_0,b_0})\cross(\kG\ovcat
Y_{a_1,b_1})}{\theta}
\]
given by
\[ \phi\colon [G,G\ra X\times
Y] \mapsto \bigl ([G,G\ra X\times Y \ra X],[G,G\ra X\times Y \ra
Y] \bigr )
\]
\and
\[ \theta\colon \bigl ([G,G\ra X],[H,H\ra Y] \bigr ) \mapsto [G\times H,G\times
H\ra X\times Y].
\]
Note that we are using that the subcategory $\kG$ is closed under
finite products.

It is very easy to see that there is a natural transformation $\id \ra
\theta\phi$, obtained by using diagonal maps, and a natural transformation $\phi\theta\ra \id$, obtained by using projections.  As a consequence, the maps $\theta$
and $\phi$ induce inverse homotopy equivalences on the nerves.  This
completes the proof.

\end{proof}

Recall from Section~\ref{se:background} 
that if $S$ is a set then $\cE(S)$ is the groupoid with object
set $S$ having a unique map between any two objects.  Let $E\colon
\Set\ra \sSet$ be given by $E(S)=N(\cE(S))$.  The $k$-simplices of
$E(S)$ may be identified with $(k+1)$-tuples of elements of $S$, and
the face and degeneracy operators are given by deletion or repetition
of entries.  One may check that
for any
simplicial set $X$  we have $\sSet(X,E(S))=\Set(X_0,S)$.  In
particular, taking $X=\Delta^1$ note that any two $0$-simplices in
$E(S)$ have a unique $1$-simplex from the first to the second.

The functor $E$ is commonly called the $0$-coskeleton functor (see
\cite{AM}).
For any $n\in\N$ we denote $E^n=E(\{0,1,\ldots,n\})$.

Recall also from Section~\ref{se:background} the notion of weak
equivalence for simplicial categories.  In particular, note that if
$\cC$ is simplicial category then the map $\cC\ra *$ is a weak
equivalence if and only if for all $a,b\in \ob \cC$ the mapping space
$\cC(a,b)$ is contractible.  


\begin{lemma}
\label{le:C(En)}
For any $n\geq 0$, the simplicial category $\jC(E^n)$ is contractible
in $\sCat$.
\end{lemma}

\begin{proof}
By Theorem~\ref{th:main} it is sufficient to prove that the mapping
space $\jCn(E^n)(i,j)$ is contractible, for every $i,j\in
\{0,1,\ldots,n\}$.  This  mapping space is the nerve of the
overcategory $(\Nec\ovcat E^n_{i,j})$.  

Observe that if $T$ is a necklace then any map $T\ra E^n$ extends
uniquely over $\Delta[T]$.  This is because maps into $E^n$ are
determined by what they do on the $0$-skeleton, and $T\inc \Delta[T]$ is an
isomorphism on $0$-skeleta.  

Consider two functors
\[ f,g\colon (\Nec\ovcat E^n_{i,j}) \ra (\Nec\ovcat E^n_{i,j}) \]
given by
\[ f\colon [T,T\llra{x} E^n] \mapsto
[\Delta[T],\Delta[T]\llra{\bar{x}} E^n]
\ \text{and}\ 
g\colon [T,T\llra{x} E^n] \mapsto [\Delta^1,\Delta^1\llra{z} E^n].
\]
Here $\bar{x}$ is the unique extension of $x$ to $\Delta[T]$, and $z$
is the unique $1$-simplex of $E^n$ connecting $i$ to $j$.  Observe
that $g$ is a constant functor.

It is easy to see that there are natural transformations $\id \ra f\la
g$.  The functor $g$  factors through the terminal category $\{*\}$, so after taking nerves the identity map is null homotopic.  
Hence $(\Nec\ovcat E^n_{i,j})$ is contractible.

\end{proof}

For completeness (and because it is short) we include the following
lemma, established in \cite[Proof of 2.2.5.1]{L}:

\begin{lemma}\label{lemma:jC preserves cofs}

The functor $\jC\taking\sSet\to\sCat$ takes monomorphism to  cofibrations.

\end{lemma}

\begin{proof}
Every monomorphism in $\sSet$ is obtained by compositions and cobase
changes from boundary inclusions of simplices.  It therefore suffices
to show that for each $n\geq 0$ the map
$\jC(\bd{n})\to\jC(\Delta^n)$ is a cofibration in $\sCat$.
Let $0\leq i,j\leq n$.  If $i>0$ or $j<n$ then every map $T\ra
\Delta^n_{i,j}$, where $T$ is a necklace, actually factors through
$\bdd{n}_{i,j}$.  It follows that
$(\Nec\ovcat\Delta^n_{i,j})\iso(\Nec\ovcat\partial\Delta^n_{i,j})$,
and therefore
$$\jC(\bd{n})(i,j)\to\jC(\Delta^n)(i,j)$$ is an
isomorphism by Proposition~\ref{pr:jC-main}.  When
$i=0$ and $j=n$, the simplicial set $\jC(\Delta^n)(i,j)$ is identified
with the cube $(\Delta^1)^{n-1}$ by Lemma~\ref{le:C(Delta^n)}, 
and it is easy to see that $\jC(\bd{n})(i,j)$
is precisely the boundary of this cube.

To summarize the above paragraph, the mapping spaces in
$\jC(\bd{n})$ and $\jC(\Delta^n)$ are identical except for the mapping
space from $0$ to $n$, and in that case the inclusion of mapping
spaces is the boundary inclusion for the cube $(\Delta^1)^{n-1}$.
Let $b$ denote this boundary inclusion.

Let $U\taking\sSet\to\sCat$ denote the functor which sends a
simplicial set $S$ to the unique simplicial category $U(S)$ with two
objects $x,y$ and morphisms $\Hom(x,x)=\Hom(y,y)=\{*\}$,
$\Hom(y,x)=\emptyset$, and $\Hom(x,y)=S$. 
There is an evident map $U(\partial((\Delta^1)^{n-1}))\ra \jC(\bdd{n})$
sending $x$ to $0$ and $y$ to $n$, and pushing out $U(b)$ along this
map precisely gives $\jC(\Delta^n)$.  

In view of the generating
cofibrations for $\sCat$ (see \cite{Bergner}), it is easy to show that
$U$ takes monomorphisms to cofibrations.  Hence $U(b)$ is a
cofibration.  Since our map $\jC(\bdd{n})\ra \jC(\Delta^n)$ is a
pushout of $U(b)$, it too is a cofibration.
\end{proof}

Recall that a simplicial category is fibrant in $\sCat$ if all its mapping spaces are Kan fibrant.

\begin{lemma}\label{lemma:N preserves fibrants}
If $\cD$ is a fibrant simplicial category then $N\cD$ is a quasi-category.
\end{lemma}

\begin{proof}
By adjointness it
suffices to show that each $\jC(j^{n,k})$ is an acyclic cofibration in
$s\Cat$, where $j^{n,k}\colon \Lambda^n_k \inc \Delta^n$ is an inner
horn inclusion ($0<k<n$).  It is a cofibration by Lemma~\ref{lemma:jC
preserves cofs}, so we must only verify that it is a weak equivalence.
Just as in the proof of (\ref{lemma:jC preserves cofs}) above,
$\jC(\Lambda^n_k)(i,j) \ra \jC(\Delta^n)(i,j)$ is an isomorphism
unless $i=0$ and $j=n$.  It only remains to show that
$\jC(\Lambda^n_k)(0,n) \ra
\jC(\Delta^n)(0,n)$ is a Kan equivalence..  An analysis as in
Example~\ref{ex:computation} identifies $\jC(\Lambda^n_k)(0,n)$ with
the result of removing one face from
the boundary of $(\Delta^1)^{n-1}$, which clearly has the same
homotopy type as the cube $(\Delta^1)^{n-1}$.
\end{proof}



\begin{prop}
\label{pr:hoinvariance}
If  $S\ra S'$ is a map of simplicial sets which is a Joyal equivalence then $\jC(S)\ra
\jC(S')$ is a weak equivalence of simplicial categories.
\end{prop}


\begin{proof}

For any simplicial set $X$, the map $\jC(X\times E^n)\ra \jC(X)$
induced by projection is a weak equivalence in $\sCat$.  This follows
by combining Proposition~\ref{pr:product} with Lemma~\ref{le:C(En)}:
\[ \jC(X\times E^n) \llra{\sim} \jC(X)\times \jC(E^n) \llra{\sim}
\jC(X).
\]
Since $X\amalg X \inc X\times E^1$ is a cofibration in $\sSet$, 
 $\jC(X)\amalg \jC(X)=\jC(X\amalg X) \ra \jC(X\times
E^1)$ is a cofibration in $\sCat$, by Lemma \ref{lemma:jC preserves cofs}.  It follows that $\jC(X\times E^1)$
is a cylinder object for $\jC(X)$ in $\sCat$.  So if $\cD$ is a
fibrant simplicial category we may compute
homotopy classes of maps $[\jC(X),\cD]$ as the coequalizer
\[ \coeq\Bigl ( \sCat(\jC(X\times E^1),\cD) \dbra
\sCat(\jC(X),\cD)\Bigr ).
\]
But using the adjunction, this is isomorphic to
\[ \coeq\Bigl ( \sSet(X\times E^1,N\cD) \dbra \sSet(X,N\cD) \Bigr
).
\]
The above coequalizer is
$[X,N\cD]_{E^1}$, and we have identified 
\begin{myequation}
\label{eq:homotopy adjunction} [\jC(X),\cD]\iso [X,N\cD]_{E^1}.
\end{myequation}

Now let $S\ra S'$ be a Joyal equivalence.   Then $\jC(S)\ra
\jC(S')$ is a map between cofibrant objects of $\sCat$.  To prove that
it is a weak equivalence in $\sCat$ it is sufficient to prove that the
induced map on homotopy classes
\[ [\jC(S'),\cD]\ra [\jC(S),\cD]
\]
is a bijection, for every fibrant object $\cD\in \sCat$.  Since $N\cD$
is a quasi-category by Lemma~\ref{lemma:N preserves fibrants} 
and $S\ra S'$ is a Joyal equivalence, we have 
that $[S',N\cD]_{E^1}\to[S,N\cD]_{E^1}$ is a bijection;  the result then follows
by (\ref{eq:homotopy adjunction}).
\end{proof}

\begin{remark}
In fact it turns out that a map of simplicial sets $S\ra S'$ is a
Joyal equivalence {\it if and only if\/} $\jC(S)\ra \jC(S')$ is a weak
equivalence of simplicial categories.  This was proven in \cite{L}, and
will be reproven in \cite{DS} using an extension of the methods from
the present paper.  
\end{remark}

%%%%%%%%%%%%%%%%%%%%%%%%%%%%%%%%%%%%%%%%%%%%%%%%%%%%%%%%%%%%%%%%%%%%%%


%%%%%%%%%%%%%%%%%%%%%%%%%%%%%%%%%%%%%%%%%%%%%%%%%%%%%%%%%%%%%%%%%%%%%%

\appendix

\section{Leftover proofs}
\label{se:appendix}
In this section we give two proofs which were postponed in the body
of the paper.  

\subsection{Products of necklaces}

Our first goal is to prove Proposition~\ref{pr:products=gadgets}.  Let
$T_1,\ldots,T_n$ be necklaces, and consider the product
$X=T_1\cross\cdots\cross T_n$.  The main thing we need to prove is
that whenever $a\poleq_X b$ in $X$ the mapping space
$\jC(X)(a,b)\simeq *$ is contractible.  For this we use two lemmas.

Let $Z$ be an ordered simplicial set, and let $u,v\in Z$.  Suppose
there is a finite set of $0$-simplices $A=\{a_1,\ldots,a_n\}$ of $Z$,
and we know that every inclusion $T\inc Z_{u,v}$, where $T$ is a
necklace, has at least one joint lying in $A$.  It is useful to think
of $T\inc Z_{u,v}$ as a ``generalized path'' from $u$ to $v$, and of
the vertices in $A$ as ``gates''.  Our assumption is that every
generalized path must pass through at least one gate.  One can then
stratify all such paths, according to which gates they pass through.
We will explain a way to understand the homotopy type of $\jC(Z)(u,v)$
by writing it as a homotopy colimit of `smaller' spaces associated to
this stratification.

To this end, consider the poset $A_0$ of vertices of $A$ under the
relation $\poleq$ (this is a poset because $A$ is an ordered
simplicial set).  Let $P$ denote the collection of linearly ordered
subsets $S$ of $A_0$ having the property that $u\poleq s\poleq v$ for
all $s\in S$.  That is, each element of $P$ is a chain $u\poleq s_1
\poleq\cdots \poleq s_n\poleq v$ where each $s_i\in A$.  We regard $P$
as a category, where the maps are inclusions.  Also let $P_0$ denote
the subcategory of $P$ consisting of all subsets except $\emptyset$.

Define a functor $H_{u,v}\colon P^{op}\ra \sSet$ by
\[ H_{u,v}(S)=\jC(Z)(u,s_1)\times \jC(Z)(s_1,s_2)\times \cdots \times
\jC(Z)(s_{n-1},s_n)\times \jC(Z)(s_n,v),
\]
where by convention we have $H_{u,v}(\emptyset)=\jC(Z)(u,v)$.  

\begin{lemma}
\label{le:decompose}
Let $Z$ be an ordered simplicial set, and let $A\subseteq Z_0$ be a
finite subset. Let $u,v\in Z$ and assume that every map $T\ra
Z_{u,v}$, where $T$ is a necklace, has at least one joint mapping into
$A$.  Then the composite map 
\[ \hocolim_{S \in P_0^{op}} H_{u,v}(S) \ra
 \colim_{S \in P_0^{op}} H_{u,v}(S) \ra
H_{u,v}(\emptyset)=\jC(Z)(u,v) \]
is a Kan equivalence.  
\end{lemma}

\begin{proof}
Define a functor $F\taking P^{op}\to\Cat$
by sending $S\in P$ to 
\[\{ [T,T\inj Z_{u,v}]\hspace{.1in} |
\hspace{.1in} S\subseteq J_T\}, 
\] 
the full subcategory of
$(\Nec\ovcat Z_{u,v})$ spanned by objects $T\To{m} Z_{u,v}$ for which
$m$ is an injection and $S\subseteq J_T$.
Let us adopt the notation
\[ M_S(u,v)=\colim_{T\in F(S)} \jC(T)(\alpha,\omega).
\]
This gives a functor $M_{(\blank)}(u,v)\colon P^{op}\ra \sSet$.  
Note that there is a natural map
\[ M_\emptyset(u,v) \lra \colim_{T\in (\Nec\ovcat S_{u,v})}
\jC(T)(\alpha,\omega) \iso \jC(Z)(u,v).
\]
The first map is not {\it a priori\/} an isomorphism because in the
definition of $F(\emptyset)$ we require that the map $T\ra Z$ be an
injection.  However, using Lemma~\ref{le:jCL-necklace} (or
Corollary~\ref{cor:maps in ordered}) it follows at once that the map
actually is an isomorphism.    

From here the argument proceeds as follows.  We will show:
\begin{enumerate}[(i)]
\item The natural map $\hocolim_{S\in P_0^{op}} M_S(u,v)\ra
\colim_{S\in P_0^{op}} M_S(u,v)$ is a Kan equivalence;
\item The map $\colim_{S\in P_0} M_S(u,v)\ra M_\emptyset(u,v)$ is an isomorphism;
\item The functor $M_{(\blank)}(u,v)$ is naturally isomorphic to
$H_{u,v}$.
\end{enumerate}
These three items will clearly complete the proof.

For (i) we refer to \cite[Section 13]{D} and use the fact that
$P_0^{op}$ has the structure of a directed Reedy category.  Indeed, we
can assign a degree function to $P$ that sends a set $S\subseteq A_0$
to the nonnegative integer $|A_0-S|$; all non-identity morphisms in
$P_0^{op}$ strictly increase this degree.  By \cite[Proposition
13.3]{D} (but with $\Top$ replaced by $\sSet$) it is enough to show
that all the latching maps $L_S$ are cofibrations, where $L_S$ is the
map $$ L_S\colon \colim_{S'\supset S} M_{S'}(u,v) \ra M_S(u,v) $$ and
the colimit is over sets $S'\in P$ which strictly contain $S$.  To see
that $L_S$ is a cofibration, suppose that one has a triple $[T,T\inc
Z_{u,v},t\in \jC(T)(\alpha,\omega)_n]$ representing an $n$-simplex of
$M_{S'}(u,v)$ and another triple $[T',T'\inc B_{u,v},t'\in
\jC(U)(\alpha,\omega)_n]$ representing an $n$-simplex of
$M_{S''}(u,v)$.  If these become identical in $M_S(u,v)$ then it must
be that they have the same flankification $\bar{T}=\bar{U}$ and
$t=t'$.  Note that every joint of $T$ is a joint of $\bar{T}$, so the
joints of $\bar{T}$ include both $S'$ and $S''$.  Because the joints
of any necklace are linearly ordered, it follows that $S'\cup S''$ is
linearly ordered.  Since $T\to\bar{T}$ is an injection, we may
consider the triple $[\bar{T},\bar{T}\inc Z_{u,v},t]$ as an
$n$-simplex in $M_{S'\cup S''}(u,v)$, which maps to the two original
triples in the colimit; this proves injectivity.

Assertion (ii) is the claim
that the latching map $L_\emptyset\colon \colim_{S\in
P_0^{op}} M_S(u,v)\ra M_\emptyset(u,v)$ is an isomorphism.
Injectivity was established above.  For surjectivity, one needs to
know that if
$T$ is a necklace and $T\inc Z_{u,v}$ is an inclusion, then $T$ must
contain at least one vertex of $A$ as a joint.  But this is precisely
our assumption on $A$.

Finally, for (iii) fix some $S\in P_0$ and let $u=a_0\prec a_1\prec\ldots\prec
a_n\prec a_{n+1}=v$ denote the complete set of elements of
$S\cup\{u,v\}$.
 A necklace $T\inj
Z_{u,v}$ whose joints include the elements of $S$ can be split along
the joints, and thus uniquely written as the wedge of necklaces
$T_i\inj Z_{a_i,a_{i+1}}$, one for each $0\leq i\leq n$.  Under this
identification, one has
$$
\jC(T)(\alpha,\omega)\iso \jC(T_0)(\alpha_0,\omega_0)
\cross\cdots\cross\jC(T_n)(\alpha_n,\omega_n).
$$ 
Thus $F(S)$ is isomorphic to the category $$
(\Nec\ovcat^m Z_{u,a_1})\cross(\Nec\ovcat^m
Z_{a_1,a_2})\cross\cdots\cross(\Nec\ovcat^m
Z_{a_{n-1},a_n})\cross(\Nec\ovcat^m Z_{a_n,v}), $$ where
$(\Nec\ovcat^m Z_{s,t})$ denotes the category whose objects are $[T,T\ra Z_{s,t}]$
where the map $T\ra Z$ is a monomorphism.

Now, it is a general fact
about colimits taken in the category of (simplicial) sets,
that if $K_i$ is a category and
$Q_i\taking K_i\to\sSet$ is a functor, for each $i\in \{1,\ldots,n\}$,
then there is an isomorphism of simplicial sets
\begin{align}\tag{A.2.2}\label{dia:colimits over products}
\colim_{K_1\cross\cdots\cross K_n} (Q_1\cross\cdots\cross Q_n)
\To{\iso}\left(\colim_{K_1}Q_1\right)\cross\cdots\cross\left(\colim_{K_n}Q_n\right).\end{align}
Applying this in our case, we find that
\begin{align*} 
M_S(u,v) & \iso \jC(Z)(u,a_1)\times \jC(Z)(a_1,a_2)\times \cdots
\times \jC(Z)(a_{n-1},a_{n})\times \jC(Z)(a_{n},v)\\
& =H_{u,v}(S).
\end{align*}
This isomorphism is readily checked to be natural in $S$, so
this proves (iii) and completes the argument.

\end{proof}



\begin{defn}

An ordered simplicial set $(X,\poleq)$ is called \dfn{strongly
ordered} if, for all $a\poleq b$ in $X$, the mapping space
$\jC(X)(a,b)$ is contractible.

\end{defn}

Note that in any ordered simplicial set $X$ with $a,b\in X_0$, we have
$a\poleq b$ if and only if $\jC(X)(a,b)\neq\emptyset$.  Thus if $X$ is
strongly ordered then its structure as a simplicial category, up
to weak equivalence, is completely determined by the ordering on
its vertices.  We also point out that every necklace $T\in\Nec$ is
strongly ordered by Corollary~\ref{cor:mapping space is cube}.

\begin{lemma}\label{lemma:strongly ordered}

Suppose given a diagram $$X\From{f} A\To{g} Y$$ where $X,Y,$ and $A$
are strongly ordered simplicial sets and both $f$ and $g$ are simple
inclusions. 
Let $B=X\amalg_A Y$ and assume 
the following conditions hold:
\begin{enumerate}[(1)]
\item $A$ has finitely many vertices;
\item  Given any $x\in X$, the set $A_{x\poleq}=\{a\in A \,|\, x\poleq_B a\}$ has
an initial element (an element which is smaller than every other
element);
\item For any $y\in Y$ and $a\in A$, if $y\poleq_Y a$ then $y\in A$.
\end{enumerate}
Then $B$
is strongly ordered.
\end{lemma}

\begin{proof}
By Lemma~\ref{lemma:pushout of simple inclusion}, $B$ is an ordered
simplicial set and the maps $X\inc B$ and $Y\inc B$ are simple
inclusions.  We must show that for $u,v\in B_0$ with $u\poleq v$, the
mapping space $\jC(B)(u,v)$ is contractible.  Suppose that $u$ and $v$
are both in $X$; then since $X\inc B$ is simple, any necklace $T\to
B_{u,v}$ must factor through $X$.  It follows from Proposition~\ref{pr:jC-main} that
$\jC(B)(u,v)=\jC(X)(u,v)$, which is contractible since $X$ is strongly
ordered.  The case $u,v\in Y$ is analogous.  We claim we cannot have
$u\in Y\backslash A$ and $v\in X\backslash A$.  For if this is so and
if $T\ra B$ is a spine connecting $u$ to $v$, then there is a last
vertex $j$ of $T$ that maps into $Y$.  The $1$-simplex leaving that
vertex then cannot belong entirely to $Y$, hence it belongs entirely
to $X$.  So $j$ is in both $X$ and $Y$, and hence it is in $A$.  Then
we have $u\poleq j$ and $j\in A$, which by assumption (3) implies
$u\in A$, a contradiction.

It remains to show that if $u\in X$, $v\in Y\backslash A$, and
$u\poleq_B v$ then $\jC(B)(u,v)$ is contractible.  We claim that any map
$T\ra B_{u,v}$, where $T$ is a necklace, must send at least one joint
of $T$ into $A$.  To see this, recall that every simplex of $B$ either
lies entirely in $X$ or entirely in $Y$.  Since $v\notin X$, there is
a last joint $j_1$ of $T$ which maps into $X$.  If $C$ denotes the
bead whose initial vertex is $j_1$, then the image of $C$ cannot lie
entirely in $X$; so it lies entirely in $Y$, which means that $j_1$
belongs to both $X$ and $Y$---hence it belongs to $A$.  

The preceding paragraph shows that we may apply
Lemma~\ref{le:decompose} to write
\begin{myequation}
\label{eq:H1}
 \jC(B)(u,v)\he \hocolim_{S\in P_0^{op}} H_{u,v}(S) 
\end{myequation}
where $P_0$ and $H_{u,v}(S)$ are defined prior to that lemma.  

For each $S$ in $P_0$ we write $S=\{s_1,\ldots,s_n\}$ for $s_i\in A$
with $u\prec s_1 \prec s_2 \prec \cdots \prec s_n\prec v$.  Then
\[
 H_{u,v}(S)= \jC(B)(u,s_1)\times \jC(B)(s_1,s_2) \times\cdots\times
\jC(B)(s_{n-1},s_n)\times \jC(B)(s_n,v).
\]
Because $X\inc B$ is a simple inclusion, $\jC(X)(u,s_1)\ra
\jC(B)(u,s_1)$ is a Kan equivalence.  For the same reason, the maps
$\jC(X)(s_i,s_{i+1}) \ra \jC(B)(s_i,s_{i+1})$ and $\jC(Y)(s_n,v)\ra
\jC(B)(s_n,v)$ are Kan equivalences (in the latter case using that
$Y\inc B$ is also a simple inclusion).  From the assumption that $X$
and $Y$ are strongly ordered we now have that all these mapping spaces
are contractible.  Hence $H_{u,v}(S)$ is contractible
(this uses a special property of $\sSet$, namely that a finite product
of contractible spaces is contractible; this is an easy consequence of
the fact that geometric realization preserves finite products).  

At this point we know that the homotopy colimit on the right side of
(\ref{eq:H1}) is Kan equivalent to the nerve of $P_0^{op}$.  It thus
suffices to prove that the nerve of $P_0$ (and hence also the nerve of
$P_0^{op}$) is contractible.
For this, write $\theta$ for the initial vertex of $A_{u\poleq}$.
Define a functor $F\colon P_0 \ra P_0$ by $F(S)=S\cup \{\theta\}$;
note that $S\cup\{\theta\}$ will be linearly ordered, so this makes
sense.  Clearly there is a natural transformation from the identity
functor to $F$, and also from the constant $\{\theta\}$ functor to $F$.
It readily follows that the identity map on $NP_0$ is homotopic to a
constant map, hence $NP_0$ is contractible.

\end{proof}

\begin{prop}\label{prop:product of necklaces}

Let $T_1,\ldots,T_m$ be necklaces.  Then their product
$T_1\cross\cdots\cross T_m$ is a strongly ordered simplicial set.

\end{prop}

\begin{proof}

We begin with the case $P=\Delta^{n_1}\cross\cdots\cross\Delta^{n_m}$,
where each necklace is a simplex, and show that $P$ is strongly
ordered.  It is ordered by Lemma~\ref{lemma:facts on ordered}, so
choose vertices $a,b\in P_0$ with $a\poleq b$.  If $T$ is a necklace,
any map $T\ra \Delta^j$ extends uniquely to a map
$\Delta[T]\ra \Delta^j$.  It follows that any map $T\ra P_{a,b}$
extends uniquely to $\Delta[T]\ra P_{a,b}$.  Consider the two
functors
\[ f,g\colon (\Nec\ovcat P_{a,b}) \ra (\Nec\ovcat P_{a,b}) \]
where $f$ sends $[T,T\ra P]$ to $[\Delta[T],\Delta[T]\ra P]$ and
$g$ is the constant functor sending everything to
$[\Delta^1,x\colon\Delta^1 \ra P]$ where $x$ is the unique edge of $P$
connecting $a$ and $b$.  Then clearly there are natural
transformations $\id \ra f$ and $g\ra f$, showing that the three maps
$\id$, $f$, and $g$ induce homotopic maps on the nerves.  So the
identity induces the null map, hence 
$\jCn(P)(a,b)=N(\Nec\ovcat P_{a,b})$ is contractible.  The result for
$P$ now follows by  Theorem~\ref{th:main}.

For the general case, assume by induction that we know the
result for all products of necklaces in which at most $k-1$ of them are
not equal to beads.  The case $k=1$ was handled by the previous
paragraph.
Consider a product
\[ Y=T_1\times \cdots \times T_k \times D\]
where each $T_i$ is a necklace and $D$ is a product of beads.
Write $T_k=B_1\Wedge B_2\Wedge\cdots\Wedge B_r$ where each $B_i$ is a
bead, and let
\[ P_{j}=(T_1\cross\cdots\cross
T_{k-1})\cross(B_{1}\vee\cdots\vee B_{j})\cross D.
\]
We know by induction that $P_1$ is strongly ordered, and we will prove
by a second induction that the same is true for each $P_j$.  So assume
that $P_j$ is strongly ordered for some $1\leq j<r$.

Let us denote $A=(T_1\cross\cdots\cross
T_{k-1})\cross\Delta^0\cross D$
and 
\[ Q=(T_1\cross\cdots\cross
T_{k-1})\cross B_{j+1}\cross D.
\]
Then we have $P_{j+1}=P_{j}\amalg_A Q$, and we know that $P_{j}, A$,
and $Q$ are strongly ordered.  Note that the maps $A\to P_{j}$ and
$A\to Q$ are simple inclusions: they are the products of the
last-vertex-map $\Delta^0\to
B_1\Wedge \cdots \Wedge B_{j}$ (resp. the initial-vertex-map
$\Delta^0\to B_{j+1}$) 
with identity maps, and
any map $\Delta^0\to V$, where $V$ is a necklace, is clearly simple.
It is easy to check that hypotheses (1)--(3) of
Lemma~\ref{lemma:strongly ordered} are satisfied, with $P_j$ playing
the role of $X$ and $Q$ playing the role of $Y$.  This finishes the proof.
\end{proof}


\begin{proof}[Proof of Proposition~\ref{pr:products=gadgets}]
This follows immediately from Proposition~\ref{prop:product of necklaces}.
\end{proof}


\subsection{The category $\jC(\Delta^n)$}

Our final goal is to give the proof of Lemma~\ref{le:C(Delta^n)}.
We must construct isomorphisms $$\jC(\Delta^n)(i,j)\to
N(P_{i,j})$$ for $n\in \N$ and $0\leq i,j\leq n$, where $P_{i,j}$ is
the poset of subsets of $\{i,i+1,\ldots,j\}$ containing $i$ and $j$.
Moreover, we must verify that these isomorphisms are compatible with
composition, thereby giving an isomorphism of simplicial categories
$\jC(\Delta^n)\ra NP$.  

\begin{proof}[Proof of Lemma~\ref{le:C(Delta^n)}]

The result is obvious when $n=0$, so we assume $n>0$.  
Let $m_i$ denote the unique map in $[n]$ from $i$ to $i+1$.  
One understands $FU([n])(i,j)$ as the set of free compositions of
sequences of morphisms in $[n]$ which start at $i$ and end at $j$.  
Such free compositions are in one-to-one correspondence with
the set of ways to ``parenthesize'' the word $m_jm_{j-1}\cdots m_{i+1}$ in
such a way that each $m_k$ is contained in exactly one parenthesis.
For example, when $n=3$ the maps in $FU([3])(0,3)$ are
\[ (m_3m_2)(m_1),\quad  (m_3)(m_2m_1), \quad (m_3m_2m_1),\quad
(m_3)(m_2)(m_1).
\]
To such a parenthesization we associate the subset of $\{i,i+1,\ldots,j\}$
consisting of $i$ together with all indices that occur directly after
a left parenthesis.  The subsets of $\{0,1,2,3\}$ corresponding to
the parenthesizations listed above, in order, are
\[ \{0,1,3\}, \quad \{0,2,3\}, \qquad \{0,3\}, \qquad \{0,1,2,3\}.\]
It is easy to see that this gives a bijection between the maps in
$FU([n])(i,j)$ and 
subsets of $\{i,i+1,\ldots,j\}$ containing $i$ and $j$.


Now let $X=(FU)_\bullet([n])(i,j)$ and $Y=P_{i,j}$.  For each $\ell\in\N$, we
will provide an isomorphism $X_\ell\iso Y_\ell$, and these will be
compatible with face and degeneracy maps.  We have already done this
when $\ell=0$.  

For $\ell>0$ one has that $X_\ell$ is the set of free compositions of
sequences of morphisms in $X_{\ell-1}$.  It is readily seen that
$X_\ell$ is in one-to-one correspondence with the set of ways to
parenthesize the word $m_jm_{j-1}\ldots m_{i+1}$ in such a way that
every element is contained in exactly $(\ell+1)$-many parentheses (and
no closed parenthesis directly follows an open parenthesis).  For
example, the nine elements of $(FU)^2([3])(0,3)$ are
\[ 
\bigl((m_3m_2m_1)\bigr), \quad
\bigl ((m_3m_2)\bigr)\bigl((m_1)\bigr),\quad 
\bigl((m_3))((m_2m_1)\bigr),\quad
\bigl
((m_3)\bigr )\bigl ((m_2)\bigr)\bigl((m_1)\bigr ),
\]
\[ \bigl ((m_3)(m_2)\bigr )\bigl ((m_1)\bigr ),\quad  
\bigl((m_1)(m_2)(m_3)\bigr),
\quad \bigl((m_3)\bigr)\bigl((m_2)(m_1)\bigr), \quad 
\bigl((m_3)(m_2m_1)\bigr), 
\]
\[\text{and}\quad  \bigl((m_3m_2)(m_1)\bigr).
\]


Given such a parenthesized sequence, one can rank the parentheses by
``interiority" (so that interior parentheses have higher rank).  The
face and degeneracy maps on $X$ are given by deleting or repeating all
the parentheses of a fixed rank.  Under this description, a vertex in
an $\ell$-simplex of $X$ is given by choosing a rank and then ignoring
all parentheses except those of that rank.  Such a vertex determines a
subset of $\{i,i+1,\ldots,j\}$ containing $i$ and $j$, as in the first
paragraph of this proof.  
And given two ranks, the subset of $\{i+1,\ldots,j\}$ corresponding to the
higher rank will contain the subset corresponding to the lower rank
(due to the nested nature of the parentheses).  

One can check that an $\ell$-simplex in $X$ is determined by its set
of vertices, and so we can identify $X_\ell$ with the set of sequences
$S_0\subseteq S_1\subseteq\cdots\subseteq
S_\ell\subseteq\{i,i+1,\ldots,j\}$ containing $i$ and $j$.  This is
precisely the set of $\ell$-simplices of $Y$, so we have our
isomorphism $X_l\iso Y_l$.  It is clearly compatible with face and
degeneracy maps.  

Finally, we point out that the composition in the category
$\jC(\Delta^n)$ corresponds to concatenation of parenthesized words.
From this one readily checks that the above isomorphisms give a map
(thus, an isomorphism) of categories $\jC(\Delta^n)\ra NP$.
\end{proof}

%%%%%%%%%%%%%%%%%%%%%%%%%%%%%%%%%%%%%%%%%%%%%%%%%%%%%%%%%%%%%%%%%%%%%%%%%%%%%%%%%%%%%%%

\bibliographystyle{amsalpha}
\begin{thebibliography}{JTT}
\bibitem[AM]{AM} M. Artin and B. Mazur, {\em \'Etale homotopy\/}, Lecture
Notes in Math. {\bf 100}, Springer-Verlag, Berlin-New York, 1969.

\bibitem[B]{Bergner} J. Bergner, {\em A model category structure on
the category of simplicial categories\/}, Trans. Amer. Math. Soc. {\bf
359} (2007), no. 5, 2043--2058.

\bibitem[Ba]{Ba} H.J. Baues, {\em Geometry of loop spaces and the cobar construction.}  Mem. Amer. Math. Soc. 25 (1980), no. 230.

\bibitem[BK]{BK} A. K. Bousfield and D. M. Kan, {\em Homotopy limits,
completions, and localizations\/}, Springer-Verlag, Berlin and Heidelberg,
1972.

\bibitem[C]{C} J-M Cordier, {\em Sur la notion de diagramme homotopiquement coh\'{e}rent.}  Third Colloquium on Categories, Part VI (Amiens, 1980).  Cahiers Topologie G�om. Diff�rentielle 23 (1982), No. 1, 93--112.
 

\bibitem[CP]{CP} J-M Cordier; T. Porter, {\em Vogt's theorem on categories of homotopy coherent diagrams.}  Math. Proc. Cambridge Philos. Soc. 100 (1986), no. 1, 65--90. 

%\bibitem[CP]{CP} J-M Cordier, T. Porter, {\em Homotopy coherent
%  category theory}, Trans. Amer. Math. Soc. {\bf 349} (1997), no. 1,
%  1-54.

\bibitem[D]{D} D. Dugger, {\em A primer on homotopy colimits},
preprint, http://math.uoregon.edu/$\sim$ddugger.


\bibitem[DS]{DS} D. Dugger and D.I. Spivak, {\em Mapping spaces in
quasi-categories\/}, preprint, 2009.

\bibitem[DK]{DK} W.~G. Dwyer and D.~M. Kan, {\em Function complexes in
homotopical algebra\/}, Topology {\bf 19} (1980), 427--440.

\bibitem[H]{H} P.~J. Hirschhorn, {\em Model categories and
localizations\/}, Mathematical Surveys and Monographs {\bf 99},
American Mathematical Society, Providence, RI, 2003.



\bibitem[J1]{J1} A. Joyal, {\em Quasi-categories and Kan complexes\/},
J. Pure Appl. Algebra {\bf 175} (2002), no. 1--3, 207--222.

\bibitem[J2]{J2} A. Joyal, {\em The theory of quasi-categories\/}, preprint.

\bibitem[L]{L} J. Lurie, {\em Higher topos theory\/}, Annals of
Mathematics Studies {\bf 170}, Princeton University Press, Princeton,
NJ, 2009.

\bibitem[ML]{ML} S. Mac\,Lane, {\em Categories for the working mathematician\/}, second edition.  Graduate texts in mathematics {\bf 5}, Springer-Verlag, New York, 1998.

\bibitem[Q]{Q} D. Quillen, {\em Higher algebraic $K$-theory. I.\/},
Algebraic $K$-theory, I: Higher $K$-theories (Proc. Conf., Batelle
Memorial Inst., Seattle, Wash., 1972), pp. 85--147.  Lecture Notes in
Math. {\bf 341}, Springer, Berlin, 1973.    
\end{thebibliography}

\end{document}



