\documentclass{beamer}
\usetheme{CambridgeUS}


\usepackage{url}
\input xy
\xyoption{all} \xyoption{poly} \xyoption{knot}\xyoption{curve}\xyoption{color}
\def\mcB{\mathcal{B}}
\newcommand{\mcC}{\mathcal{C}}
\newcommand{\mcD}{\mathcal{D}}
\def\mcE{\mathcal{E}}
\newcommand{\mcG}{\mathcal{G}}
\def\RR{{\mathbb R}}
\newcommand{\Ob}{{\bf Ob}}
\newcommand{\Arr}{{\bf Arr}}
\newcommand{\Fun}{{\bf Fun}}
\newcommand{\taking}{\colon}
\newcommand{\too}{\rightarrow}
\newcommand{\toooo}{\longrightarrow}
\newcommand{\To}{\xrightarrow}
\newcommand{\tto}{\rightrightarrows}
\def\from{\leftarrow}
\newcommand{\From}{\xleftarrow}
\newcommand{\id}{\textnormal{id}}
\newcommand{\im}{\textnormal{im}}
\newcommand{\Sets}{{\bf Set}}
\newcommand{\Set}{{\bf Set}}
\newcommand{\set}{{\text \textendash}{\bf Set}}
\newcommand{\Cat}{{\bf Cat}}
\newcommand{\Strings}{{\bf Strings}}
\newcommand{\NN}{{\mathbb N}}
\newcommand{\cross}{\times}
\newcommand{\iso}{\cong}
\newcommand{\sss}{\subset}
\newcommand{\tn}[1]{\textnormal{#1}}
\newcommand{\hsp}{\hspace{.2in}}
\newcommand{\comment}[1]{}
\newcommand{\bD}{{\bf \Delta}}
\def\Hask{{\bf Hask}}
\def\CPO{{\bf CPO}}
\newcommand{\obox}[3]{\stackrel{#1}{\fbox{\parbox{#2}{#3}}}}
\newcommand{\labox}[2]{\obox{#1}{1.6in}{#2}}
\newcommand{\mebox}[2]{\obox{#1}{1in}{#2}}
\newcommand{\smbox}[2]{\stackrel{#1}{\fbox{#2}}}
\newcommand{\LA}[2]{\ar[#1]^-{\tn {#2}}}
\newcommand{\LAL}[2]{\ar[#1]_-{\tn {#2}}}




\newcommand{\LMO}[1]{\bullet^{#1}}
\newcommand{\LTO}[1]{\bullet^{\tn{#1}}}



%%%%%%%%%%%%%%%%%%

\begin{document}

\title{Databases are Categories II}
\subtitle{Refinements and Extensions}

\author{David I. Spivak}

\date{Presented on 2010/10/22}

\institute[MIT]{
  \url{dspivak@math.mit.edu}\\
  Mathematics Department\\
  Massachusetts Institute of Technology \\

}


\frame{\titlepage}

\newcommand{\fst}[2]{\subsection{#1}\frame{\frametitle{#1} #2}}


\section{Introduction}

\fst{Last time}{

\begin{itemize}
	\item On June 3, 2010 I gave a talk here at Galois called ``Databases are categories."
	\item The main idea was that:
	\begin{itemize}
		\item a database schema is a category $\mcC$, and
		\item a state on that schema is a functor $\delta\taking\mcC\too\Set$.
	\end{itemize}
	\item Here is a quick review:
\end{itemize}
}

\frame{\frametitle{Review of basic category theory}
\begin{itemize}
	\item A category $\mcC$ is a system of objects and arrows, and a composition law.
	\item A functor $\mcC\too\mcD$ is a mapping that preserves these structures.
	\item $\Set$ is the category whose objects are sets, whose arrows are (total) functions, and whose composition law is the specification of how functions compose.
	\item If $\mcC$ is the category on the left below, then a functor $\delta\taking\mcC\too\Set$ might look like this: $$\mcC:=\fbox{\xymatrix{A\ar[r]^f\ar[d]_g&B\\C}};\hspace{.2in} \delta=\fbox{\xymatrix{\fbox{$\bullet_{a_1}\bullet_{a_2}\bullet_{a_3}$}\ar[rr]^{(b_1,b_1,b_2)}\ar[d]&&\fbox{$\bullet_{b_1}\bullet_{b_2}$}\\\fbox{$\bullet_{c_1}$}}}$$
	
\end{itemize}

}

\frame{\frametitle{How databases fit in}

A category $\mcC$ as a schema: Each object $A\in\Ob(\mcC)$ is a table, each arrow $A\too B$ is a foreign key column of table $A$ pointing to table $B$.
\tiny \begin{align*}&\begin{tabular}{| l || l | l | l | l |}\hline\multicolumn{5}{| c |}{Employee}\\\hline {\bf Id}&{\bf First}&{\bf Last}&{\bf Mgr}&{\bf Dpt}\\\hline 101&David&Hilbert&103&q10\\\hline 102&Bertrand&Russell&102&x02\\\hline 103&Alan&Turing&103&q10\\\hline\end{tabular}&\begin{tabular}{| l || l | l |}\hline\multicolumn{3}{| c |}{Department}\\\hline {\bf Id}&{\bf Name}&{\bf Secr'y}\\\hline q10&Sales&101\\\hline x02&Production&102\\\hline\end{tabular}&&\begin{tabular}{| l |}\hline\multicolumn{1}{| c |}{ String}\\\hline {\bf Id}\\\hline a\\\hline b\\\hline \vdots\\\hline z\\\hline aa\\\hline ab\\\hline\vdots\\\hline\end{tabular}&\end{align*}

\tiny\begin{align*}&\mcC=&\delta\taking\mcC\too\Sets\\&\fbox{\parbox{1.4in}{\underline{$m;d=d$}\hsp  \underline{$s;d=\id_D$}\\\xymatrix{\bullet_E\ar@<.5ex>[rr]^d\ar@(l,u)[]^m\ar@/_.5pc/[d]_f\ar@/^.5pc/[d]^l&&\bullet_D\ar@<.5ex>[ll]^s\ar[dll]^n\\\bullet_S&}}}&\fbox{\parbox{1.8in}{\xymatrix{\fbox{101,102,103}\ar@<.5ex>[rr]^d\ar@(l,u)[]^m\ar@/_.5pc/[d]_f\ar@/^.5pc/[d]^l&&\fbox{q10,x02}\ar@<.5ex>[ll]^s\ar[dll]^n\\\fbox{a,b,\ldots,z,aa,ab,\ldots}}}}\end{align*}.\normalsize

}

\fst{Purpose of this talk}{
The purpose of this talk is: 
\begin{itemize}
	\item to refine the above ``databases are categories" notion a bit:\begin{itemize}\item a database schema is really ``a sketch." \\ In other words, a database schema is a presentation of a category.\item We can use a similar idea to discuss incomplete data.\end{itemize}
	\item to discuss querying and data migration.
	\item to discuss typing and calculated fields.
	\item to talk about ontologies, or ``ologs," and a vision of ``functorial communication" based on the above ideas.
\end{itemize}

}

\section{Sketching schemas and functors}

\fst{Subtlety in ``database schema=category" idea}{

Question: So, what's wrong with saying that the schema for the database on the left is the category on the right?

\tiny \begin{align*}\parbox{2in}{\begin{tabular}{| l || l | l | l | l |}\hline\multicolumn{5}{| c |}{Employee}\\\hline {\bf Id}&{\bf First}&{\bf Last}&{\bf Mgr}&{\bf Dpt}\\\hline 101&David&Hilbert&103&q10\\\hline 102&Bertrand&Russell&102&x02\\\hline 103&Alan&Turing&103&q10\\\hline\end{tabular}\\\begin{tabular}{| l || l | l |}\hline\multicolumn{3}{| c |}{Department}\\\hline {\bf Id}&{\bf Name}&{\bf Secr'y}\\\hline q10&Sales&101\\\hline x02&Production&102\\\hline\end{tabular}}&&
\fbox{\parbox{1.4in}{\underline{$m;d=d$}\hsp  \underline{$s;d=\id_D$}\\\xymatrix{\bullet_E\ar@<.5ex>[rr]^d\ar@(l,u)[]^m\ar@/_.5pc/[d]_f\ar@/^.5pc/[d]^l&&\bullet_D\ar@<.5ex>[ll]^s\ar[dll]^n\\\bullet_S&}}}\end{align*}\normalsize

Answer: \begin{itemize}\item Categories have a composition law defined for every pair of composable arrows. \item But in the picture on the right, we don't have an arrow for $s;f$. \item Similarly in the schema on the left, there is no ``secretary's first name" column in the department table. \item Conclusion: we are only imposing a composition law where we need to.\end{itemize}

}

\fst{Why venture into sketches?}{

\begin{itemize}
	\item We want to be able to write down a set of objects, arrows, and composition laws, without having to throw in an arrow for every possible composition.  How do we encode that?  
	\item We want to present categories by generators (objects and arrows) and relations (composition laws).
	\item We want to ``sketch a category."
\end{itemize}

}

\fst{Information schemas}{

Suppose we don't have any composition laws to speak of.  What information do we need in order to specify such a schema?\\

Information schema: \begin{align*}\mcG:=\fbox{\xymatrix@=15pt{\LTO{Table}&&\LTO{Column}\ar@/^1pc/[ll]^{\tn{foreign\_table}}\ar@/_1pc/[ll]_{\tn{in}}}}&=\fbox{\xymatrix@=15pt{\LTO{Object}&&\LTO{Arrow}\ar@/^1pc/[ll]^{\tn{target}}\ar@/_1pc/[ll]_{\tn{source}}}}\end{align*}

Here's an example state on this schema, a functor $\delta\taking\mcG\too\Set$: 

\tiny\begin{align*}\begin{tabular}{| l || l |}\hline\multicolumn{1}{| c |}{Object}\\\hline {\bf Id}\\\hline 1\\\hline 2\\\hline 3\\\hline\end{tabular}&\hsp\begin{tabular}{| l || l | l |}\hline\multicolumn{3}{| c |}{Arrow}\\\hline {\bf Id}&{\bf source}&{\bf target}\\\hline v&1&2\\\hline w&1&2\\\hline x&2&2\\\hline y&2&3\\\hline z&3&2\\\hline\end{tabular}&\fbox{\xymatrix{\LMO{1}\ar@<.5ex>[rr]^v\ar@<-.5ex>[rr]_w&&\LMO{2}\ar@(r,u)[]_x\ar@<.5ex>[dl]^y\\&\LMO{3}\ar@<.5ex>[ru]^z}}\end{align*}

}

\frame{\frametitle{Information schema for category sketches}
Schema for category sketches:  \begin{align*}\bD:=\fbox{\xymatrix{\LTO{Object}&&\LTO{Arrow}\ar@/^1pc/[ll]^{\tn{target}}\ar@/_1pc/[ll]_{\tn{source}}&&\LTO{Comm. triangle}\ar@/_1.5pc/[ll]_{\tn{first\_map}}\ar[ll]_{\tn{composite}}\ar@/^1.5pc/[ll]^{\tn{second\_map}}}}\end{align*}

Here we would need three additional equations coming from the triangle \small \begin{align*}\parbox{1in}{\xymatrix{&\LMO{2}\\\LMO{0}\ar[r]_{\tn{first\_map}}\ar[ur]^{\tn{composite}}&\LMO{1}\ar[u]_{\tn{second\_map}}}}&\hsp\hsp\parbox{2in}{\begin{align*}\tn{first\_map;source} &= \tn{composite;source}\\\tn{first\_map;target}&=\tn{second\_map;source}\\\tn{second\_map;target}&=\tn{composite;target}\end{align*}}\end{align*}\normalsize
}

\frame{\frametitle{Sketching our example, $\mcC$}

\begin{itemize}
	\item $\mcC$ is a schema for a department store.  
	\item But how do we record the schema $\mcC$ itself, in terms of the information schema $\bD$?
\end{itemize}

\vspace{-.2in}\tiny
\begin{align*}\tn{Record}\hsp\parbox{1.2in}{\begin{center}$\mcC=$\end{center}\vspace{-.1in}\fbox{\parbox{1.2in}{\underline{$m;d=d$}\hsp  \underline{$s;d=\id_D$}\\\xymatrix@=15pt{\bullet_E\ar@<.5ex>[rr]^d\ar@(l,u)[]^m\ar@/_.5pc/[d]_f\ar@/^.5pc/[d]^l&&\bullet_D\ar@<.5ex>[ll]^s\ar[dll]^n\\\bullet_S&}}}}&\hsp \tn{in terms of}\hsp\parbox{2.4in}{\begin{center}$\bD=$\end{center}\vspace{-.1in}\fbox{\xymatrix@=12pt{\LTO{Object}&&\LTO{Arrow}\ar@/^1pc/[ll]^{\tn{target}}\ar@/_1pc/[ll]_{\tn{source}}&&\LTO{Comm. triangle}\ar@/_1.5pc/[ll]_{\tn{first\_map}}\ar[ll]_{\tn{composite}}\ar@/^1.5pc/[ll]^{\tn{second\_map}}}}}\end{align*}

\begin{align*}
\begin{tabular}{| l ||}\hline\multicolumn{1}{| c |}{Object}\\\hline {\bf Id}\\\hline E\\\hline D\\\hline S\\\hline\end{tabular}&\hsp\begin{tabular}{| l || l | l |}\hline\multicolumn{3}{| c |}{Arrow}\\\hline {\bf Id}&{\bf source}&{\bf target}\\\hline m&E&E\\\hline d&E&D\\\hline s&D&E\\\hline f&E&S\\\hline l&E&S\\\hline n&D&S\\\hline i&D&D\\\hline\end{tabular}&\hsp\begin{tabular}{| l || l | l | l |}\hline\multicolumn{4}{| c |}{Comm. triangle}\\\hline{\bf Id}&{\bf first\_map}&{\bf second\_map}&{\bf composite}\\\hline 1&m&d&d\\\hline 2&s&d&i\\\hline 3&d&i&d\\\hline 4&i&s&s\\\hline 5&i&n&n\\\hline 6&i&i&i\\\hline\end{tabular}
\end{align*}

\normalsize

}

\fst{Conclusion}{

\small\begin{align*}\bD:=\fbox{\xymatrix@=18pt{\LTO{Object}&&\LTO{Arrow}\ar@/^1pc/[ll]^{\tn{target}}\ar@/_1pc/[ll]_{\tn{source}}&&\LTO{Comm. triangle}\ar@/_1.5pc/[ll]_{\tn{first\_map}}\ar[ll]_{\tn{composite}}\ar@/^1.5pc/[ll]^{\tn{second\_map}}}}\end{align*}\normalsize
\begin{itemize}

	\item We can write any schema $\mcC$ as a database state on $\bD$.
	\item We refine our definition from last time to say that a database schema is a state on $\bD$.
	\item I'll call these ``pre-categories."  They are basically equivalent to categories.
	\item A functor between pre-categories is just a morphism of states on $\bD$.
	\item I swept the difference under the rug last time because I wanted to emphasize the tight connection between database schemas and categories.  That connection is still tight.
	
\end{itemize}

}

\fst{But once we're here...}{

\begin{itemize}
	\item Now that we're working with presentations of categories rather than categories,
	\item it might be nice to ``present" other facts, besides composition laws.
	\item We can use more complex information schemas (beef up $\bD$) to specify that a certain table in our schema $\mcC$
	\begin{itemize}
		\item is a product of other tables,
		\item is a fiber product, 
		\item is a colimit, 
		\item is an exponential object, 
		\item is empty (i.e. has no rows),
		\item has only one row,
		\item etc.
	\end{itemize}
	\item We can do all this with sketches.
\end{itemize}

}

\fst{Sketches and sketch maps}{

\begin{itemize}
	\item ``Sketch" is category-theory terminology for ``category specification."
	\item In a sketch we can specify that a certain object must be the limit or the colimit of some diagram.
	\item This could be used, e.g., in a database where we want to have a table of ``airplane seats" which is the coproduct of the tables ``first class seats" and ``economy class seats".
	\item We specify that we want $S=F\amalg E$ in the sketch $\mcC$.
	\item Now, instead of states being functors $\mcC\to\Set$, states are ``sketch maps" $\delta\taking\mcC\to\Set$; i.e. functors that preserve all the specified facts.
	\item For example for $\delta\taking\mcC\to\Set$ to be a sketch map, we must have $$\delta(S)=\delta(F)\amalg\delta(E).$$
\end{itemize}

}

\fst{Language of sketches as formal UML}{

\begin{itemize}
	\item At this point, one recognizes that sketches are quite similar to UML (Unified Modeling Language) diagrams for database schemas.
	\item You just specify what you want.
	\item What's new here is the connection between database theory and category theory.
	\item Category theory brings formal reasoning to the picture that was already there. 
\end{itemize}

}

\fst{Sketching states with ``RDF triple stores"}{

\begin{itemize}
	\item Recall from the first talk that given a category $\mcC$ and a functor $\delta\taking\mcC\too\Set$ one can take the Grothendieck construction $Gr(\delta)$.
	\item Suppose given the following example, considered as $\delta\taking\mcC\to\Set$ \tiny \begin{align*}&\begin{tabular}{| l || l | l | l | l |}\hline\multicolumn{5}{| c |}{Employee}\\\hline {\bf Emp\_Id}&{\bf First}&{\bf Last}&{\bf Mgr}&{\bf Dpt}\\\hline 101&David&Hilbert&103&q10\\\hline 102&Bertrand&Russell&102&x02\\\hline 103&Alan&Turing&103&q10\\\hline\end{tabular}&\begin{tabular}{| l || l | l |}\hline\multicolumn{3}{| c |}{Department}\\\hline {\bf Dept\_Id}&{\bf Name}&{\bf Secr'y}\\\hline q10&Sales&101\\\hline x02&Production&102\\\hline\end{tabular}\end{align*}\normalsize

Applying the Grothendieck construction, we get a category $Gr(\delta)$:

\tiny\begin{align*}\parbox{2.6in}{\fbox{\xymatrix@=1.1pt{\LMO{101}\ar@/_1pc/[dddddd]_{f}\ar@/_2pc/[ddddddrr]_<<<<<<{l}\ar@/_1pc/[rr]_{m}\ar@/^1pc/[rrrrr]^{d}&\LMO{102}&\LMO{103}&&&\LTO{q10}&\LTO{x02}\ar@/^2pc/[lllll]_s\ar@/^1pc/[llldddddd]^n\\\\\\\\\ldots&\LTO{Alan}&\LTO{Alao}&\ldots\\\ldots&\LTO{Bertranc}&\LTO{Bertrand}&\ldots\\\LTO{David}&\ldots&\LTO{Hilbert}&\LTO{Production}\\\LTO{Russell}&\LTO{Sales}&\LTO{Turing}&\ldots}}}\end{align*}

\end{itemize}

%At the following close-brace "}" something significantly changes, in terms of the .aux file.
%Before the "}", putting either \frame{}} or \begin{itemize}\item DOES NOT cause a "strong failure" in the sense that the .aux files need to be trashed in order to re-compile.
%On the other hand, after the "}", putting either \frame{}} or \begin{itemize}\item DOES cause a "strong failure" in the sense that the .aux files need to be trashed in order to re-compile.
}
%At the preceding close-brace "}" something significantly changes, in terms of the .aux file.
%Before the "}", putting either \frame{}} or \begin{itemize}\item DOES NOT cause a "strong failure" in the sense that the .aux files need to be trashed in order to re-compile.
%On the other hand, after the "}", putting either \frame{}} or \begin{itemize}\item DOES cause a "strong failure" in the sense that the .aux files need to be trashed in order to re-compile.

\fst{Change of perspective}{

 Given $\delta\taking\mcC\too\Set$, the Grothendieck construction of $\delta$ gives a functor $$\pi\taking Gr(\delta)\too\mcC.$$ 
 
\vspace{-.3in}\tiny\begin{align*}\parbox{2.6in}{\begin{center}$Gr(\delta)$:=\end{center}\vspace{-.1in}\fbox{\xymatrix@=1.1pt{\LMO{101}\ar@/_1pc/[dddddd]_{f}\ar@/_2pc/[ddddddrr]_<<<<<<{l}\ar@/_1pc/[rr]_{m}\ar@/^1pc/[rrrrr]^{d}&\LMO{102}&\LMO{103}&&&\LTO{q10}&\LTO{x02}\ar@/^2pc/[lllll]_s\ar@/^1pc/[llldddddd]^n\\\\\\\\\ldots&\LTO{Alan}&\LTO{Alao}&\ldots\\\ldots&\LTO{Bertranc}&\LTO{Bertrand}&\ldots\\\LTO{David}&\ldots&\LTO{Hilbert}&\LTO{Production}\\\LTO{Russell}&\LTO{Sales}&\LTO{Turing}&\ldots}}}\hsp\To{\pi}\hsp\parbox{1.2in}{\begin{center}$\mcC:=$\end{center}\vspace{-.1in}\fbox{\parbox{1.2in}{\underline{$m;d=d$}\hsp  \underline{$s;d=\id_D$}\\\xymatrix@=15pt{\bullet_E\ar@<.5ex>[rr]^d\ar@(l,u)[]^m\ar@/_.5pc/[d]_f\ar@/^.5pc/[d]^l&&\bullet_D\ar@<.5ex>[ll]^s\ar[dll]^n\\\bullet_S&}}}}\end{align*}\normalsize

The fiber over (inverse image of) every object $X\in\mcC$ is a set of objects in $\pi^{-1}(X)\in Gr(\delta)$.  That set is $\delta(X)$.
	
}

\fst{Allowing for incomplete, non-atomic, or bad data}{

\begin{itemize}\small
	\item We can think of any functor $\pi\taking\mcD\to\mcC$ as a ``pre-state" on $\mcC.$
	\item Such a functor $\pi$ can encode incomplete, non-atomic, or bad data.
	\tiny\begin{align*}\parbox{2in}{$\mcD:=$\fbox{\xymatrix@=5pt{\LTO{101}\ar[rr]&&\LTO{Hello}\\\LTO{102}\ar[urr]\\\LTO{103}&&\LTO{Goodbye}\\\LTO{104}\ar[uuurr]\ar[urr]}}}\\\parbox{2in}{\hspace{.5in}$\pi\downarrow$}\\\parbox{2in}{$\mcC:=$\fbox{\xymatrix@=5pt{\LMO{A}\ar[rrr]^f&&&\LMO{B}}}}\end{align*}\small
	
	\item Row 103 has no data in the $f$ cell, and row 104 has too much.
	\item Bad data (data not conforming to declared composition laws) can also be encoded as a functor $\pi\taking\mcD\to\mcC$.
	\item Any pre-state on $\mcC$ can be ``corrected" in a canonical way to a state.
	\item Conclusion: we can use category theory as a model even when things are awry.
\end{itemize}
\normalsize
}


\section{Functorial data migration and querying}

\fst{Data migration}{
\begin{itemize}
	\item Let's go back to the simple picture of database schemas as categories or pre-categories $\mcC$ and states as functors $\delta\taking\mcC\to\Set$.  (Not sketchy).
	\item Given a schema $\mcC$, the category of states on $\mcC$ is denoted $\mcC\set$.
	\item Given a morphism between schemas, we want to be able to move data back and forth in canonical ways.
	\item That means, given $F\taking\mcC\to\mcD$ we want functors $$\mcC\set\to\mcD\set$$ and $$\mcD\set\to\mcC\set.$$
\end{itemize}

}

\fst{The ``easy" migration functor, $\Delta$}{
\begin{itemize}
	\item Given a morphism of schemas (i.e. a functor) $$F\taking\mcC\too\mcD,$$ we can transform states on $\mcD$ to states on $\mcC$ as follows: $$(\delta\taking\mcD\too\Set) \mapsto ((\delta\circ F)\taking\mcC\too\Set)$$
	\item Thus we have a functor $\Delta_F\taking\mcD\set\too\mcC\set$.
	\item $\Delta_F$ basically operates by ``re-indexing."  Using it, one can duplicate or drop tables or columns.

\end{itemize}

}

\fst{The ``harder" migration functors, $\Sigma$ and $\Pi$}{
Given a morphism of schemas (i.e. a functor) $F\taking\mcC\too\mcD$, 
\begin{itemize}
	\item the functor $\Delta_F\taking\mcD\set\to\mcC\set$ has two adjoints:
	\begin{itemize}
		\item a left adjoint $\Sigma_F\taking\mcC\set\to\mcD\set$, and
		\item a right adjoint $\Pi_F\taking\mcC\set\to\mcD\set$,$$\xymatrix{\mcC\set\ar@<.5ex>[r]^{\Sigma_F}&\mcD\set\ar@<.5ex>[l]^{\Delta_F}\ar@<.5ex>[r]^{\Delta_F}&\mcC\set\ar@<.5ex>[l]^{\Pi_F}}$$		
	\end{itemize}
	\item Thus, given a morphism $F$ of schemas, these three functors, $$\Delta_F,\Sigma_F, \tn{ and } \Pi_F$$ allow one to move data back and forth between $\mcC$ and $\mcD$ in canonical ways.
\end{itemize}

}

\fst{Views as polynomial functors}{

These functors can be arbitrarily composed to create views.\vspace{-.7in}\tiny$$\parbox{1.5in}{$\mcB$:=\\\fbox{\xymatrix@=3pt{&&\LTO{boy}\ar[ddll]\ar[ddrr]\\&&\LTO{man}\ar[dll]\ar[drr]\\\LTO{name}&&&&\LTO{address}\ar[dd]\\&&\LTO{woman}\ar[ull]\ar[urr]\\&&\LTO{girl}\ar[uurr]\ar[uull]&&\LTO{street}\ar[dd]\\\\&&&&\LTO{city}}}}\From{F}
\parbox{1.4in}{$\mcC$:=\\\fbox{\xymatrix@=3pt{&&\LTO{boy}\ar[ddll]\ar[ddrr]\\&&\LTO{man}\ar[dll]\ar[drr]\\\LTO{name}&&&&\LTO{street}\\&&\LTO{woman}\ar[ull]\ar[urr]\\&&\LTO{girl}\ar[uurr]\ar[uull]}}}\To{G}
\parbox{1.3in}{\vspace{1in}$\mcD$:=\\\fbox{\xymatrix@=1pt{&&\LTO{male}\ar[ddll]\ar[ddrr]\\\\\LTO{name}&&&&\LTO{street}\\\\&&\LTO{female}\ar[uurr]\ar[uull]}}\begin{center}$\downarrow H$\end{center}\hspace{-.15in}
$\mcE$:=\fbox{\xymatrix@=1pt{&&\LTO{male}\ar[ddll]\ar[ddrr]\\\\\LTO{name}&&\LMO{Q}\ar[uu]\ar[dd]\ar@{}[rr]|(.4)*+{\checkmark}&&\LTO{street}\\\\&&\LTO{female}\ar[uurr]\ar[uull]}}}$$\normalsize

Given a state $\gamma\taking\mcB\to\Set$, what is $\Pi_H\circ\Sigma_G\circ\Delta_F(\gamma)\taking\mcE\to\Set$?
}

\frame{\frametitle{A simple ``SELECT" query}

SELECT title, isbn\\FROM book\\WHERE price $>100$\\

\tiny$$\parbox{1.7in}{$\mcC:=$\\\fbox{\xymatrix@=5pt{&&\LTO{book}\ar[rr]^{\tn{title}}\ar[rrdd]^{\tn{isbn}}\ar[dd]^{\tn{price}}&&\LTO{String}\\\\\LMO{\RR_{>100}}\ar[rr]&&\LMO{\RR}&&\LTO{isbn-num}}}}\To{F}\parbox{1.7in}{$\mcD:=$\\\fbox{\xymatrix@=5pt{\LMO{W}\ar[rr]\ar[dd]\ar@{}[ddrr]|{\checkmark}&&\LTO{book}\ar[rr]^{\tn{title}}\ar[rrdd]^{\tn{isbn}}\ar[dd]^{\tn{price}}&&\LTO{String}\\\\\LMO{\RR_{>100}}\ar[rr]&&\LMO{\RR}&&\LTO{isbn-num}}}}\From{G}\parbox{1in}{$\mcE:=$\\\fbox{\xymatrix@=5pt{\LMO{W}\ar[rr]^{\tn{title}}\ar[rrdd]^{\tn{isbn}}&&\LTO{String}\\\\&&\LTO{isbn-num}}}}$$\normalsize

$\Delta_G\circ \Pi_F$ is the appropriate view.\\For any $\delta\taking\mcC\to\Set$, we materialize the view as $\Delta_G\circ \Pi_F(\delta).$

}

\fst{Interfacing between schemas}{

\begin{itemize}
	\item We are often interested in taking data from one enterprise model $\mcC$ and transferring it to another enterprise model $\mcD$.
	\item Such transfers can also be accomplished using polynomial functors.
	\item However, if $\mcC$ and $\mcD$ are not basic schemas (they're too ``sketchy") then the ``harder" migration functors, $\Sigma$ and $\Pi$, might not exist. 
	\item Also, we might need to perform calculations such as concatenation, addition, comparison, conversion of units, etc. in order to interface these schemas.
	\item For this we'll need an underlying typing category. 
\end{itemize}
}

\section{Typing databases}

\fst{Incorporating data types and functions}{

\begin{itemize}
	\item In the example: \small$$\parbox{1.7in}{$\mcC:=$\\\fbox{\xymatrix@=7pt{&&\LTO{book}\ar[rr]^{\tn{title}}\ar[rrdd]^{\tn{isbn}}\ar[dd]^{\tn{price}}&&\LTO{String}\\\\\LMO{\RR_{>100}}\ar[rr]&&\LMO{\RR}&&\LTO{isbn-num}}}}$$\normalsize
how do we know that $\LTO{String}, \LMO{\RR},$ and $\LMO{\RR_{>100}}$ are what they say they are?  
	\item That is, given $\delta\taking\mcC\to\Set$, how do we specify that $\delta(\LMO{\RR})\in\Set$ is some pre-defined data type like Float. 
\end{itemize}
}

\fst{Power of category theory: connection is easy}{

\begin{itemize}
	\item Let $\Hask$ denote a category of types and functions that has been implemented on a computer and for which (at least theoretically) there exists a functor $V\taking\Hask\to\Set$.
	\begin{itemize}
		\item Think of $\Hask$ as all Haskell data types and the definable functions between them, as well as all new types that could possibly be output by modules.
	\end{itemize}  
	\item Now $\Hask$ begins to look like a schema and $V$ a ``canonical state."
	\item Since database schemas are categories and $\Hask$ is a category, we can integrate the two.
\end{itemize}

}

\frame{\frametitle{Example}
\begin{itemize}
	\item Here a nice functor $$\parbox{1.6in}{$\mcB:=$\\\fbox{\xymatrix@=3pt{&&&\LTO{String}\\\\\LMO{\RR_{>100}}\ar[rr]&&\LMO{\RR}}}}\To{F}\Hask$$ $F(\LTO{String})=\tn{char(40)}, \;\;F(\LMO{\RR})=\tn{Float},\;\; F(\LMO{\RR_{>100}})=$ some new type.
	\item There is also an obvious functor \small$$\mcB=\parbox{1.4in}{\fbox{\xymatrix@=3pt{&&&\LTO{String}\\\\\LMO{\RR_{>100}}\ar[rr]&&\LMO{\RR}}}}\To{G}\parbox{2in}{\fbox{\xymatrix@=7pt{&&\LTO{book}\ar[rr]^{\tn{title}}\ar[rrdd]^{\tn{isbn}}\ar[dd]^{\tn{price}}&&\LTO{String}\\\\\LMO{\RR_{>100}}\ar[rr]&&\LMO{\RR}&&\LTO{isbn-num}}}}=\mcC$$\normalsize
	\item We are interested in functors $\delta\taking\mcC\to\Set$ equipped with a map $\Delta_G\delta\to \Delta_FV$.
\end{itemize}

}

\frame{\frametitle{Typing in general}

\begin{itemize}
	\item If we need to enforce data types, our schema $\mcC$ will more than just a category (or sketch),
	\item It will be a category (or sketch) plus something like above: $$\Hask\From{F}\mcB\To{G}\mcC.$$
	\item And we won't interested in any old state $\delta\taking\mcC\to\Set$ but only those with a map $\Delta_G\delta\to \Delta_FV$, where $V\taking\Hask\to\Set$ is as above.
	\item By definition of adjunction, that's just $$\delta\to \Pi_G\Delta_F(V),$$ and $\Pi_G\Delta_F(V)$ is some huge fixed state on $\mcC$ that encodes our typing requirements.
	\item So the category of typed states is $\mcC\set_{/\Pi_G\Delta_F(V)}$.  This is a topos.
\end{itemize}

}

\fst{Summary}{

\begin{itemize}
	\item The longer portion of this talk is over.
	\item I discussed how to tighten the connection between databases and mathematics.
	\begin{itemize}
		\item How to only include the columns and composition laws you want (not necessarily all of them).
		\item How to force tables to be products or coproducts (etc.) of other tables.
		\item How to model incomplete, non-atomic, or bad data.
		\item How functors between schemas give rise to views and data migration.
		\item How to encode typing information and calculated fields to database schemas.
	\end{itemize}
\end{itemize}

}

\section{Ologs}

\fst{Bringing these concepts to researchers}{
\begin{itemize}
	\item In this short second section, I want to look at these categories and sketches from a different perspective.
	\item Issue: how to allow people (e.g. scientists) to record very precise conceptual ideas.
	\begin{itemize}
		\item A big problem in science is to know where data comes from and how its been manipulated to arrive at conclusions.
		\item Another is to have the flexibility to design new databases on the fly, so as to capture unexpected nuances of data.
		\item Finally, academic researchers need to be able to record precisely what they mean in a computer-searchable way, rather than in silos of prose. 
	\end{itemize}
	\item We need to democratize information storage.
\end{itemize}

}

\fst{Ologs: linguistic categories}{

\begin{itemize}
	\item The word ``Olog" 
	\begin{itemize}
		\item From ``ontology log," a play on ``Blog."
		\item Alternatively, a suffix for ``study of."
	\end{itemize}
	\item An olog is a category or sketch, but with natural language labels.
	\item The goal is to capture meaning using explicit structure.
	
\end{itemize}

}

\frame{\frametitle{Example olog}
\tiny\begin{align*}\label{dia:arganine}\fbox{\xymatrix{&\smbox{1}{arganine}\ar@{}[dr]|{\checkmark}\LA{r}{}\LA{d}{is}&\obox{2}{.9in}{an electrically-charged side chain}\LA{d}{is}\\&\smbox{3}{an amino acid}\LAL{dl}{has}\LA{dr}{has}\LA{r}{has}&\smbox{4}{a side chain}\\\smbox{5}{an amine group}&&\smbox{6}{a carboxylic acid}}}\end{align*}  \normalsize

``Arganine is an amino acid.  An amino acid has a side chain, a carboxylic acid, and an amine group.  An electrically-charged side chain is a side chain.  Arganine's side chain is electrically-charged."
}

\frame{\frametitle{Another example}
\tiny$$\fbox{\parbox{4in}{\hspace{1.3in}\underline{(1,2,4,5) and (2,3,5,6) are fiber products.}\\\\\xymatrix@=10pt{\smbox{1}{a material $x$ at room temperature}\LA{d}{is}\LA{r}{}&\smbox{2}{room temperature}\LA{rr}{in kelvin}\LA{d}{is}&&\smbox{3}{$\{x\in\RR : 293\leq x\leq 298\}$}\LA{d}{}\\\smbox{4}{a material $x$ at a temperature $y$}\LA{d}{x}\LA{r}{y}&\smbox{5}{a temperature}\LA{rr}{in kelvin}&&\obox{6}{.1in}{$\RR$}\\\smbox{7}{a material}}}}$$\normalsize

\begin{itemize}
	\item This is clearly a sketch defining room temperature and what it means to be a material at room temperature.
	\item An underlying typing system could handle the arrow $3\to 6$.
	\item This olog could be referenced and extended by many scientists.
	\item It represents a fragment of a world-view.
	\item Moreover, each object and arrow could be ``clickable" meaning one could open it up as a table of values.
\end{itemize}

}

\frame{\frametitle{Another example}
\vspace{-.2in}
\begin{align*} \scriptsize\fbox{\xymatrix@=14pt{\mebox{1}{a company that has hosted me}\ar[drr]&&\smbox{2}{Galois Inc.}\ar[ll]\ar[rr]\ar[d]\ar@{}[dll]|(.3)*{\checkmark}\ar@{}[drr]|(.3)*{\checkmark}&&\mebox{3}{a company in Portland Oregon}\ar[d]\ar[dll]\\&&\smbox{4}{a company}&&\smbox{5}{Portland Oregon}\\\smbox{6}{(Galois, John Launchbury)}\ar[rr]\ar[ddrr]\ar[uurr]\ar@{}[urr]|{\checkmark}\ar@{}[drr]|{\checkmark}&&\parbox{.8in}{$\mebox{7}{a pair $(x,y)$ where $x$ is a company and $y$ is its founder}$}\ar[u]^-x\ar[d]_-y\\&&\smbox{8}{a founder}\ar[rr]&&\smbox{9}{a person}\\&&\smbox{10}{John Launchbury}\ar[u]\ar[rr]\ar@{}[urr]|{\checkmark}&&\smbox{11}{a person I play go with}\ar[u]}}
\end{align*}
\normalsize
}

\fst{Dreams and speculation}{

\begin{itemize}
	\item Imagine a world of people authoring ologs and connecting to each other's ologs, forming a network of knowledge.
	\item These connections could given by functors (or ``spans"), which could only be instantiated if the two ologs are compatible.
	\item My dream is an ``olog network" of people recording their world-views.
	\begin{itemize}
		\item This is analogous to the semantic web vision (in fact it is possible to convert an olog into a RDF or OWL schema).
		\item Politicians could record their views in ologs.  It would be interesting to note precise differences between Obama's olog and Palin's olog.
		\item Information in the olog network is neither right nor wrong, just ``linked into" by others or not.
	\end{itemize}
	\item Laws should be query-able.  Could laws and legislation be recorded in this precise format?  
\end{itemize}
}

\section{Conclusion}

\fst{Conclusion}{

\begin{itemize}
	\item The purpose of this talk was to refine and extend the ``databases are categories" idea from last time.
	\item With schemas as categories, views, data migration, and integrating with PL become natural.
	\item Recording world-views as ologs may yield an information network that is instantly compatible with database systems.
	\item ``It's all categories and functors!" ---\\ I hope people see category theory as a unified modeling language for information storage, processing, and transfer.
	
\end{itemize}
	
\begin{center}{\bf Thanks for hosting me and inviting me to speak!}\end{center}

}

\end{document}

