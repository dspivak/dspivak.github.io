\documentclass{amsart}

\oddsidemargin0in
\textwidth5.8in

\def\to{\rightarrow}
\def\To{\xrightarrow}
\def\taking{\colon}
\def\from{\leftarrow}
\def\From{\xleftarrow}
\def\A{\textnormal{A}}
\def\B{\textnormal{B}}
\def\C{\textnormal{C}}
\def\cross{\times}


\begin{document}

\title{A set of rules for labeling ologs}

\maketitle

\begin{abstract}

The following is a set of rules for labeling ologs, written on December 6, 2009.  One sees immediately that it is unrefined and ad hoc.  One goal for the Informatics seminar is to add to this set of rules, to refine it, and to make it natural (even functorial if possible). 

Feel free to challenge a given entry, to propose a correction to a given entry, or to validate a given entry in this set of rules.   How should boxes be labeled?  How should arrows be labeled?  What if they are fiber products, graphs, images, power objects, etc.?  

\end{abstract}

\begin{enumerate}

\item If $A$ is a box for which we do not specify a label, the label will be taken to be ``an A".

\item An unlabeled arrow is by default labeled ``has an associated".

\item Any identity arrow is labeled ``is of course".

\item Never use a plural noun to label a box (i.e. say ``a person" rather than ``people").

\item In a box $A$, the name for each structural outgoing arrow should be a variable (i.e. part of the structure, but not part of the data of $A$).  

\item Suppose an arrow $f\taking A\to B$ is labeled $A\To{blahs}B$.  Then the label of the graph box for $f$ is ``an A, $x$,  which blahs a B, $y$" (where $x$ and $y$ are variables denoting the arrows out of the graph box.  $x$ points at A and $y$ points at B.  See examples below.)

\begin{enumerate}

\item Example: Consider the label ``is" for a certain arrow $A\to B$.   Then label of the graph box is ``an A, $x$, which is a B, $y$."  

\item Example: Consider the label ``has as father" for a certain arrow $f\taking$Person$\to$Person.  Then label of the graph box is ``a person, $x$, which has as father a person, $y$."

\end{enumerate}

\item The graph box for $f\taking A\to B$ has two structural arrows out, namely $x$ and $y$.  Suppose that $f$ is labeled ``blahs".  Then $x$ is labeled ``is in particular" and $y$ is labeled ``blahs".  Note that $x$ is an isomorphism; its inverse is labeled ``can be construed as".

\item The composition of two arrows is labeled by concatenating the label of the first arrow, the label of its target box, the word ``which", and then the label of the second arrow.

\item An arrow may be labeled ``is an instance of," but note that such arrows do not compose.  However, ``is an instance of" does compose with ``is" to give ``is an instance of." 

\item Consider an arrow $A\To{blahs}B$.  The image box is labeled ``a B, $x$, for which there exists some A that blahs $x$."  If $I$ is the image, then the map $A\to I$ is labeled ``blahs" and the map $I\to B$ is labeled ``is".  

\item If $x\taking A\to B$ is a structural arrow, then its image box has a more intricate naming system, coming from the structure of $A$; I need to think about this more.

\item Boxes labeled by a definite noun *blah* (e.g. ``the book" or ``Kathleen") should be relabeled ``an idea relating to *blah*".

\item The pullback of an arrow is given the same name.

\item Let $A, B$, and $C$ be boxes and let $x\taking B\to A$ and $y\taking C\to A$ be arrows (forming a ``potential fiber product" diagram).  Suppose the label for $A$ is ``an apple," the label for $B$ is ``an apple $x$, and more, all satisfying B-ness" and the label for $C$ is ``an apple $y$, and more, all satisfying C-ness".  Then the label for the fiber product is ``an apple $x$, and more, satisfying B-ness, together with an apple $y$, and more, satisfying C-ness, for which $x$ and $y$ agree."  

\item Suppose that $A\To{f}B\From{is} C$ is a potential fiber product diagram in which at least one arrow is labeled ``is".  Then the fiber product object $A\cross_BC$ is labeled ``An $A$, $x$, such that $x$ $f$ $C$."

\item Suppose that $A\To{f}B\From{is} C$ is as above and that $f$ is a structural arrow of $A$.  Then we simply take the label of $A$ and replace the $f$-part with the label from $C$.  For example, suppose $A$ is labeled ``a man $x$ and a facial expression $y$ such that $x$ can make $y$", that $B$ is labeled ``a facial expression", and that $C$ is labeled ``a kind face".  Then the fiber product is labeled ``a man $x$ and a kind face $y$ such that $x$ can make $y$".



\end{enumerate}




\end{document}

