\documentclass{amsart}

\usepackage{amssymb, amscd,stmaryrd,setspace,hyperref,color}

\input xy
\xyoption{all} \xyoption{poly} \xyoption{knot}\xyoption{curve}

\newcommand{\comment}[1]{}

\newcommand{\longnote}[2][4.9in]{\fcolorbox{black}{yellow}{\parbox{#1}{\color{black} #2}}}
\newcommand{\note}[1]{\fcolorbox{black}{yellow}{\color{black} #1}}
\newcommand{\q}[1]{\begin{question}#1\end{question}}
\newcommand{\g}[1]{\begin{guess}#1\end{guess}}

\def\tn{\textnormal}
\def\mf{\mathfrak}
\def\mc{\mathcal}

\def\ZZ{{\mathbb Z}}
\def\QQ{{\mathbb Q}}
\def\RR{{\mathbb R}}
\def\CC{{\mathbb C}}
\def\AA{{\mathbb A}}
\def\PP{{\mathbb P}}
\def\NN{{\mathbb N}}

\def\bD{{\bf \Delta}}
\def\Str{{\bf Str}}

\def\Hom{\tn{Hom}}
\def\Fun{\tn{Fun}}
\def\Nat{\tn{Nat}}
\def\Ob{\tn{Ob}}
\def\Op{\tn{Op}}

\def\to{\rightarrow}
\def\from{\leftarrow}
\def\cross{\times}
\def\taking{\colon}
\def\inj{\hookrightarrow}
\def\surj{\twoheadrightarrow}
\def\too{\longrightarrow}
\def\fromm{\longleftarrow}
\def\down{\downarrow}
\def\Down{\Downarrow}
\def\Up{\Uparrow}
\def\tooo{\longlongrightarrow}
\def\tto{\rightrightarrows}
\def\ttto{\equiv\!\!>}
\def\ss{\subset}
\def\superset{\supset}
\def\iso{\cong}
\def\|{{\;|\;}}
\def\m1{{-1}}
\def\op{^\tn{op}}
\def\loc{\tn{loc}}
\def\la{\langle}
\def\ra{\rangle}
\def\wt{\widetilde}
\def\wh{\widehat}
\def\we{\simeq}
\def\ol{\overline}
\def\ul{\underline}
\def\qeq{\mathop{=}^?}
\def\setto{\colon\hspace{-.25cm}=}

\def\ullimit{\ar@{}[rd]|(.3)*+{\lrcorner}}
\def\urlimit{\ar@{}[ld]|(.3)*+{\llcorner}}
\def\lllimit{\ar@{}[ru]|(.3)*+{\urcorner}}
\def\lrlimit{\ar@{}[lu]|(.3)*+{\ulcorner}}
\def\ulhlimit{\ar@{}[rd]|(.3)*+{\diamond}}
\def\urhlimit{\ar@{}[ld]|(.3)*+{\diamond}}
\def\llhlimit{\ar@{}[ru]|(.3)*+{\diamond}}
\def\lrhlimit{\ar@{}[lu]|(.3)*+{\diamond}}
\newcommand{\clabel}[1]{\ar@{}[rd]|(.5)*+{#1}}

\newcommand{\arr}[1]{\ar@<.5ex>[#1]\ar@<-.5ex>[#1]}
\newcommand{\arrr}[1]{\ar@<.7ex>[#1]\ar@<0ex>[#1]\ar@<-.7ex>[#1]}
\newcommand{\arrrr}[1]{\ar@<.9ex>[#1]\ar@<.3ex>[#1]\ar@<-.3ex>[#1]\ar@<-.9ex>[#1]}
\newcommand{\arrrrr}[1]{\ar@<1ex>[#1]\ar@<.5ex>[#1]\ar[#1]\ar@<-.5ex>[#1]\ar@<-1ex>[#1]}

\newcommand{\To}[1]{\xrightarrow{#1}}
\newcommand{\Too}[1]{\xrightarrow{\ \ #1\ \ }}
\newcommand{\From}[1]{\xleftarrow{#1}}


\newcommand{\Adjoint}[4]{\xymatrix@1{#2 \ar@<.5ex>[r]^-{#1} & #3 \ar@<.5ex>[l]^-{#4}}}
\newcommand{\adjoint}[4]{\xymatrix@1{#1\colon #2\ar@<.5ex>[r]& #3\;:#4 \ar@<.5ex>[l]}}

\def\id{\tn{id}}
\def\Top{{\bf Top}}
\def\Cat{{\bf Cat}}
\def\Sets{{\bf Sets}}
\def\sSets{{\bf sSets}}
\def\Grpd{{\bf Grpd}}
\def\Pre{{\bf Pre}}
\def\Shv{{\bf Shv}}
\def\Rings{{\bf Rings}}

\def\colim{\mathop{\tn{colim}}}

\def\mcA{\mc{A}}
\def\mcB{\mc{B}}
\def\mcC{\mc{C}}
\def\mcD{\mc{D}}
\def\mcE{\mc{E}}
\def\mcF{\mc{F}}
\def\mcG{\mc{G}}
\def\mcH{\mc{H}}
\def\mcI{\mc{I}}
\def\mcJ{\mc{J}}
\def\mcK{\mc{K}}
\def\mcL{\mc{L}}
\def\mcM{\mc{M}}
\def\mcN{\mc{N}}
\def\mcO{\mc{O}}
\def\mcP{\mc{P}}
\def\mcQ{\mc{Q}}
\def\mcR{\mc{R}}
\def\mcS{\mc{S}}
\def\mcT{\mc{T}}
\def\mcU{\mc{U}}
\def\mcV{\mc{V}}
\def\mcW{\mc{W}}
\def\mcX{\mc{X}}
\def\mcY{\mc{Y}}
\def\mcZ{\mc{Z}}

\newtheorem{theorem}{Theorem}[subsection]
\newtheorem{lemma}[theorem]{Lemma}
\newtheorem{proposition}[theorem]{Proposition}
\newtheorem{corollary}[theorem]{Corollary}
\newtheorem{fact}[theorem]{Fact}

\theoremstyle{remark}
\newtheorem{remark}[theorem]{Remark}
\newtheorem{example}[theorem]{Example}
\newtheorem{warning}[theorem]{Warning}
\newtheorem{question}[theorem]{Question}
\newtheorem{guess}[theorem]{Guess}
\newtheorem{answer}[theorem]{Answer}
\newtheorem{construction}[theorem]{Construction}

\theoremstyle{definition}
\newtheorem{definition}[theorem]{Definition}
\newtheorem{notation}[theorem]{Notation}
\newtheorem{conjecture}[theorem]{Conjecture}
\newtheorem{postulate}[theorem]{Postulate}

\def\DT{{\bf DT}}
\def\GD{{\bf GD}}
\def\DB{\GD}
\def\Sch{{\bf Sch}}
\def\Null{{\bf Null}}
\def\Strings{{\bf Strings}}
\def\ND{{\bf ND}}
\def\Tables{{\bf Tables}}
\def\'{\tn{'}}
\def\disunion{\amalg}
\def\Rel{{\bf Rel}}
\def\mcRel{{\bf \mcR el}}
\def\Cech{$\check{\tn{C}}$ech }
\def\C{\check{\tn{C}}}
\def\Fin{{\bf Fin}}
\def\singleton{{\{*\}}}
\def\Sub{{\bf Sub}}
\def\card{\tn{card}}
\def\Data{{\bf DB}}
\def\DB{{\bf DB}}
\def\im{\tn{im}}
\def\'{\tn{'}}
\def\start{\note{start here}}





\def\TT{{\mathbb T}}
\def\UU{{\mathbb U}}
\def\Sch{{\bf Sch}}

\begin{document}

\title{A category-theoretic model of information and communication}

\author{David I. Spivak}

\begin{abstract}

The following outline is rough.  The main idea is this.  A database is a span connecting a schema on a theory and an algebra on that theory; the span is ``obedient" to the schema.  Spans are everywhere in this world.  In particular, a morphism of databases is a span between spans: $$\xymatrix{&U\ar[dl]_a\ar[dr]^{a'}\\T\ar[dr]^\Gamma\ar@/_1pc/[ddr]^>>>>>>\mcK\ar@/_1pc/[dddr]_\sigma\ar@{}[rr]|{\stackrel{\alpha_\Gamma,\alpha_\mcK,\alpha_\sigma}{\Longrightarrow}}&&T'\ar[dl]_{\Gamma'}\ar@/^1pc/[ddl]_>>>>>>{\mcK'}\ar@/^1pc/[dddl]^{\sigma'}\\\ar@{}[r]|{\stackrel{\beta}{\Rightarrow}}&\Set\ar@{=}[d]&\ar@{}[l]|{\stackrel{\beta'}{\Leftarrow}}\\\ar@{}[r]|{\stackrel{\gamma}{\Leftarrow}}&\Set\ar@{=}[d]&\ar@{}[l]|{\stackrel{\gamma'}{\Rightarrow}}\\&\Set}$$  Here, $\alpha_\Gamma\taking\Gamma\circ a\to\Gamma'\circ a', \alpha_\mcK\taking\mcK\circ a\to\mcK'\circ a'$, and $\alpha_\sigma\taking\sigma\circ a\to \sigma'\circ a'$ are natural transformations, as are $\gamma\taking\mcK\to\sigma$ and $\gamma'\taking\mcK'\to\sigma'$, and as are $\beta\taking\mcK\to\Gamma$ and $\beta'\taking\mcK'\to\Gamma'$, the last two of which are assumed obedient.  Finally, there is a commutativity of diagrams: $\alpha_\Gamma\circ\beta=\beta'\circ\alpha_\mcK$ and $\alpha_\sigma\circ\gamma=\gamma'\circ\alpha_\mcK$.

\end{abstract}

\maketitle

\section{Introduction}

\section{A brief introduction to algebraic theories}

\subsection{Category theory}

\begin{itemize}

\item Category
\item Functor
\item Natural transformation
\item Categories of functors
\item Products
\item Relations (in a category)

\end{itemize}

\subsection{Single-sorted algebraic theories}

\subsection{Multi-sorted algebraic theories}

\subsection{Example: discrete theories}

\subsection{Example: the theory generated by a category}

\subsection{Topos-theoretic language for creating types}

\section{Schemas and their basis categories}

\subsection{Schemas}

\begin{definition}

Let $\TT$ be an algebraic theory.  A {\em schema on $\TT$} is a functor $\sigma\taking\TT\to\Set$.  A {\em morphism of schemas on $\TT$} is a natural transformation of functors.  We denote the category of schemas on $\TT$ by $\Sch_\TT:=\Set^\TT$.

A {\em schema} is a pair $(\TT,\sigma)$, where $\TT$ is an algebraic theory and $\sigma$ is a schema on $\TT$.  A {\em simplicial schema} is a schema $(\TT,\sigma)$ in which $\TT$ is a discrete algebraic theory.

\end{definition}

Above we define a morphism of schemas over a fixed theory, but we do not define a morphism of schemas in general.  That is because there are many options for how that can go.

\begin{definition}

Suppose that $\alpha\taking\TT\to\TT'$ is a morphism of algebraic theories.  This induces three functors, as defined in Definition \ref{***}: \begin{itemize}\item $\alpha^*\taking\Sch(\TT')\to\Sch(\TT)$,\item $\alpha_*\taking\Sch(\TT)\to\Sch(\TT')$, and \item $\alpha_!\taking\Sch(\TT)\to\Sch(\TT').$ \end{itemize}

\end{definition}

\begin{definition}

Let $\TT$ denote an algebraic theory and let $\TT^\flat$ denote the underlying discrete theory.  The inclusion of categories $\TT^\flat\to\TT$ induces a functor $\Sch_\TT\to\Sch_{\TT^\flat}$, and the image of a schema $\sigma$ on $\TT$, which is a simplicial schema, is called its {\em the simplicial schema underlying $\sigma$.}

\end{definition}

Define:

\begin{definition}

A {\em type designation} is a pair $(\TT,\Gamma)$, where $\TT$ is an algebraic theory and $\Gamma\taking\TT\to\Set$ is a $\TT$-algebra.  A {\em morphism of type designations} $(\TT,\Gamma)\to(\TT',\Gamma')$ consists of a diagram $$\xymatrix@=12pt{&\UU\ar[ld]\ar[rd]\ar@{}[dd]|{\stackrel{\alpha}{\Rightarrow}}\\\TT\ar[rd]_\Gamma&&\TT'\ar[ld]^{\Gamma'}\\&\Set}$$  In other words, it consists of a sequence $(a,a',\alpha)$, where $a\taking\UU\to\TT$ and $b\taking\UU\to\TT'$ are functors and $\alpha\taking\Gamma\circ a\to\Gamma'\circ a'$ is a natural transformation.

\end{definition}

\begin{example}

Suppose that $\TT$ is the discrete theory generated by objects called $\ZZ$ and $\ZZ-0$ and that $\TT'$ is the discrete theory generated by an object called $\QQ$.  Let $\Gamma$ and $\Gamma'$ be the maps indicated by the naming of objects.  Let $\UU$ denote a single sorted discrete algebraic theory.  Map it to $\TT$ by sending the generator to $\ZZ\cross(\ZZ-0)$ and map it to $\TT'$ by sending the generator to the unique generator.  The natural transformation $\alpha$ is isomorphic to the quotient by an ``obvious" equivalence relation. 

\end{example}

\subsection{Bases}

\begin{itemize}

\item Face maps
\item Generalization of non-degeneracy
\item Category of non-degeneracies

\end{itemize}

\subsection{Sheaves}

\section{Algebraic databases}

\subsection{Fixed types and schema}

\subsection{Change of schema}

\subsection{Change of types}

\subsection{General situation: allegories everywhere (?)}

\subsection{Ontologies}

The issue of incomplete data.

\section{Static communication networks}

\subsection{Networks}

\subsection{Communication protocol}

\subsection{The process of communication}

\subsection{The canonical problem}

The canonical problem is for each vertex to communicate the shape of the network, the local theories, the local schemas, and the local datas, from his perspective.  

\subsection{What's wrong with this model}

In the real world, what can we say about these invisible databases over the higher simplices?  What are they?  How are these created by the participants?

\section{Dynamic communication networks}

(Involving time.   Fixed network shape.)

Data provenance.
Cloud databases.

\section{Open questions}

\begin{itemize}

\item Variable network shape?
\item Fuzziness of new data
\item Self-similarity, fractal nature.
\item Encoding context.
\item Role of history?
\item Prediction?
\item What is learning?
\item Can mathematics be fully encoded here?

\end{itemize}

\end{document}