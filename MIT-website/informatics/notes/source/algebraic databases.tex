\documentclass{amsart}

\usepackage{amssymb, amscd,stmaryrd,setspace,hyperref,color}

\input xy
\xyoption{all} \xyoption{poly} \xyoption{knot}\xyoption{curve}

\newcommand{\comment}[1]{}

\newcommand{\longnote}[2][4.9in]{\fcolorbox{black}{yellow}{\parbox{#1}{\color{black} #2}}}
\newcommand{\note}[1]{\fcolorbox{black}{yellow}{\color{black} #1}}
\newcommand{\q}[1]{\begin{question}#1\end{question}}
\newcommand{\g}[1]{\begin{guess}#1\end{guess}}

\def\tn{\textnormal}
\def\mf{\mathfrak}
\def\mc{\mathcal}

\def\ZZ{{\mathbb Z}}
\def\QQ{{\mathbb Q}}
\def\RR{{\mathbb R}}
\def\CC{{\mathbb C}}
\def\AA{{\mathbb A}}
\def\PP{{\mathbb P}}
\def\NN{{\mathbb N}}

\def\bD{{\bf \Delta}}
\def\Str{{\bf Str}}

\def\Hom{\tn{Hom}}
\def\Fun{\tn{Fun}}
\def\Nat{\tn{Nat}}
\def\Ob{\tn{Ob}}
\def\Op{\tn{Op}}

\def\to{\rightarrow}
\def\from{\leftarrow}
\def\cross{\times}
\def\taking{\colon}
\def\inj{\hookrightarrow}
\def\surj{\twoheadrightarrow}
\def\too{\longrightarrow}
\def\fromm{\longleftarrow}
\def\down{\downarrow}
\def\Down{\Downarrow}
\def\Up{\Uparrow}
\def\tooo{\longlongrightarrow}
\def\tto{\rightrightarrows}
\def\ttto{\equiv\!\!>}
\def\ss{\subset}
\def\superset{\supset}
\def\iso{\cong}
\def\|{{\;|\;}}
\def\m1{{-1}}
\def\op{^\tn{op}}
\def\loc{\tn{loc}}
\def\la{\langle}
\def\ra{\rangle}
\def\wt{\widetilde}
\def\wh{\widehat}
\def\we{\simeq}
\def\ol{\overline}
\def\ul{\underline}
\def\qeq{\mathop{=}^?}
\def\setto{\colon\hspace{-.25cm}=}

\def\ullimit{\ar@{}[rd]|(.3)*+{\lrcorner}}
\def\urlimit{\ar@{}[ld]|(.3)*+{\llcorner}}
\def\lllimit{\ar@{}[ru]|(.3)*+{\urcorner}}
\def\lrlimit{\ar@{}[lu]|(.3)*+{\ulcorner}}
\def\ulhlimit{\ar@{}[rd]|(.3)*+{\diamond}}
\def\urhlimit{\ar@{}[ld]|(.3)*+{\diamond}}
\def\llhlimit{\ar@{}[ru]|(.3)*+{\diamond}}
\def\lrhlimit{\ar@{}[lu]|(.3)*+{\diamond}}
\newcommand{\clabel}[1]{\ar@{}[rd]|(.5)*+{#1}}

\newcommand{\arr}[1]{\ar@<.5ex>[#1]\ar@<-.5ex>[#1]}
\newcommand{\arrr}[1]{\ar@<.7ex>[#1]\ar@<0ex>[#1]\ar@<-.7ex>[#1]}
\newcommand{\arrrr}[1]{\ar@<.9ex>[#1]\ar@<.3ex>[#1]\ar@<-.3ex>[#1]\ar@<-.9ex>[#1]}
\newcommand{\arrrrr}[1]{\ar@<1ex>[#1]\ar@<.5ex>[#1]\ar[#1]\ar@<-.5ex>[#1]\ar@<-1ex>[#1]}

\newcommand{\To}[1]{\xrightarrow{#1}}
\newcommand{\Too}[1]{\xrightarrow{\ \ #1\ \ }}
\newcommand{\From}[1]{\xleftarrow{#1}}


\newcommand{\Adjoint}[4]{\xymatrix@1{#2 \ar@<.5ex>[r]^-{#1} & #3 \ar@<.5ex>[l]^-{#4}}}
\newcommand{\adjoint}[4]{\xymatrix@1{#1\colon #2\ar@<.5ex>[r]& #3\;:#4 \ar@<.5ex>[l]}}

\def\id{\tn{id}}
\def\Top{{\bf Top}}
\def\Cat{{\bf Cat}}
\def\Sets{{\bf Sets}}
\def\sSets{{\bf sSets}}
\def\Grpd{{\bf Grpd}}
\def\Pre{{\bf Pre}}
\def\Shv{{\bf Shv}}
\def\Rings{{\bf Rings}}

\def\colim{\mathop{\tn{colim}}}

\def\mcA{\mc{A}}
\def\mcB{\mc{B}}
\def\mcC{\mc{C}}
\def\mcD{\mc{D}}
\def\mcE{\mc{E}}
\def\mcF{\mc{F}}
\def\mcG{\mc{G}}
\def\mcH{\mc{H}}
\def\mcI{\mc{I}}
\def\mcJ{\mc{J}}
\def\mcK{\mc{K}}
\def\mcL{\mc{L}}
\def\mcM{\mc{M}}
\def\mcN{\mc{N}}
\def\mcO{\mc{O}}
\def\mcP{\mc{P}}
\def\mcQ{\mc{Q}}
\def\mcR{\mc{R}}
\def\mcS{\mc{S}}
\def\mcT{\mc{T}}
\def\mcU{\mc{U}}
\def\mcV{\mc{V}}
\def\mcW{\mc{W}}
\def\mcX{\mc{X}}
\def\mcY{\mc{Y}}
\def\mcZ{\mc{Z}}

\newtheorem{theorem}{Theorem}[subsection]
\newtheorem{lemma}[theorem]{Lemma}
\newtheorem{proposition}[theorem]{Proposition}
\newtheorem{corollary}[theorem]{Corollary}
\newtheorem{fact}[theorem]{Fact}

\theoremstyle{remark}
\newtheorem{remark}[theorem]{Remark}
\newtheorem{example}[theorem]{Example}
\newtheorem{warning}[theorem]{Warning}
\newtheorem{question}[theorem]{Question}
\newtheorem{guess}[theorem]{Guess}
\newtheorem{answer}[theorem]{Answer}
\newtheorem{construction}[theorem]{Construction}

\theoremstyle{definition}
\newtheorem{definition}[theorem]{Definition}
\newtheorem{notation}[theorem]{Notation}
\newtheorem{conjecture}[theorem]{Conjecture}
\newtheorem{postulate}[theorem]{Postulate}

\def\DT{{\bf DT}}
\def\GD{{\bf GD}}
\def\DB{\GD}
\def\Sch{{\bf Sch}}
\def\Null{{\bf Null}}
\def\Strings{{\bf Strings}}
\def\ND{{\bf ND}}
\def\Tables{{\bf Tables}}
\def\'{\tn{'}}
\def\disunion{\amalg}
\def\Rel{{\bf Rel}}
\def\mcRel{{\bf \mcR el}}
\def\Cech{$\check{\tn{C}}$ech }
\def\C{\check{\tn{C}}}
\def\Fin{{\bf Fin}}
\def\singleton{{\{*\}}}
\def\Sub{{\bf Sub}}
\def\card{\tn{card}}
\def\Data{{\bf DB}}
\def\DB{{\bf DB}}
\def\im{\tn{im}}
\def\'{\tn{'}}
\def\start{\note{start here}}





\def\TT{\mathbb T}
\def\Set{{\bf Set}}
\def\AD{{\bf AD}}
\def\SD{{\bf SD}}
\def\Pre{{\bf Pre}}
\def\Fin{{\bf Fin}}
\def\ND{{\bf ND}}

\begin{document}

\title{Algebraic databases}

\author{David I. Spivak}

\maketitle


Let $\TT$ denote a multi-sorted algebraic theory.  The objects of $\TT$ are called {\em types}.  For a type $t\in\TT$, let $\Delta^t\taking\TT\to\Set$ denote the functor $\Hom_{\TT}(t,-)$; we call it {the $t$-simplex}.  A morphism of simplices $\delta\taking \Delta^t\to \Delta^u$ is called a {\em face inclusion} if the corresponding map $\delta^\vee\taking u\to t$ in $\TT$ is a projection.  

\begin{definition}

The category $\Set^\TT$ is called the category of {\em $\TT$-schema}.  

A morphism $g\taking A\to B$ of $\TT$-schema is called {\em obedient} if  for every diagram of the first kind (left) there exists a diagram of the second kind (right) $$\xymatrix{\Delta^t\ar[r]\ar[d]_f&A\ar[d]^g&&\Delta^t\ar[r]\ar@/_1.5pc/[dd]_f\ar[d]&A\ar[dd]^g\\\Delta^u\ar[r]&B&\Rightarrow&\Delta^{u'}\ar[d]^\delta\ar[ur]\\&&&\Delta^u\ar[r]&B}$$ where $\delta\taking\Delta^{y'}\to\Delta^y$ is a face inclusion.  

Let $\Omega_X\ss\Set^\TT_{/X}$ denote the full subcategory of $\TT$-schema over $X$ spanned by the obedient morphisms.

\end{definition}

\begin{example}

Let $P$ be the pushout in the diagram $$\xymatrix{\Delta^{\QQ}\ar[r]^{pr}\ar[d]_\div&\Delta^{\QQ,\ZZ}\ar[d]^g\\\Delta^{\ZZ,\ZZ}\ar[r]&P.}$$  Then the structure map $g\taking\Delta^{\QQ,\ZZ}\to P$ is not obedient (in fact the pushout square itself is the counter-example).  

The idea is that if the schema includes some variable $x$ to be ``computed," then that computation must be a part of any data concerning $x$.  In this case the computation is $\div$; since the implicated $\QQ$ variable is part of the map $g$, this map is not obedient.

\end{example}

\begin{example}

More importantly, the obedience concept precludes degeneracies.  If $\Delta^u\To{g}\Delta^t$ is a degeneracy (i.e. is dual to a diagonal map $t\to u$) then $g$ is not obedient.  The idea is that if $B$ is asking for say two integers, then we cannot allow ourselves to provide as data more than two integers and pretend that we are being obedient to $B$.

\end{example}

Let $p\taking\TT\to\Set$ denote a $\TT$-algebra, i.e. a product-preserving functor.  

\begin{definition}

An  {\em algebraic database on $p$} is a sequence $(X,K,\tau)$ where $X$ and $K$ are  $\TT$-schema, and $\tau\taking K\to X\cross p$ is a natural transformation such that the induced map $K\to X$ is obedient.

A morphism of algebraic databases on $p$, written $(X,K_X,\tau_X)\to(Y,K_Y,\tau_Y)$, consists of a map $f\taking Y\to X$ and a map $f^\sharp\taking K_X\cross_XY\to K_Y$ over $Y\cross p$.

The category of algebraic databases on $p$ will be denoted $\AD^p$.

\end{definition}

Note that for any $f\taking X\to Y$, the map $K_X\cross_XY\to Y$ is obedient.

Let $\TT_1\ss\TT$ denote the full subcategory spanned by the generators (atomic types) of the theory.

\begin{proposition}

Suppose $\TT$ is a discrete multi-sorted algebraic theory, let $p$ be a $\TT$-algebra and $p_1\taking\TT_1\to\Set$ its restriction to $\TT_1$.  Define $\pi\taking Gr(p_1)\to\TT_1$ to be the Grothendieck construction of $p_1$.  Then the category $\SD^\pi$ of simplicial databases on $\pi$ (in the sense of \cite{SD}) is equivalent to the category $\AD^p$ of algebraic databases on $p$.

\end{proposition}

\begin{proof}

If $X\in\Pre(\Fin\down\TT_1)$ is a schema in the sense of \cite{SD}, then the Grothendieck construction induces an equivalence between the category $\Pre(\ND(X))$ and the category of obedient $\TT$-schema over $X$.

***

\end{proof}

Given a database $(X,K,\tau\taking K\to X\cross p)$ on $p$, we can consider $pr_2\circ\tau\taking K\to p$ to be data, and $pr_1\circ\tau\taking K\to X$ to be hidden knowledge.  Finally, we can consider $\tau$ to be information.  It's shaped data.  

We can consider $p$ to be a culture.  Our next step should be to allow a change of culture.  In other words, a change of theory.

\bibliographystyle{amsalpha}
\begin{thebibliography}{JTT}

\bibitem{SD}[SD] D.I. Spivak ``Simplicial databases."

\end{thebibliography}

\end{document}