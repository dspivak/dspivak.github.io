\documentclass{amsart}

\usepackage{amssymb, amscd,stmaryrd,setspace,hyperref,color}

\input xy
\xyoption{all} \xyoption{poly} \xyoption{knot}\xyoption{curve}
\input{diagxy}


\newcommand{\comment}[1]{}

\newcommand{\longnote}[2][4.9in]{\fcolorbox{black}{yellow}{\parbox{#1}{\color{black} #2}}}
\newcommand{\note}[1]{\fcolorbox{black}{yellow}{\color{black} #1}}
\newcommand{\q}[1]{\begin{question}#1\end{question}}
\newcommand{\g}[1]{\begin{guess}#1\end{guess}}

\def\tn{\textnormal}
\def\mf{\mathfrak}
\def\mc{\mathcal}

\def\ZZ{{\mathbb Z}}
\def\QQ{{\mathbb Q}}
\def\RR{{\mathbb R}}
\def\CC{{\mathbb C}}
\def\AA{{\mathbb A}}
\def\PP{{\mathbb P}}
\def\NN{{\mathbb N}}

\def\Hom{\tn{Hom}}
\def\Fun{\tn{Fun}}
\def\Ob{\tn{Ob}}
\def\Op{\tn{Op}}

\def\to{\rightarrow}
\def\from{\leftarrow}
\def\cross{\times}
\def\taking{\colon}
\def\inj{\hookrightarrow}
\def\surj{\twoheadrightarrow}
\def\too{\longrightarrow}
\def\tooo{\longlongrightarrow}
\def\tto{\rightrightarrows}
\def\ttto{\equiv\!\!>}
\def\ss{\subset}
\def\superset{\supset}
\def\iso{\cong}
\def\down{\downarrow}
\def\|{{\;|\;}}
\def\m1{{-1}}
\def\op{^\tn{op}}
\def\loc{\tn{loc}}
\def\la{\langle}
\def\ra{\rangle}
\def\wt{\widetilde}
\def\wh{\widehat}
\def\we{\simeq}
\def\ol{\overline}
\def\ul{\underline}
\def\qeq{\mathop{=}^?}

\def\ullimit{\ar@{}[rd]|(.3)*+{\lrcorner}}
\def\urlimit{\ar@{}[ld]|(.3)*+{\llcorner}}
\def\lllimit{\ar@{}[ru]|(.3)*+{\urcorner}}
\def\lrlimit{\ar@{}[lu]|(.3)*+{\ulcorner}}
\def\ulhlimit{\ar@{}[rd]|(.3)*+{\diamond}}
\def\urhlimit{\ar@{}[ld]|(.3)*+{\diamond}}
\def\llhlimit{\ar@{}[ru]|(.3)*+{\diamond}}
\def\lrhlimit{\ar@{}[lu]|(.3)*+{\diamond}}
\newcommand{\clabel}[1]{\ar@{}[rd]|(.5)*+{#1}}

\newcommand{\arr}[1]{\ar@<.5ex>[#1]\ar@<-.5ex>[#1]}
\newcommand{\arrr}[1]{\ar@<.7ex>[#1]\ar@<0ex>[#1]\ar@<-.7ex>[#1]}
\newcommand{\arrrr}[1]{\ar@<.9ex>[#1]\ar@<.3ex>[#1]\ar@<-.3ex>[#1]\ar@<-.9ex>[#1]}
\newcommand{\arrrrr}[1]{\ar@<1ex>[#1]\ar@<.5ex>[#1]\ar[#1]\ar@<-.5ex>[#1]\ar@<-1ex>[#1]}

\newcommand{\To}[1]{\xrightarrow{#1}}
\newcommand{\Too}[1]{\xrightarrow{\ \ #1\ \ }}
\newcommand{\From}[1]{\xleftarrow{#1}}

\newcommand{\Adjoint}[4]{\xymatrix@1{#2 \ar@<.5ex>[r]^-{#1} & #3 \ar@<.5ex>[l]^-{#4}}}

\def\id{\tn{id}}
\def\Top{{\bf Top}}
\def\Cat{{\bf Cat}}
\def\Str{{\bf Str}}
\def\Sets{{\bf Sets}}
\def\Set{{\bf Set}}
\def\sSet{{\bf sSet}}
\def\sSets{{\bf sSets}}
\def\Grpd{{\bf Grpd}}
\def\Pre{{\bf Pre}}
\def\Shv{{\bf Shv}}
\def\Rings{{\bf Rings}}

\def\colim{\mathop{\tn{colim}}}

\def\mcA{\mc{A}}
\def\mcB{\mc{B}}
\def\mcC{\mc{C}}
\def\mcD{\mc{D}}
\def\mcE{\mc{E}}
\def\mcF{\mc{F}}
\def\mcG{\mc{G}}
\def\mcH{\mc{H}}
\def\mcI{\mc{I}}
\def\mcJ{\mc{J}}
\def\mcK{\mc{K}}
\def\mcL{\mc{L}}
\def\mcM{\mc{M}}
\def\mcN{\mc{N}}
\def\mcO{\mc{O}}
\def\mcP{\mc{P}}
\def\mcQ{\mc{Q}}
\def\mcR{\mc{R}}
\def\mcS{\mc{S}}
\def\mcT{\mc{T}}
\def\mcU{\mc{U}}
\def\mcV{\mc{V}}
\def\mcW{\mc{W}}
\def\mcX{\mc{X}}
\def\mcY{\mc{Y}}
\def\mcZ{\mc{Z}}

\newtheorem{theorem}{Theorem}[section]
\newtheorem{lemma}[theorem]{Lemma}
\newtheorem{proposition}[theorem]{Proposition}
\newtheorem{corollary}[theorem]{Corollary}
\newtheorem{fact}[theorem]{Fact}

\theoremstyle{remark}
\newtheorem{remark}[theorem]{Remark}
\newtheorem{example}[theorem]{Example}
\newtheorem{warning}[theorem]{Warning}
\newtheorem{question}[theorem]{Question}
\newtheorem{guess}[theorem]{Guess}
\newtheorem{answer}[theorem]{Answer}
\newtheorem{construction}[theorem]{Construction}

\theoremstyle{definition}
\newtheorem{definition}[theorem]{Definition}
\newtheorem{notation}[theorem]{Notation}
\newtheorem{conjecture}[theorem]{Conjecture}
\newtheorem{postulate}[theorem]{Postulate}

\def\Finm{{\bf Fin_{m}}}
\def\El{{\bf El}}
\def\Gr{{\bf Gr}}
\def\DT{{\bf DT}}
\def\DB{{\bf DB}}
\def\Tables{{\bf Tables}}
\def\Sch{{\bf Sch}}
\def\Fin{{\bf Fin}}
\def\P{{\bf P}}
\def\SC{{\bf SC}}
\def\ND{{\bf ND}}
\def\Poset{{\bf Poset}}

%%%%%

\begin{document}

\title{Simplicial database aggregation}

\author{David I. Spivak}

\thanks{This project was supported in part by a grant from the Office of Naval Research: N000140910466.}

\maketitle

\section{Introduction}

In \cite{SD} we defined a category of simplicial databases, denoted $\DB$.  An object of $\mcX\in\DB$ consists of a pair $(X,\mcO_X)$, where $X$ is a labelled simplicial set whose simplices represent table-types, and where $\mcO_X$ is a sheaf of records on $X$.  The geometric realization of $X$ is drawn as a union of $n$-simplices, in which connections indicate foreign key constraints via restriction.  We call $X$ the schema of $\mcX$; it takes the place of an ER-diagram for the database $\mcX$.  

Queries such as selects and joins correspond to limits in $\DB$ and manipulations such as insertions and unions correspond to colimits in $\DB$.  Issues of ``database definition" correspond to morphisms of schemas $f\taking X\to Y$.  Such a change in structure induces functors $f_*\taking\DB_X\to\DB_Y,f_!\taking\DB_X\to\DB_Y,$ and $f^*\taking\DB_Y\to\DB_X$ which are adjoints as indicated by the notation (in the parlance of sheaves,\cite{MM}).

We showed in \cite{TM} that the geometric viewpoint has practical meaning.  One can picture  tables as ``tiles" which can be joined together to form complex data definitions. That is, a database administrator could assemble tiles as needed for the particular query operation to be performed and the categorical underpinnings would take care of referential integrity issues.  Moreover, curves drawn through such a formation of tiles correspond to the queries themselves, again indicating the value of the geometric aspect of the theory.

In this paper we describe the operation of expanding and collapsing sub-databases. This is done for various reasons....  

\begin{example}

Draw two 2-simplices glued together along a common edge.  One simplex is labeled $(A_1,A_2,B)$ and the other labeled $(A_1,A_2,C)$; they are glued together along the edge $(A_1,A_2)$.  Draw a 1-simplex connecting $A_1$ to a new vertex $D$.  

Suppose we wish to collapse the simplex $(A_1,A_2)$. The result is shaped like a Y: it has four vertices labeled $A,B,C,D$, it has three 1-simplices labeled $(A,B),(C,A),$ and $(D,A)$, and it has no $n$-simplices for $n\geq 2$.  

\end{example}

\begin{definition}

Let $\pi\taking U\to\DB$ denote a type signature and let $\mcA=(A,\mcO_A)$ denote a database of type $\pi$.  Let $\ol{\mcA}=(\Delta^{A_0},\mcO_A(A))$ denote the ``join" of $\mcA$, i.e. table whose columns are the vertices of $A$ and whose data are global sections in $\mcA$.  Let $$T_\mcA=\Ob(\Tables(A_0)_{/\mcO_A(A)}$$ denote the set of tables mapping to $\mcA$.

We define the {\em addition of $\mcA$ to $\pi$}, denoted $\pi\oplus\ol{\mcA}$ to be the function $$\xymatrix{U\amalg T_\mcA\ar[d]\\\DT\amalg\{`T_\mcA`\}.}$$  We denote the 0-simplex $\Delta^{\{`T_\mcA`\}}$ by $\lozenge\in\Ob(\Sch^{\pi'})$.
 
\end{definition}

At present, we have a working construction only in the case that the schema $X$ is a ``simplicial complex" as in Definition \ref{def:simplicial complex}.  While the work in \cite{SD} is much more general, most databases that currently exist are modeled on a simplicial complex.  Roughly that means that no two tables can have the same set of attributes (though they can have the same set of datatypes).  

In the definition below, we write $\P(V)$ to denote the power-set of a set $V$, and for elements $s,t\in\P(V)$ we write $t\leq s$ to mean that $t\ss s\ss V$.  Given a function $f\taking V\to V'$ and subset $s\ss V$, we write $f_!(s)$ to denote its image in $V'$.

\begin{definition}\label{def:simplicial complex}

A {\em simplicial complex over $\DT$ } consists of a pair $(V,S)$ where $V$ is a set over $\DT$ and $S\ss\P(V)$ is a set of subsets of $V$ such that if $s\in S$ and $t\leq s$ then $t\in S$.  Note that $S$ is in fact a poset.

A {\em morphism of simplicial complexes over $\DT$} $f\taking (V,S)\to (V',S')$ is a function $f\taking V\to V'$ over $\DT$ such that $f_!(s)\in S'$ for all $s\in S$.  

An {\em attribute-defined simplicial database} is a simplicial database $(X,\mcK,\tau)$ such that $X=(V,S)$ is a simplicial complex over $\DT$.  In other words $\mcK$ is a functor $S\op\to\Sets$ and $\tau\taking\mcK\to\mcU_X$ is a natural transformation, defined as in \cite{SD}.

\end{definition}

Suppose that $A\ss V$ is a subset; we write $V/A$ to denote the pushout of the diagram $V\from A\to\{\lozenge\}$, and we write  $q\taking V\to V/A$ to denote the quotient map, which is isomorphic on $V-A$ and sends all of $A$ to $\lozenge$.

\begin{definition}

Let $X=(V,S)$ be a simplicial complex and let $A\ss V$ be a set of vertices.  Let $X/A=(W,T)$ where $W=X/A$ and $T=\{f_!(s)|s\in S\}$.  

If $X$ is a simplicial complex over $\DT$, and $A\ss V$

\end{definition}



\end{document}

