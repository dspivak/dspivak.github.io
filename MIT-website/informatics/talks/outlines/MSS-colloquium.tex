\documentclass{amsart}

\usepackage{amssymb, amscd,stmaryrd,setspace,hyperref,color}
\usepackage{lmodern}
\usepackage{outline}

\oddsidemargin0in
\evensidemargin0in
\topmargin-.2in
\textheight9.4in
\textwidth6.5in

\input xy
\xyoption{all} \xyoption{poly} \xyoption{knot}\xyoption{curve}

\def\sub{\begin{outline}\item}
\def\endsub{\end{outline}}
\def\eendsub{\endsub\endsub}
\def\eeendsub{\endsub\endsub\endsub}
\def\next{\item}
\def\nnext{\endsub\next}
\def\nnnext{\eendsub\next}

\def\to{\rightarrow}
\def\To{\xrightarrow}
\def\tto{\rightrightarrows}
\def\from{\leftarrow}
\def\too{\longrightarrow}
\def\fromm{\longleftarrow}
\def\NN{{\mathbb N}}
\def\ZZ{{\mathbb Z}}
\def\RR{{\mathbb R}}
\def\String{{\bf String}}
\def\DT{{\bf DT}}
\def\U{{\bf U}}
\def\DB{{\bf DB}}
\def\taking{\colon}
\def\mcO{{\mathcal O}}
\def\mcU{{\mathcal U}}
\def\Sub{{\bf Sub}}
\def\op{^{\text{op}}}
\def\Sets{{\bf Sets}}
\def\Tables{{\bf Tables}}
\def\To{\xrightarrow}
\def\Fin{{\bf Fin}}
\def\cross{\times}
\def\Ob{{\bf Ob}}
\def\Hom{{\bf Hom}}
\def\Set{{\bf Set}}
\def\Arr{{\bf Arr}}
\def\Path{{\bf Path}}
\def\id{\tn{id}}
\def\Pre{{\bf Pre}}
\def\Fin{{\bf Fin}}
\def\El{{\bf El}}
\def\Cat{{\bf Cat}}
\def\down{\downarrow}
\def\set{{\bf -Set}}
\def\mcC{{\mathcal C}}
\def\mcD{{\mathcal D}}
\def\mcE{{\mathcal E}}
\def\mcK{{\mathcal K}}
\def\mcS{{\mathcal S}}
\def\P{{\bf P}}
\newcommand{\LMO}[1]{\bullet^{#1}}
\newcommand{\LTO}[1]{\bullet^{\tn{#1}}}
\newcommand{\LA}[2]{\ar[#1]^-{\tn {#2}}}

\newcommand{\obox}[3]{\stackrel{#1}{\fbox{\parbox{#2}{#3}}}}
\newcommand{\smbox}[2]{\stackrel{#1}{\fbox{#2}}}
\newcommand{\labox}[2]{\obox{#1}{1.6in}{#2}}
\newcommand{\mebox}[2]{\obox{#1}{1in}{#2}}


\def\edge{\ar@{-}}

\def\ullimit{\ar@{}[rd]|(.3)*+{\lrcorner}}
\def\urlimit{\ar@{}[ld]|(.3)*+{\llcorner}}
\def\lllimit{\ar@{}[ru]|(.3)*+{\urcorner}}
\def\lrlimit{\ar@{}[lu]|(.3)*+{\ulcorner}}

\theoremstyle{definition}
\newtheorem{definition}{Definition}
\newtheorem{example}{Example}



\def\tn{\textnormal}
\def\too{\longrightarrow}

\newcommand{\longnote}[2][4.9in]{\fcolorbox{black}{yellow}{\parbox{#1}{\color{black} #2}}}
\newcommand{\shortnote}[1]{\fcolorbox{black}{yellow}{\color{black} #1}}

\begin{document}

\title{\large Categorical Information Theory\\~\\\normalsize MSS Colloquium at UIUC\\2011/11/29}

\author{David I. Spivak}

\maketitle

\begin{abstract}

Classical information theory, as developed by Claude Shannon in 1948, studies how to optimize the quantity of data that can be transmitted across a noisy channel, but it ignores what this information "means". It is clear that information is intended to mean or signify something -- this is its only purpose -- but how can such a thing as meaning be formalized? In this talk I'll discuss how category theory may be useful in this endeavor. Going further, we may postulate that information is always intended as a communication from one party to another (perhaps from an entity to a later version of itself). In this case we can ask, "what is the relationship between the structure of the information being transferred and the structure of the two communicating parties?" I will outline a possible answer to this question in the form of a category-theoretic communication protocol for transferring information between parties. 

No prior knowledge of category theory will be necessary to understand this talk.


\end{abstract}

\sub Introduction
	\sub What is Information?
		\sub ``Information theory" vs. Meaning
		\next Examples -- dictionary, engineers schematic diagram, database
		\next Information vs. data
		\next Relational, not absolute
	\nnext Historical context
		\sub Frege -- Begriffsschrift
		
		\next Hilbert: ``This formula game is carried out according to certain definite rules, in which the technique of our thinking is expressed. [...] The fundamental idea of my proof theory is none other than to describe the activity of our understanding, to make a protocol of the rules according to which our thinking actually proceeds" (Hilbert, 1928, 475).
		
		\next Dan Kan: ``Information is inherently a combinatorial affair." Facts are fit together in appropriate ways to create new facts.
	\nnext Category theory
		\sub History
			\sub Invented to connect topology and algebra
			\next Became unifying language for lots of math
			\next Successfully applied in CS -- programming languages
		\nnext Algebraic geometry
			\sub Grothendieck and Weil conjectures (proved power of CT)
			\next Idea: sheaves (hard to talk about without categories)
			\next Local sections (we'll talk more about this)
			\endsub
	\nnext Databases
		\sub Why Databases? Information!
		\next Example -- tables connected in a specific way
		\next I'll give a simple CT formulation.
	\nnext Outline of the talk
		\sub Basic category theory
		\next Databases
		\next Change of schema
		\next Queries
		\next Communication (time permitting)
	\eendsub
\next Basic category theory
	\sub Categories
		\sub A category $\mcC$ is a graph in which paths can be declared equivalent
			\sub A set $\Ob(\mcC)$
			\next A set of arrows $\Arr(\mcC)$ each with a source and target object.
			\next An equivalence relation on paths ($\Path(\mcC)$), respecting source and target, and preserved under composition.
		\nnext Examples of categories
			\sub Random thing.
			\next $E\tto V$
			\next $\Set$ (saturated)
			\next Topological spaces, manifolds, rings, etc.
			\endsub
	\nnext Functors
		\sub A functor $F\taking\mcC\to\mcD$ is a mapping between categories
			\sub A map $\Ob(\mcC)\to\Ob(\mcD)$
			\next A map $\Arr(\mcC)\to\Path(\mcD)$
				\sub If $\mcD$ is saturated then we may assume $\Arr(\mcC)\to\Arr(\mcD)$.
			\nnext Equivalent paths in $\mcC$ are sent to equivalent paths in $\mcD$.
		\nnext Examples
			\sub {\bf Rings}$\to\Set$
			\next $\fbox{$E\tto V$}\to\fbox{\xymatrix{N\ar@(u,r)[]^~}}$
			\next $\fbox{$E\tto V$}\to\Set$. (Draw tables)
			\next $\fbox{\xymatrix{N\ar@(u,r)[]^~}}\to\Set$
			\endsub
		\endsub
	\endsub
\next Databases
	\sub  \begin{align}\label{dia:basic cat} \mcC:=\fbox{\xymatrix{\LTO{Employee}\ar@<.5ex>[rr]^{\tn{Dpt}}\ar@(l,u)[]^{\tn{Mgr}}\ar@/_1pc/[dd]_{\tn{First}}\ar@/^1pc/[dd]^{\tn{Last}}&&\LTO{Department}\ar@<.5ex>[ll]^{\tn{Secr'y}}\ar@/^1pc/[ddll]^{\tn{Name}}\\\\\LTO{String}}}\end{align} 

	\next $\mcC\set$ is a topos
		\sub Morphisms are called natural transformations
		\next Subobjects, unions, products, quotients
		\next Logic (typed lambda calculus)
	\nnext The graph example \fbox{$E\tto V$} in terms of databases
		\sub Morphisms of graphs
	\eendsub
\next Morphism of schemas
	\sub Example: $\fbox{$E\tto V$}\to\fbox{\xymatrix{N\ar@(u,r)[]^~}}$
		\sub Pulling back data
		\next Pushing forward data?
	\nnext For every functor $F\taking\mcC\to\mcD$ exists 
		\sub $\Delta_F\taking\mcD\set\to\mcC\set$,
		\next $\Sigma_F\taking\mcC\set\to\mcD\set$, and
		\next $\Pi_F\taking\mcC\set\to\mcD\set$.
	\nnext DB example
		\sub \small\begin{align}\label{dia:basic example} \stackrel{\mcC:=}{\fbox{\xymatrix{&\tn{SSN}\\&\tn{First}\\T1\ar[uur]\ar[ur]\ar[dr]&&T2\ar[ul]\ar[dl]\ar[ddl]\\&\tn{Last}\\&\tn{Salary}}}}&\stackrel{F}{\too}\stackrel{\mcD:=}{\fbox{\xymatrix{&\tn{SSN}\\&\tn{First}\\U\ar[uur]\ar[ur]\ar[dr]\ar[ddr]\\&\tn{Last}\\&\tn{Salary}}}}\stackrel{G}{\fromm}\stackrel{\mcE:=}{\fbox{\xymatrix{&\tn{SSN}\\&\tn{First}\\V\ar[uur]\ar[ur]\ar[dr]\\&\tn{Last}}}}\end{align}\normalsize
		\endsub
	\endsub

\next Queries
	\sub Grothendieck construction 
		\sub Given $\delta\taking\mcC\to\Set$, get $\int\delta\to\mcC$
		\next Objects: $\{(c,x)|c\in\Ob(\mcC), x\in\delta(c)\}$
		\next An arrow $(c,x)\to(c',x')$ for every $f\taking c\to c'$ such that $f(x)=x'$.
		\next In other words, objects are id-cells, arrows are other cells.
		\next The fiber over each object is its set of instances.
	\nnext Local sections
		\sub E.g. $$\parbox{1.2in}{\fbox{\xymatrix{\LTO{Emp1}\ar[dr]\\&\LTO{Last}\\\LTO{Emp2}\ar[ur]}}}\too\parbox{2.2in}{\fbox{\xymatrix{\LTO{Employee}\ar@<.5ex>[rr]^{\tn{Dpt}}\ar@(l,u)[]^{\tn{Mgr}}\ar@/_1pc/[dd]_{\tn{First}}\ar@/^1pc/[dd]^{\tn{Last}}&&\LTO{Department}\ar@<.5ex>[ll]^{\tn{Secr'y}}\ar@/^1pc/[ddll]^{\tn{Name}}\\\\\LTO{String}}}}$$
		\next A lift is the same as two employees with same last name.
		\next Find two employees, one named Bob, one named Sue, that have the same manager: $$\parbox{.9in}{\fbox{\xymatrix{&\LTO{Bob}\\\\&\LTO{Sue}}}}\too\parbox{1.2in}{\fbox{\xymatrix{\LTO{Emp1}\ar[r]\ar[dr]&\LTO{First}\\&\LTO{Mgr}\\\LTO{Emp2}\ar[r]\ar[ur]&\LTO{First}}}}$$
		\next Same idea: I have a topological space $X$. I know that $\pi_1(X)=pi_2(X)=0,\pi_3(X)=\ZZ$ and $H_1(X)=0$.  $$\mcC:=\parbox{4in}{\fbox{\xymatrix{\obox{P}{1.1in}{a positive number $n\in\NN_{\geq1}$}&\obox{M}{1in}{a positive number $n\in\NN_{\geq 1}$ and a space $X$}\ar[l]_{n}\ar[r]^X\ar[dl]_(.6){\pi_n(X)}\ar[dr]^(.6){H_n(X)}&\obox{S}{.7in}{a space $X$}\\\obox{G}{.6in}{a group}&&\obox{A}{1in}{an abelian group}}}}$$
		\next Like integer sequence database.
		\endsub
	\endsub
\next Communication
	\sub Basic idea: 
		\sub Everyone has a schema and data
		\next Interacting groups have their own schema and data mapping to each subgroup
		\next A simplicial complex of interactions.
	\nnext There is a nice model for communicating new information
		\sub Can be done within category theory.
		\next See me afterwards if interested.
	\eendsub
\endsub




\end{document}