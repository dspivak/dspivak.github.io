\documentclass{amsart}

\usepackage{amssymb, amscd,stmaryrd,setspace,hyperref,color}

\setcounter{secnumdepth}{2}

\input xy
\xyoption{all} \xyoption{poly} \xyoption{knot}\xyoption{curve}


\newcommand{\comment}[1]{}

\comment{The following is Eli Lebow's trick for making the table of contents include all definitions and labels.

\setcounter{tocdepth}{5}

\newcommand{\tocnote}[1]{\addcontentsline{toc}{subsubsection}{#1}}

%\newcommand {\mylabel} {\label}
\newcommand{\mylabel}[1]{\addcontentsline{toc}{subsubsection}{$\{$\texttt{#1}$\}$}\label{#1}}

%\newcommand{\defword}[1]{\textit{#1}}
\newcommand{\defword}[1]{\textit{#1}\addcontentsline{toc}{subsubsection}{\textit{#1}}}

}

\newcommand{\longnote}[2][4.9in]{\fcolorbox{black}{yellow}{\parbox{#1}{\color{black} #2}}}
\newcommand{\note}[1]{\fcolorbox{black}{yellow}{\color{black} #1}}
\newcommand{\shortnote}[1]{\fcolorbox{black}{yellow}{\color{black} #1}}
\newcommand{\q}[1]{\begin{question}#1\end{question}}
\newcommand{\g}[1]{\begin{guess}#1\end{guess}}
\newcommand{\beqn}[1]{\begin{eqnarray}\label{#1}}
\newcommand{\eeqn}{\end{eqnarray}}

\def\tn{\textnormal}
\def\mf{\mathfrak}
\def\mc{\mathcal}

\def\ZZ{{\mathbb Z}}
\def\QQ{{\mathbb Q}}
\def\RR{{\mathbb R}}
\def\CC{{\mathbb C}}
\def\AA{{\mathbb A}}
\def\PP{{\mathbb P}}
\def\NN{{\mathbb N}}

\def\Cech{$\check{\textnormal{C}}$ech }
\def\C{\check C}

\def\Aut{\tn{Aut}}
\def\Tor{\tn{Tor}}
\def\Sym{\tn{Sym}}
\def\im{\tn{im}}
\def\coker{\tn{coker}}
\def\Spec{\tn{Spec}}
\def\Supp{\tn{Supp }}
\def\dim{\tn{dim}}
\def\sheafHom{\mathcal{H}om}
\def\Stab{\tn{Stab}}
\def\Fun{\tn{Fun}}
\def\mod{{\bf \tn{-mod}}}
\def\alg{{\bf \tn{-alg}}}
\def\ho{\tn{ho}}

\def\Hom{\tn{Hom}}
\def\Ob{\tn{Ob}}
\def\Mor{\tn{Mor}}
\def\End{\tn{End}}
\def\Map{\tn{Map}}
\def\sheafMap{\mathbf{Map}}
\def\map{\mathbf{map}}
\def\sheafmap{\mathbf{map}}
\def\coeq{\tn{CoEq}}
\def\Op{\tn{Op}}

\def\to{\rightarrow}
\def\from{\leftarrow}
\def\cross{\times}
\def\taking{\colon}
\def\inj{\hookrightarrow}
\def\surj{\twoheadrightarrow}
\def\too{\longrightarrow}
\def\tooo{\longlongrightarrow}
\def\tto{\rightrightarrows}
\def\ttto{\equiv\!\!>}
\def\ss{\subset}
\def\superset{\supset}
\def\iso{\cong}
\def\|{{\;|\;}}
\def\m1{{-1}}
\def\op{^\tn{op}}
\def\loc{\tn{loc}}
\def\la{\langle}
\def\ra{\rangle}
\def\wt{\widetilde}
\def\wh{\widehat}
\def\we{\simeq}
\def\ol{\overline}
\def\ul{\underline}
\def\qeq{\mathop{=}^?}

\def\ullimit{\ar@{}[rd]|(.3)*+{\lrcorner}}
\def\urlimit{\ar@{}[ld]|(.3)*+{\llcorner}}
\def\lllimit{\ar@{}[ru]|(.3)*+{\urcorner}}
\def\lrlimit{\ar@{}[lu]|(.3)*+{\ulcorner}}
\def\ulhlimit{\ar@{}[rd]|(.3)*+{\diamond}}
\def\urhlimit{\ar@{}[ld]|(.3)*+{\diamond}}
\def\llhlimit{\ar@{}[ru]|(.3)*+{\diamond}}
\def\lrhlimit{\ar@{}[lu]|(.3)*+{\diamond}}
\newcommand{\clabel}[1]{\ar@{}[rd]|(.5)*+{#1}}

\newcommand{\arr}[1]{\ar@<.5ex>[#1]\ar@<-.5ex>[#1]}
\newcommand{\arrr}[1]{\ar@<.7ex>[#1]\ar@<0ex>[#1]\ar@<-.7ex>[#1]}
\newcommand{\arrrr}[1]{\ar@<.9ex>[#1]\ar@<.3ex>[#1]\ar@<-.3ex>[#1]\ar@<-.9ex>[#1]}
\newcommand{\arrrrr}[1]{\ar@<1ex>[#1]\ar@<.5ex>[#1]\ar[#1]\ar@<-.5ex>[#1]\ar@<-1ex>[#1]}

\newcommand{\To}[1]{\xrightarrow{#1}}
\newcommand{\Too}[1]{\xrightarrow{\ \ #1\ \ }}
\newcommand{\From}[1]{\xleftarrow{#1}}

\newcommand{\push}[4]{\xymatrix{#1\ar[r]\ar[d] \ar@{}[rd]|(.7)*+{\lrcorner} & #2 \ar[d] \\ #3 \ar[r] & #4}}
\newcommand{\Push}[8]{\xymatrix{#1\ar[r]^-{#5}\ar[d]_-{#6} \ar@{}[rd]|(.7)*+{\lrcorner} & #2 \ar[d]^-{#7} \\ #3 \ar[r]_-{#8} & #4}}
\newcommand{\pull}[4]{\xymatrix{#1\ar[r]\ar[d] \ar@{}[rd]|(.3)*+{\ulcorner} & #2 \ar[d] \\ #3 \ar[r] & #4}}
\newcommand{\Pull}[8]{\xymatrix{#1\ar[r]^-{#5}\ar[d]_-{#6} \ar@{}[rd]|(.3)*+{\ulcorner} & #2 \ar[d]^-{#7} \\ #3 \ar[r]_-{#8} & #4}}
\newcommand{\hpush}[4]{\xymatrix{#1\ar[r]\ar[d] \ar@{}[rd]|(.7)*+{\diamond} & #2 \ar[d] \\ #3 \ar[r] & #4}}
\newcommand{\hPush}[8]{\xymatrix{#1\ar[r]^{#5}\ar[d]_{#6} \ar@{}[rd]|(.7)*+{\diamond} & #2 \ar[d]^{#7} \\ #3 \ar[r]_{#8} & #4}}
\newcommand{\hpull}[4]{\xymatrix{#1\ar[r]\ar[d] \ar@{}[rd]|(.3)*+{\diamond} & #2 \ar[d] \\ #3 \ar[r] & #4}}
\newcommand{\hPull}[8]{\xymatrix{#1\ar[r]^-{#5}\ar[d]_-{#6} \ar@{}[rd]|(.3)*+{\diamond} & #2 \ar[d]^-{#7} \\ #3 \ar[r]_-{#8} & #4}}

\newcommand{\sq}[4]{\xymatrix{#1\ar[r]\ar[d] & #2 \ar[d] \\ #3 \ar[r] & #4}}
\newcommand{\Sq}[8]{\xymatrix{#1\ar[r]^-{#5}\ar[d]_-{#6} & #2 \ar[d]^-{#7} \\ #3 \ar[r]_-{#8} & #4}}
\newcommand{\sqlabel}[5]{\xymatrix{#1\ar[r]\ar[d]\clabel{#5} & #2 \ar[d] \\ #3 \ar[r] & #4}}
\newcommand{\Sqlabel}[9]{\xymatrix{#1\ar[r]^-{#5}\ar[d]_-{#6}\clabel{#9} & #2 \ar[d]^-{#7} \\ #3 \ar[r]_-{#8} & #4}}

\newcommand{\hsq}[4]{\xymatrix{#1\ar[r]\ar[d]\clabel{\diamond} & #2 \ar[d] \\ #3 \ar[r] & #4}}
\newcommand{\hSq}[8]{\xymatrix{#1\ar[r]^-{#5}\ar[d]_-{#6}\clabel{\diamond} & #2 \ar[d]^-{#7} \\ #3 \ar[r]_-{#8} & #4}}

\newcommand{\adjoint}[2]{\xymatrix@1{#1\ar@<.5ex>[r] & #2 \ar@<.5ex>[l]}}
\newcommand{\Adjoint}[4]{\xymatrix@1{#2 \ar@<.5ex>[r]^-{#1} & #3 \ar@<.5ex>[l]^-{#4}}}
\newcommand{\lamout}[3]{\xymatrix{#1 \ar[r]\ar[d] & #2\\ #3 &}}
\newcommand{\lamin}[3]{\xymatrix{& #1\ar[d]\\ #2\ar[r]& #3}}
\newcommand{\Lamout}[5]{\xymatrix{#1 \ar[r]^{#4}\ar[d]_{#5} & #2\\ #3 &}}
\newcommand{\Lamin}[5]{\xymatrix{& #1\ar[d]^{#4}\\ #2\ar[r]^{#5}& #3}}

\newcommand{\overcat}[1]{_{/#1}}

\def\id{\tn{id}}
\def\Top{{\bf Top}}
\def\Cat{{\bf Cat}}
\def\Sets{{\bf Sets}}
\def\sSets{{\bf sSets}}
\def\Grpd{{\bf Grpd}}
\def\Pre{{\bf Pre}}
\def\She{{\bf Shv}}
\def\Rings{{\bf Rings}}

\def\colim{\mathop{\tn{colim}}}
\def\hocolim{\mathop{\tn{hocolim}}}
\def\holim{\mathop{\tn{holim}}}

\def\mfC{\mf{C}}

\def\mcA{\mc{A}}
\def\mcB{\mc{B}}
\def\mcC{\mc{C}}
\def\mcD{\mc{D}}
\def\mcE{\mc{E}}
\def\mcF{\mc{F}}
\def\mcG{\mc{G}}
\def\mcH{\mc{H}}
\def\mcI{\mc{I}}
\def\mcJ{\mc{J}}
\def\mcK{\mc{K}}
\def\mcL{\mc{L}}
\def\mcM{\mc{M}}
\def\mcN{\mc{N}}
\def\mcO{\mc{O}}
\def\mcP{\mc{P}}
\def\mcQ{\mc{Q}}
\def\mcR{\mc{R}}
\def\mcS{\mc{S}}
\def\mcT{\mc{T}}
\def\mcU{\mc{U}}
\def\mcV{\mc{V}}
\def\mcW{\mc{W}}
\def\mcX{\mc{X}}
\def\mcY{\mc{Y}}
\def\mcZ{\mc{Z}}

\def\star{\ast}
\def\singleton{{\{\ast\}}}
\def\tensor{\otimes}

\newtheorem{theorem}[subsection]{Theorem}
\newtheorem{lemma}[subsection]{Lemma}
\newtheorem{proposition}[subsection]{Proposition}
\newtheorem{corollary}[subsection]{Corollary}
\newtheorem{fact}[subsection]{Fact}

\theoremstyle{remark}
\newtheorem{remark}[subsection]{Remark}
\newtheorem{example}[subsection]{Example}
\newtheorem{warning}[subsection]{Warning}
\newtheorem{question}[subsection]{Question}
\newtheorem{guess}[subsection]{Guess}
\newtheorem{answer}[subsection]{Answer}
\newtheorem{construction}[subsection]{Construction}
\newtheorem{problem}[subsection]{Problem}

\theoremstyle{definition}
\newtheorem{definition}[subsection]{Definition}
\newtheorem{notation}[subsection]{Notation}
\newtheorem{conjecture}[subsection]{Conjecture}
\newtheorem{postulate}[subsection]{Postulate}



\def\Shv{\She}
\def\UM{{\bf UM}}
\def\bD{{\bf \Delta}}
\def\Fuz{{\bf Fuz}}
\def\sFuz{{\bf sFuz}}

\begin{document}

\begin{abstract}

We discuss fuzzy simplicial sets, and their relationship to something like metric spaces.  Namely, we present an adjunction between the categories: a metric realization functor and fuzzy singular complex functor.

The following document is a rough draft and may have (substantial) errors.

\end{abstract}

\title{Metric realization of fuzzy simplicial sets}

\author{David I. Spivak}

\thanks{This project was supported in part by a grant from the Office of Naval Research: N000140910466.}

\maketitle

\section{Fuzzy simplicial sets}

Let $I$ denote the Grothendieck site whose objects are initial open intervals contained in the half-open unit interval $[0,1)\in\RR$, whose morphisms are inclusions of open subsets, and whose covers are open covers.  In other words, as a category, $I$ is equivalent to the partially ordered set $(0,1]$ under the relation $\leq$.  

A sheaf $S\in\Shv(I)$ on $I$ is a functor $S\taking I\op\to\Sets$ satisfying the sheaf condition.  Explicitly, $S$ consists of a set $S([0,a))$ for all $a\in(0,1]$, which we choose to denote by $S^{\geq a}$,  and restriction maps $\rho_{b,a}\taking S^{\geq b}\to S^{\geq a}$ for all $b\geq a$, such that if $c\geq b\geq a$ then $\rho_{b,a}\circ\rho_{c,b}=\rho_{c,a}$, and such that for all $a\in I$, one has $$S^{\geq a}\iso\lim_{a'<a}S^{\geq a'}.$$

A sheaf $S$ is called a {\em fuzzy set} if for each $b\geq a$ in $(0,1]$, the restriction map $\rho_{b,a}$ is injective.  Let $\Fuz$ denote the full subcategory of $\Shv(I)$ spanned by the fuzzy sets.  This definition is slightly different than Goguen's \cite{Gog}, but is closely related.  See \cite{Bar}.  The difference between fuzzy sets $T$ and arbitrary sheaves $S\in\Shv(I)$ is that, in $T$ two elements are either equal or they are not, whereas two elements $x\neq y\in S^{\leq a}$ may be equal {\em to a certain degree}, $\rho_{a,c}(x)=\rho_{a,c}(y)$ for some $c<a$.

Suppose $S\in\Shv(I)$ is a sheaf.   For $a\in (0,1],$ let $S(a)=S^{\geq a}-\colim_{b>a}S^{\geq b}$, and note that $S^{\geq a}=\colim_{b\geq a}\rho_{b,a}[S(b)].$  If $T$ is a fuzzy set, we can make this easier on the eyes: $$T^{\geq a}=\coprod_{b\geq a}T(b).$$  We write $x\in S$ and say that {\em $x$ is an element of $S$}, if there exists $a\in (0,1]$ such that $x\in S(a)$; in this case we may say that {\em $x$ is an element of $S$ with strength $a$}.

The following lemma says that, under a map of fuzzy sets, an element cannot be sent to an element of lower strength.

\begin{lemma}

Suppose that $S$ and $T$ are fuzzy sets.  If $f\taking S\to T$ is a morphism of fuzzy sets, then for all $a,b\in (0,1]$, if $x\in S(a)$ then $f(x)\in T(b)$ for some $b\geq a$.

\end{lemma}

\begin{proof}

Since $x\in S^{\geq a}$, we have by definition that $f(x)\in T^{\geq a}$, so $x\in T(b)$ for some $b\geq a$.

\end{proof}

\begin{lemma}

The forgetful functor $\Fuz\to\Shv(I)$ is fully faithful and has a left adjoint $m$.  Thus $\Fuz$ is closed under taking colimits.

\end{lemma}

\begin{proof}

Given a sheaf $S\taking I\op\to\Sets$ and $a\in (0,1]$, let $(mS)^{\geq a}=S^{\geq a}/\sim$, where for $x,x'\in S^{\geq a}$, we set $x\sim x'$ if there exists $b\leq a$ such that $\rho_{a,b}(x)=\rho_{a,b}(x')$.  Clearly, $mS$ is a fuzzy set, and one checks that $m$ is left adjoint to the forgetful functor.  

To compute the colimit of a diagram in $\Fuz$, one applies the forgetful functor, takes the colimit in $\Shv(I)$, and applies the left adjoint.

\end{proof}

Let $\bD$ denote the simplicial indexing category, and denote its objects by $[n]$ for $n\in\NN$.

\begin{definition}

A {\em fuzzy simplicial set} is a functor $\bD\op\to\Fuz$.  A {\em morphism of fuzzy simplicial sets} is a natural transformation of functors.   The category of fuzzy simplicial sets is denoted $\sFuz$.

\end{definition}

A fuzzy simplicial set is a simplicial set in which every simplex has a strength.  A simplex has strength at most the minimum of its faces.  All degeneracies of a simplex have the same strength as the simplex.

A fuzzy simplicial set $X\taking\bD\op\to\Fuz$ can be rewritten as a sheaf $X\taking(\bD\cross I)\op\to\Sets$, where $\bD$ has the trivial Grothendieck topology and $\bD\cross I$ has the product Grothendieck topology.  We write $X_n^{<a}$ to denote the set $X([n],[0,a)).$

For $n\in\NN$ and $i\in I$, let $\Delta^n_i\in\sFuz$ denote the functor represented by $(n,i)$.  If $i=[0,a)$  we may also write $\Delta^n_{<a}$ to denote $\Delta^n_i$.  Note that a map $f\taking[n]\to[m]$ induces a unique map $F\taking\Delta^n_{<a}\to\Delta^m_{<b}$ if and only if $a\leq b$; otherwise there can be no such $F$.

Any fuzzy simplicial set $X$ can be canonically written as the colimit of its diagram of simplices: $$\colim_{\Delta^n_{<a}\to X}\Delta^n_{<a}\Too{\iso} X$$


\section{uber-metric spaces}

We define a category of uber-metric spaces, which are metric spaces except with the possibility of $d(x,y)=\infty$ or $d(x,y)=0$ for $x\neq y$.

\begin{definition}

An {\em uber-metric space} is a pair $(X,d)$, where $X$ is a set and $d\taking X\cross X\to [0,\infty]$, such that for all $x,y,z\in X$, \begin{enumerate}\item $d(x,x)=0$,\item $d(x,y)=d(y,x)$, and\item $d(x,z)\leq d(x,y)+d(y,z)$.\end{enumerate}  Here we consider $x\leq\infty$ and $x+\infty=\infty+x=\infty$ for all $x\in [0,\infty]$.  We call $d$ an {\em uber-metric} or just a {\em metric} on $X$.

A {\em morphism of uber-metric spaces}, denoted $f\taking (X,d_X)\to (Y,d_Y)$ is a function $f\taking X\to Y$ such that $d_Y(f(x_1),f(x_2))\leq d_X(x_1,x_2)$ for all $x_1,x_2\in X$.  Such functions are also called {\em non-expansive}.

These objects and morphisms define a category called {\em the category of uber-metric spaces} and denoted $\UM$.

\end{definition}

\begin{lemma}

The category $\UM$ is closed under colimits.

\end{lemma}

\begin{proof}

We must show that $\UM$ has an initial object, arbitrary coproducts, and coequalizers.  The set $\emptyset$ is the initial object in $\UM$.  

Let $A$ be a set and for all $a\in A$, let $(X_a,d_a)$ denote a metric space.  Let $X_A$ denote the set $\coprod_{a\in A}X_a$; and let $d_A$ denote the metric such that for all $y,y'\in X_A$, if there exists $a\in A$ such that $y,y'\in X_a$ then $d_A(y,y')=d_a(y,y')$, but if instead $y$ and $y'$ are in separate components then $d_A(y,y')=\infty$.  One checks that $(X_A,d_A)$ is an uber-metric space and that it satisfies the universal property for a coproduct.

Finally, suppose that $$\xymatrix{A\ar@<.5ex>[r]^f\ar@<-.5ex>[r]_g& X\ar[r]^{[-]}& Y}$$ is a coequalizer diagram of sets.  Write $x\sim x'$ if there exists $a\in A$ with $x=f(a), y=g(a)$; then $Y=X/\sim$ is the set of equivalence classes.  If $y=m(x)$.   If $d_X$ is a metric on $X$, we define a metric (\cite{Wik}) $d_Y$ on $Y$ by $$d_Y([x],[x'])=\inf(d_X(p_1,q_1)+d_X(p_2,q_2)+\cdots+d_X(p_n,q_n)),$$ where the infemum is taken over all pairs of sequences $(p_1,\ldots,p_n), (q_1,\ldots,q_n)$ of elements of $X$, such that $p_1\sim x$, $q_n\sim x'$, and $p_{i+1}\sim q_i$ for all $1\leq i\leq n-1$.  Again, one checks that $(Y,d_Y)$ is an uber-metric space which satisfies the universal property of a coequalizer.

\end{proof}

\section{Metric realization}

In order to define a metric realization functor $Re\taking\sFuz\to\UM$, we first define it on the representable sheaves in $\sFuz$ and then extend to the whole category using colimits (i.e. using a left Kan extention).  

Recall the usual metric on Euclidean space $\RR^m$ and let $\RR^m_{\geq 0}$ denote the $m$-tuples all of whose entries are non-negative.  Recall also that objects of $I$ are of the form $[0,a)$ for $0<a\leq 1$.  For an object $([n],[0,a))\in \NN\cross I$, define $Re(\Delta^n_{<a})$, as a set, to be $$\{(x_0,x_1,\ldots,x_n)\subset\RR^{n+1}|x_0+x_1+\cdots+x_n=1-a\}$$  We take as our metric on $Re(\Delta^n_{<a})$ to be that induced by its inclusion as a subspace of $\RR^{n+1}$. 

A morphism $([n],[0,a))\to ([m],[0,b))$ exists if $a\leq b$, and in that case consists of a morphism $\sigma\taking [n]\to[m]$.  We define $Re(\sigma,a\leq b)\taking Re(\Delta^n_{<a})\to Re(\Delta^m_{<b})$ to be the map $$(x_0,x_1,\ldots,x_n)\mapsto \frac{1-b}{1-a}\left(\sum_{i_0\in\sigma^\m1(0)}x_{i_0},\sum_{i_1\in\sigma^\m1(1)}x_{i_1},\ldots,\sum_{i_m\in\sigma^\m1(m)}x_{i_m}\right).$$  Note that this map is non-expansive because $1-b\leq 1-a$.

We are ready to define $Re$ on a general $X$ as $$Re(X):=\colim_{\Delta^n_{<a}\to X}Re(\Delta^n_{<a}).$$  This functor preserves colimits, so it has a right adjoint, which we denote $Sing\taking\UM\to\sFuz$.  It is given on $Y\in\UM$ by $$Sing(Y)_n^{<a}=\Hom_\UM(Re(\Delta^n_{<a}),Y).$$

\bibliographystyle{amsalpha}
\begin{thebibliography}{JTT}

\bibitem[Gog]{Gog}Goguen.

\bibitem[Bar]{Bar}Barr. Fuzzy sets form a topos.

\bibitem[Isb]{Isb}Isbell.  Category of metric spaces.

\bibitem[Wik]{Wik}Wikipedia, Metric space, http://en.wikipedia.org/wiki/Metric\_space (Quotient metric spaces) (as of Sept. 18, 2009, 17:58 GMT).

\end{thebibliography}

\end{document}