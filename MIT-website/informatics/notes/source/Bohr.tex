\documentclass{amsart}

\begin{document}

\author{David I. Spivak}

\title{A Bohr's Model for Information}

\thanks{This project was supported in part by the Office of Naval Research.}

\maketitle

Our senses informed us that there was something important, something all around us, something obvious and present, something which we now call matter.  In order to manipulate matter more effectively, we began to study its behavior.  We wanted to understand how it worked, how it fit together and came apart, and in our attempt to do so, we looked to smaller and smaller pieces, assuming the existence of an un-cuttable piece -- an atom.  If only we could fully understand atoms and how they interact, we could understand the world of matter.

We still do not understand atoms nor how they come together to create the world of matter, but we understand them better than we did.  One of the crucial steps toward that understanding was the Bohr model of the atom: one nucleus surrounded by electrons in orbital shells.  This model is now considered primitive, but it was highly useful anyway because it provided a common language that was well-suited for discussing the salient properties of the atom.

As humans, we perceive things other than matter.  For example, we perceive light.  Our attempt to understand light has been fairly confounding, because light exhibits properties of both a wave and a particle.  Light is not matter -- it is not composed of atoms or quarks in any way -- but it interacts with matter.  As another example, in our daily life we perceive forces such as gravity.  Gravity is not made of atoms or quarks but it interacts with them.  

The topic of this research is information.  Just as we perceive matter and light, we also perceive information.  We have the sensation of knowing, understanding, being informed, finding meaning in a statement.  We may not know what it is that causes these sensations, but it is clear that we perceive information and meaning in the world around us.  A certain alarm {\em means} it is time to wake up; a certain declaration on TV {\em means} that Barack Obama has won the presidency; a certain paper {\em means} that Poincare and Fermat were correct.  We perceive information.  We are surrounded by information and we know it when we see it.  As a phenomenon it may not be as physical as matter or light, but information floods our daily lives.  

What is this so-called ``information"?  Is it made of matter or light or force, or is it distinct from them?

Steven Hawking and John Preskill made a bet in 1997 as to whether something physicists call ``information" could ever be lost, in particular whether it could be lost upon entering a black hole.  Hawking conceded in 2004 that information was not obliterated in a black hole, only scrambled.  Is the information that these physicists are talking about the same as the information we mean here and in our daily lives?  Or is their terminology a reversal of the common event in which lay people usurp physics words that seem to fit, like ``a magnetic personality" or ``bad energy in the room"?  The type of information physicists discuss has the property that it can remain information even if it is scrambled, whereas to humans scrambled information is not informative unless we have access to an descrambler.

Whether or not physicists are discussing the same phenomenon or a different one, they are viewing information from the perspective of its interactions with atoms and other elementary particles.  In this work, we are concerned with how it presents itself to us as humans.  We wish to model the phenomena of receiving, processing, verifying, understanding, storing, manipulating, and transferring information.   

We do not need to descend to the most fundamental properties of an object to understand how it works in the world.  When we think of a circle, we do not think of constant Euclidean distance (as computed by the Pythagorean theorem) between points in $\mathbb{R}^2$ and a fixed point.  We do not descend further to think of these points in $\mathbb{R}^2$ as pairs of Dedikind cuts in the field of fractions of the ring obtained by freely adjoining additive inverses to the natural numbers, which are themselves isomorphism classes of finite sets under disjoint union and Cartesian product.  No.  When I speak of a circle, you see it as it is in the world -- as a tire, as defining the ratio $\pi$, as it relates to sine and cosine, as it is used in the effort to classify topological spaces up to homotopy.  We talk about a circle without having to consider the extreme complexity of its construction.  In fact, the circle presents itself to us by its properties and behavior, and we humans create mathematics to understand it.

Similarly, information on a subject presents itself to us in myriad ways.  We can view, measure, and conceptualize the same object from countless perspectives.  The way it presents itself to me is different than the way it presents itself to you.  And yet if it is the same object, then there must be ways to translate aspects of my conception into aspects of yours and vice versa.  Our goal as like beings is to find a way that we can agree on the structure and behavior of the objects.  This is not always easy: people talk past each other, experts disagree on the subject of their expertise, the student does not always grasp the obvious.  

So what is information?  When two people discuss a topic, what are they exchanging?  It is not mere words or air vibrations: something is being communicated, which can change the structure of the human 
brain in a deliberate way!  

The study of information is called informatics.  Our goal is to find a firm foundation for the theory of informatics, within the field of mathematics.  The language of mathematics, and category theory in particular, is well suited to the problem of finding firm foundations for a given theory.  Finding a good model, even if it eventually proves primitive, is essential.








\end{document}