\documentclass{amsart}

\usepackage{amssymb, amscd,stmaryrd,setspace,hyperref,color}

\input xy
\xyoption{all} \xyoption{poly} \xyoption{knot}\xyoption{curve}

\newcommand{\comment}[1]{}

\newcommand{\longnote}[2][4.9in]{\fcolorbox{black}{yellow}{\parbox{#1}{\color{black} #2}}}
\newcommand{\note}[1]{\fcolorbox{black}{yellow}{\color{black} #1}}
\newcommand{\q}[1]{\begin{question}#1\end{question}}
\newcommand{\g}[1]{\begin{guess}#1\end{guess}}

\def\tn{\textnormal}
\def\mf{\mathfrak}
\def\mc{\mathcal}

\def\ZZ{{\mathbb Z}}
\def\QQ{{\mathbb Q}}
\def\RR{{\mathbb R}}
\def\CC{{\mathbb C}}
\def\AA{{\mathbb A}}
\def\PP{{\mathbb P}}
\def\NN{{\mathbb N}}

\def\bD{{\bf \Delta}}
\def\Str{{\bf Str}}

\def\Hom{\tn{Hom}}
\def\Fun{\tn{Fun}}
\def\Nat{\tn{Nat}}
\def\Ob{\tn{Ob}}
\def\Op{\tn{Op}}

\def\to{\rightarrow}
\def\from{\leftarrow}
\def\cross{\times}
\def\taking{\colon}
\def\inj{\hookrightarrow}
\def\surj{\twoheadrightarrow}
\def\too{\longrightarrow}
\def\fromm{\longleftarrow}
\def\down{\downarrow}
\def\Down{\Downarrow}
\def\Up{\Uparrow}
\def\tooo{\longlongrightarrow}
\def\tto{\rightrightarrows}
\def\ttto{\equiv\!\!>}
\def\ss{\subset}
\def\superset{\supset}
\def\iso{\cong}
\def\|{{\;|\;}}
\def\m1{{-1}}
\def\op{^\tn{op}}
\def\loc{\tn{loc}}
\def\la{\langle}
\def\ra{\rangle}
\def\wt{\widetilde}
\def\wh{\widehat}
\def\we{\simeq}
\def\ol{\overline}
\def\ul{\underline}
\def\qeq{\mathop{=}^?}
\def\setto{\colon\hspace{-.25cm}=}

\def\ullimit{\ar@{}[rd]|(.3)*+{\lrcorner}}
\def\urlimit{\ar@{}[ld]|(.3)*+{\llcorner}}
\def\lllimit{\ar@{}[ru]|(.3)*+{\urcorner}}
\def\lrlimit{\ar@{}[lu]|(.3)*+{\ulcorner}}
\def\ulhlimit{\ar@{}[rd]|(.3)*+{\diamond}}
\def\urhlimit{\ar@{}[ld]|(.3)*+{\diamond}}
\def\llhlimit{\ar@{}[ru]|(.3)*+{\diamond}}
\def\lrhlimit{\ar@{}[lu]|(.3)*+{\diamond}}
\newcommand{\clabel}[1]{\ar@{}[rd]|(.5)*+{#1}}

\newcommand{\arr}[1]{\ar@<.5ex>[#1]\ar@<-.5ex>[#1]}
\newcommand{\arrr}[1]{\ar@<.7ex>[#1]\ar@<0ex>[#1]\ar@<-.7ex>[#1]}
\newcommand{\arrrr}[1]{\ar@<.9ex>[#1]\ar@<.3ex>[#1]\ar@<-.3ex>[#1]\ar@<-.9ex>[#1]}
\newcommand{\arrrrr}[1]{\ar@<1ex>[#1]\ar@<.5ex>[#1]\ar[#1]\ar@<-.5ex>[#1]\ar@<-1ex>[#1]}

\newcommand{\To}[1]{\xrightarrow{#1}}
\newcommand{\Too}[1]{\xrightarrow{\ \ #1\ \ }}
\newcommand{\From}[1]{\xleftarrow{#1}}


\newcommand{\Adjoint}[4]{\xymatrix@1{#2 \ar@<.5ex>[r]^-{#1} & #3 \ar@<.5ex>[l]^-{#4}}}
\newcommand{\adjoint}[4]{\xymatrix@1{#1\colon #2\ar@<.5ex>[r]& #3\;:#4 \ar@<.5ex>[l]}}

\def\id{\tn{id}}
\def\Top{{\bf Top}}
\def\Cat{{\bf Cat}}
\def\Sets{{\bf Sets}}
\def\sSets{{\bf sSets}}
\def\Grpd{{\bf Grpd}}
\def\Pre{{\bf Pre}}
\def\Shv{{\bf Shv}}
\def\Rings{{\bf Rings}}

\def\colim{\mathop{\tn{colim}}}

\def\mcA{\mc{A}}
\def\mcB{\mc{B}}
\def\mcC{\mc{C}}
\def\mcD{\mc{D}}
\def\mcE{\mc{E}}
\def\mcF{\mc{F}}
\def\mcG{\mc{G}}
\def\mcH{\mc{H}}
\def\mcI{\mc{I}}
\def\mcJ{\mc{J}}
\def\mcK{\mc{K}}
\def\mcL{\mc{L}}
\def\mcM{\mc{M}}
\def\mcN{\mc{N}}
\def\mcO{\mc{O}}
\def\mcP{\mc{P}}
\def\mcQ{\mc{Q}}
\def\mcR{\mc{R}}
\def\mcS{\mc{S}}
\def\mcT{\mc{T}}
\def\mcU{\mc{U}}
\def\mcV{\mc{V}}
\def\mcW{\mc{W}}
\def\mcX{\mc{X}}
\def\mcY{\mc{Y}}
\def\mcZ{\mc{Z}}

\newtheorem{theorem}{Theorem}[subsection]
\newtheorem{lemma}[theorem]{Lemma}
\newtheorem{proposition}[theorem]{Proposition}
\newtheorem{corollary}[theorem]{Corollary}
\newtheorem{fact}[theorem]{Fact}

\theoremstyle{remark}
\newtheorem{remark}[theorem]{Remark}
\newtheorem{example}[theorem]{Example}
\newtheorem{warning}[theorem]{Warning}
\newtheorem{question}[theorem]{Question}
\newtheorem{guess}[theorem]{Guess}
\newtheorem{answer}[theorem]{Answer}
\newtheorem{construction}[theorem]{Construction}

\theoremstyle{definition}
\newtheorem{definition}[theorem]{Definition}
\newtheorem{notation}[theorem]{Notation}
\newtheorem{conjecture}[theorem]{Conjecture}
\newtheorem{postulate}[theorem]{Postulate}

\def\DT{{\bf DT}}
\def\GD{{\bf GD}}
\def\DB{\GD}
\def\Sch{{\bf Sch}}
\def\Null{{\bf Null}}
\def\Strings{{\bf Strings}}
\def\ND{{\bf ND}}
\def\Tables{{\bf Tables}}
\def\'{\tn{'}}
\def\disunion{\amalg}
\def\Rel{{\bf Rel}}
\def\mcRel{{\bf \mcR el}}
\def\Cech{$\check{\tn{C}}$ech }
\def\C{\check{\tn{C}}}
\def\Fin{{\bf Fin}}
\def\singleton{{\{*\}}}
\def\Sub{{\bf Sub}}
\def\card{\tn{card}}
\def\Data{{\bf DB}}
\def\DB{{\bf DB}}
\def\im{\tn{im}}
\def\'{\tn{'}}
\def\start{\note{start here}}





\usepackage{graphicx}

%\oddsidemargin .5in 
%\evensidemargin .5in 
%textwidth 5.3in 

\def\Loc{{\bf Loc}}
\def\bigjoin{\bigvee}
\def\bigmeet{\bigwedge}
\def\join{\vee}
\def\meet{\wedge}

%\usepackage{showkeys}

\begin{document}

\author{David I. Spivak}

\thanks{This project was supported in part by the Office of Naval Research.}

\title{Localic databases}

\maketitle

\section{Locales}

\subsection{Morphisms between locales of subobjects}

Suppose that $f\taking X\to Y$ is a morphism of simplicial sets.  Let $\mcX=\Sub(X)$ (respectively $\mcY=\Sub(Y)$) denote the locale of subobjects of $X$ (resp. $Y$).  The map $f$ induces an adjunction $$\adjoint{f^\m1}{\mcY}{\mcX}{\mcF}$$ where $\mcF(X')=\bigjoin(Y'\ss Y|f^\m1(Y')\ss X')$.  Note that $f^\m1$ preserves finite meets, so $\mcF\taking\mcX\to\mcY$ denotes a morphism of locales.  This morphism further induces an adjunction between Grothendieck topoi $$\adjoint{\mcF^*}{\Shv(\mcY)}{\Shv(\mcX)}{\mcF_*}$$ defined as follows for sheaves $A\in\Shv(\mcX)$ and $B\in\Shv(\mcY)$.  For any $U\in\Sub(X)$ we take \begin{eqnarray}\label{dia:mcF^*}\mcF^*B(U):=B(f(U)),\end{eqnarray} where $f(U)\in\Sub(Y)$ is the image of $U$ in $Y$.  For any $V\in\Sub(Y)$ we take \begin{eqnarray}\label{dia:mcF_*}\mcF_*A(V):=A(f^\m1(V)),\end{eqnarray} where $f^\m1(V)$ is the preimage of $V$ in $X$.

\subsection{Morphisms between locales}

We introduced the above ideas using a map of simplicial sets $f\taking X\to Y$, but in actuality the definitions of $\mcF^*$ and $\mcF_*$ work whenever $\mcF$ is a {\em open morphism of locales}, as we shall soon see.  To get to that point, however, it is best to start all over, by taking $\mcX$ and $\mcY$ to be arbitrary locales and considering an arbitrary adjunction $$\adjoint{f^*}{\mcY}{\mcX}{f_*}$$ as a morphism $\mcF=f_*$ of locales.  The left adjoint $f^*$ induces an adjunction of Grothendieck topoi denoted $$\adjoint{\mcF^*}{\Shv(\mcY)}{\Shv(\mcX)}{\mcF_*}$$ by $\mcF_*$, which is given by the formula \begin{align}\label{dia:again mcF^*} \mcF_*(A)(V):=A(f^*V).\end{align}  The left adjoint $\mcF^*$ is easy to understand on representable functors, but in general is just understandable as a colimit.

We say that $\mcF$ is {\em an open morphism} if $f^*$ has a left adjoint $f_!\taking\mcX\to\mcY$ satisfying the ``Frobenius identity" $$f_!(x\meet f^*y)=f_!(x)\meet y$$ for all $x\in\mcX$ and $y\in\mcY$.  In this case, we can do a better job of understanding $\mcF^*$.   Namely, for a sheaf $B\in\Shv(\mcY)$ and object $U\in\mcX$, we define \begin{align}\label{dia:open}\mcF^*B(U):=B(f_!(U))\end{align} and still 

To bring it together, a morphism $f\taking X\to Y$ of simplicial sets always induces an open morphism of locales.  Thus, Equations (\ref{dia:mcF^*}) and (\ref{dia:mcF_*}) becomes Equations (\ref{dia:open}) and (\ref{dia:again mcF^*}).


\section{Elementary localic databases}

Let $\Loc$ denote the category of locales (see \cite{Bor3}).  Recall that a morphism between locales is an adjunction $(f^*,f_*)$ for which $f^*$ preserves finite meets; such a morphism is denoted by the right adjoint part.

\begin{definition}

A {\em type specification} consists of a pair $\mcD=(D,\Gamma)$, where $D$ is a category and $\Gamma\in\Pre(D)$ is a presheaf on $D$.  A {\em morphism of type specifications}, denoted $$(F,F^\sharp)\taking\mcD\to\mcD',$$ consists of a functor $F\taking D'\to D$ and a map of presheaves $F^\sharp\taking F^*(\Gamma)\to\Gamma'$ on $D'$.  

\end{definition}

\begin{example}

Let $D=(\Fin\down\{m,a\})$ and $D'=(\Fin\down\{F\})$ be categories.  Given $\sigma\taking C\to\{m,a\}$, let $\Gamma(\sigma)=\Hom(C,\RR)$, and similarly given $\sigma'\taking C'\to\{F\}$, let $\Gamma'(\sigma')=\Hom(C',\RR)$.  There is a morphism of type specifications $$(\mu,\mu^\sharp)\taking (D,\Gamma)\to(D',\Gamma')$$ given as follows.  

First off, the functor $\mu\taking D'\to D$ sends $\sigma'\taking C'\to\{F\}$, to $\mu(\sigma'):=C'\cross\{m,a\}\to\{m,a\}$.  This induces a functor $$\mu^*\taking\Pre(D)\to\Pre(D');\hspace{.5in}X(-)\mapsto X(\mu(-)).$$

Now, let $\mu^\sharp\taking\mu^*\Gamma\to\Gamma'$ denote the map of presheaves on $D'$ defined as follows for $\sigma\taking C'\to\{F\}$.  We calculate that $\mu^*\Gamma(\sigma)=\Gamma(\mu(\sigma))=(\RR\cross\RR)^{C'}$ and that $\Gamma'(C')=\RR^{C'}$.  Now define $$\mu^\sharp\taking(\RR\cross\RR)^{C'}\to\RR^{C'}$$ by composing with the usual multiplication map $\RR\cross\RR\to\RR$.  

This change of type specifications allows one to turn any ``mass/accelaration" table into a ``Force" table.  We write it as $F=ma$.

More generally, suppose that $x_1,\ldots,x_m$ are the variables of $D$, taking values in $\Gamma(x_1),\ldots,\Gamma(x_m)$ and $X_1,\ldots,X_n$ are the ``variables" of $D'$ taking values in $\Gamma(X_1),\ldots,\Gamma(X_n)$, then we can define $X_i$ in terms of any set of $x_1,\ldots x_m$, using any formula.  Writing down the variables involved in these formulas constitutes a map $$D'=(\Fin\down\{X_1,\ldots,X_n\})\too D=(\Fin\down\{x_1,\ldots,x_m\}).$$

\end{example}

\begin{definition}

Let $\mcD=(D,\Gamma)$ denote a type specification.  A {\em schema on $\mcD$} consists of a pair $(L,\sigma)$, where $L$ is a locale and $\sigma\taking L\to\Pre(D)$ preserves colimits and finite limits.  A {\em morphism of schemas} is denoted $$\mcF=(f_*,f^\sharp)\taking (L,\sigma)\to(L',\sigma'),$$ where $\adjoint{f^*}{L'}{L}{f_*}$ is a morphism of locales, and $f^\sharp\taking \sigma\circ f^*\to \sigma'$ is a natural transformation of functors $L'\to\Pre(D)$.

\end{definition}

\begin{definition}

Let $\mcD=(D,\Gamma)$ denote a type specification, and let $X=(L,\sigma)$ denote a schema over $\mcD$.  The {\em universal sheaf on $X$}, denoted $\mcU_X\taking L\op\to\Sets$, is the functor which assigns $$\ell\mapsto\Hom_{\Pre(D)}(\sigma(\ell),\Gamma) \hspace{.5in} \tn{ for } \ell\in L.$$   It is clearly contravariant and takes joins in $L$ to limits in $\Sets$.  

Let $X'=(L',\sigma')$.  Given a morphism $\mcF=(f,f^\sharp)\taking X\to X'$ of schemas, there is an induced morphism $\mcU_\mcF\taking\mcU_{X'}\to\mcF_*(\mcU_X)$ given as follows for $\ell'\in L'$: $$\mcU_{L'}(\ell')=\Hom_{\Pre(D)}(\sigma'\ell',\Gamma)\To{f^\sharp}\Hom_{\Pre(D)}(\sigma f^*\ell',\Gamma)=\mcF_*(\mcU_L)(\ell').$$

\end{definition}

\begin{example}

Let $\mcD=(D,\Gamma)$ and $\mcD'=(D',\Gamma')$ denote type specifications, and let $(F,F^\sharp)\taking\mcD\to\mcD'$ denote a morphism of type specifications.  Recall that $F\taking D'\to D$ is a functor and $F^\sharp\taking F^*\Gamma\to\Gamma'$ is a morphism of presheaves on $D'$.

Let $X=(L,\sigma)$ denote a schema over $\mcD$, where $\sigma\taking L\to\Pre(D)$.  Given $\ell\in L$, the value of the universal sheaf $\mcU_X$ at $\ell$ is $$\mcU_X(\ell)=\Hom_{\Pre(D)}(\sigma(\ell),\Gamma).$$  If we apply the left adjoint $F^*$ to both sides and then compose with $F^\sharp$ we obtain a map \begin{align*}\mcU_X(\ell)=\Hom(\sigma(\ell),\Gamma)&\to\Hom(F^*\sigma(\ell),F^*\Gamma)\\&\to\Hom(F^*\sigma(\ell),\Gamma')\end{align*}  This may be seen as how databases on $\mcD$ become databases on $\mcD'$.

\end{example}

\begin{definition}

Let $\mcD=(D,\Gamma)$ denote a type specification.  A {\em database of type $\mcD$} is a sequence $(X,\mcK_X,\tau_X)$, where $X=(L,\sigma)$ is a schema, $\mcK_X\in\Shv(L)$ is a sheaf of sets on $L$ and $\tau_X\taking\mcK_X\to\mcU_X$ is a morphism of sheaves over the universal sheaf $\mcU_X$.

A {\em morphism of databases} consists of a pair $$(\mcF,\mcF^\sharp)\taking(X,\mcK_X,\tau_X)\too(Y,\mcK_Y,\tau_Y),$$ where $\mcF\taking X\to X'$ is a morphism of schema and $\mcF^\sharp\taking \mcK_Y\to\mcF_*\mcK_X$ is a morphism of sheaves on $Y$, such that the diagram $$\xymatrix{\mcK_Y\ar[r]^{\tau_Y}\ar[d]_{\mcF^\sharp}&\mcU_Y\ar[d]_{\mcU_\mcF}\\\mcF_*\mcK_X\ar[r]_{\mcF_*\tau_X}&\mcF_*\mcU_X}$$ commutes.

\end{definition}

\section{Theory}

\longnote{There may be issues with the variance of locales.  That is, we may be naming by left adjoints instead of right adjoints.  This calls a lot of the following into question.  For now, we are just making a first pass through the space.}

Let $L$ be a locale.  For an object $x\in L$, we define $i_x\taking\down x\inj L$ to be the sublocale with objects $\{y\in L|y\leq x\}$. 


\begin{definition}

Suppose given a diagram $$\xymatrix@=13pt{L\ar[dr]_{\sigma_L}&&M\ar[dl]^{\sigma_M}\\&\mcC}$$ where $L$ and $M$ are locales and $\mcC$ is a category.  Define $\Gamma^L_M\in\Shv(L)$ to be the sheaf $$\Gamma^L_M(x):=\{f\taking\down x\to M|\sigma_M\circ f=\sigma_L\circ i_x\}.$$

For locales $L,M$ (in absence of $\mcC,\sigma_L,\sigma_M$) define $\Gamma^L_M\in\Shv(L)$ by $\Gamma^L_M(x):=\{f\taking\down x\to M\}$.  Clearly, this is the same thing as one gets by taking $\mcC=\{*\}$ to be the terminal category and $\sigma_L,\sigma_M$ to be the unique functors.

Given a map $\pi\taking M\to L$, let $\Gamma_\pi$ be as defined by the diagram $$\xymatrix@=13pt{L\ar[dr]_{\id_L}&&M\ar[dl]^\pi\\&L}$$

\end{definition}

\begin{definition}

Suppose given a diagram $$\xymatrix@=13pt{L\ar[dr]_{\sigma_L}&&M\ar[dl]^{\sigma_M}\\&\mcC}$$ where $L$ and $M$ are locales and $\mcC$ is a category.  Define $\mcU^L_M\in\Shv(L)$ to be the sheaf whose value on $x\in L$ is given by $$\mcU^L_M(x):=\{(f,f^\sharp)|f\taking\down x\to M, f^\sharp\taking\sigma_L(x)\to\sigma_M\circ f(x).\}$$

\end{definition}

Let $D$ be a locale and $\Gamma$ a presheaf on $D$.  Given a functor $\sigma\taking L\to\Pre(D)$, where $L$ is a locale, we have a diagram $$\xymatrix@=13pt{L\ar[dr]_{\sigma}&&\{*\}\ar[dl]^{\Gamma}\\&\Pre(D)}$$ so we can define $\mcU^L_\Gamma\in\Shv(L)$ as above.  Tracing through definitions, this sheaf is simply given on $x\in L$ by $$\mcU^L_\Gamma(x)=\Hom_{\Pre(\mcD)}(\sigma(x),\Gamma),$$ as expected.

\begin{definition}

Let $\mcC$ denote a category.  A {\em schema on $\mcC$} is a left exact left adjoint $\Gamma\taking M\to\mcC$, where $M$ is a locale.

\end{definition}

\begin{definition}

Let $\Gamma\taking M\to\mcC$ denote a schema on $\mcC$.  A {\em database on $\Gamma$} consists of a sequence $(\sigma, \mcK,\tau)$, where $\sigma\taking L\to\mcC$ is a schema, $\mcK\in\Shv(L)$ is a sheaf on $L$, and $\tau\taking\mcK\to\mcU^\sigma_\Gamma$ is a morphism of sheaves.  

\end{definition}

Let $\sigma_L\taking L\to\mcC$ denote a schema and let $\mcK_L\in\Shv(L)$ denote a sheaf.  This data constitutes a schema $(\sigma_L\cross\mcK_L)\taking L\to(\mcC\cross\Sets\op)$.  



\section{Next steps..}

\begin{itemize}

\item I'd like to study locales given by $\Sub(S)$ for a set $S$.  There should be an adjunction $\Adjoint{\Sub}{\Sets}{\Loc}{}$ which I can use.  Locales of the form $\Sub(S)$ may turn out to have a particularly easy ``grammar" in that the elements of $S$ serve well as ``variables."  There should be some kind of more general theory of variables, but I think it's easiest for $\Sub(S)$.
\item A computer program is data interpreted as instructions.  This should be phraseable in our language.  But what is instruction?

\end{itemize}

\bibliographystyle{amsalpha}
\bibliography{biblio}

\end{document}