\documentclass{amsart}

\usepackage{amssymb, amscd,stmaryrd,setspace,hyperref,color}

\input xy
\xyoption{all} \xyoption{poly} \xyoption{knot}\xyoption{curve}

\newcommand{\comment}[1]{}

\newcommand{\longnote}[2][4.9in]{\fcolorbox{black}{yellow}{\parbox{#1}{\color{black} #2}}}
\newcommand{\note}[1]{\fcolorbox{black}{yellow}{\color{black} #1}}
\newcommand{\q}[1]{\begin{question}#1\end{question}}
\newcommand{\g}[1]{\begin{guess}#1\end{guess}}

\def\tn{\textnormal}
\def\mf{\mathfrak}
\def\mc{\mathcal}

\def\ZZ{{\mathbb Z}}
\def\QQ{{\mathbb Q}}
\def\RR{{\mathbb R}}
\def\CC{{\mathbb C}}
\def\AA{{\mathbb A}}
\def\PP{{\mathbb P}}
\def\NN{{\mathbb N}}

\def\bD{{\bf \Delta}}
\def\Str{{\bf Str}}

\def\Hom{\tn{Hom}}
\def\Fun{\tn{Fun}}
\def\Nat{\tn{Nat}}
\def\Ob{\tn{Ob}}
\def\Op{\tn{Op}}

\def\to{\rightarrow}
\def\from{\leftarrow}
\def\cross{\times}
\def\taking{\colon}
\def\inj{\hookrightarrow}
\def\surj{\twoheadrightarrow}
\def\too{\longrightarrow}
\def\fromm{\longleftarrow}
\def\down{\downarrow}
\def\Down{\Downarrow}
\def\Up{\Uparrow}
\def\tooo{\longlongrightarrow}
\def\tto{\rightrightarrows}
\def\ttto{\equiv\!\!>}
\def\ss{\subset}
\def\superset{\supset}
\def\iso{\cong}
\def\|{{\;|\;}}
\def\m1{{-1}}
\def\op{^\tn{op}}
\def\loc{\tn{loc}}
\def\la{\langle}
\def\ra{\rangle}
\def\wt{\widetilde}
\def\wh{\widehat}
\def\we{\simeq}
\def\ol{\overline}
\def\ul{\underline}
\def\qeq{\mathop{=}^?}
\def\setto{\colon\hspace{-.25cm}=}

\def\ullimit{\ar@{}[rd]|(.3)*+{\lrcorner}}
\def\urlimit{\ar@{}[ld]|(.3)*+{\llcorner}}
\def\lllimit{\ar@{}[ru]|(.3)*+{\urcorner}}
\def\lrlimit{\ar@{}[lu]|(.3)*+{\ulcorner}}
\def\ulhlimit{\ar@{}[rd]|(.3)*+{\diamond}}
\def\urhlimit{\ar@{}[ld]|(.3)*+{\diamond}}
\def\llhlimit{\ar@{}[ru]|(.3)*+{\diamond}}
\def\lrhlimit{\ar@{}[lu]|(.3)*+{\diamond}}
\newcommand{\clabel}[1]{\ar@{}[rd]|(.5)*+{#1}}

\newcommand{\arr}[1]{\ar@<.5ex>[#1]\ar@<-.5ex>[#1]}
\newcommand{\arrr}[1]{\ar@<.7ex>[#1]\ar@<0ex>[#1]\ar@<-.7ex>[#1]}
\newcommand{\arrrr}[1]{\ar@<.9ex>[#1]\ar@<.3ex>[#1]\ar@<-.3ex>[#1]\ar@<-.9ex>[#1]}
\newcommand{\arrrrr}[1]{\ar@<1ex>[#1]\ar@<.5ex>[#1]\ar[#1]\ar@<-.5ex>[#1]\ar@<-1ex>[#1]}

\newcommand{\To}[1]{\xrightarrow{#1}}
\newcommand{\Too}[1]{\xrightarrow{\ \ #1\ \ }}
\newcommand{\From}[1]{\xleftarrow{#1}}


\newcommand{\Adjoint}[4]{\xymatrix@1{#2 \ar@<.5ex>[r]^-{#1} & #3 \ar@<.5ex>[l]^-{#4}}}
\newcommand{\adjoint}[4]{\xymatrix@1{#1\colon #2\ar@<.5ex>[r]& #3\;:#4 \ar@<.5ex>[l]}}

\def\id{\tn{id}}
\def\Top{{\bf Top}}
\def\Cat{{\bf Cat}}
\def\Sets{{\bf Sets}}
\def\sSets{{\bf sSets}}
\def\Grpd{{\bf Grpd}}
\def\Pre{{\bf Pre}}
\def\Shv{{\bf Shv}}
\def\Rings{{\bf Rings}}

\def\colim{\mathop{\tn{colim}}}

\def\mcA{\mc{A}}
\def\mcB{\mc{B}}
\def\mcC{\mc{C}}
\def\mcD{\mc{D}}
\def\mcE{\mc{E}}
\def\mcF{\mc{F}}
\def\mcG{\mc{G}}
\def\mcH{\mc{H}}
\def\mcI{\mc{I}}
\def\mcJ{\mc{J}}
\def\mcK{\mc{K}}
\def\mcL{\mc{L}}
\def\mcM{\mc{M}}
\def\mcN{\mc{N}}
\def\mcO{\mc{O}}
\def\mcP{\mc{P}}
\def\mcQ{\mc{Q}}
\def\mcR{\mc{R}}
\def\mcS{\mc{S}}
\def\mcT{\mc{T}}
\def\mcU{\mc{U}}
\def\mcV{\mc{V}}
\def\mcW{\mc{W}}
\def\mcX{\mc{X}}
\def\mcY{\mc{Y}}
\def\mcZ{\mc{Z}}

\newtheorem{theorem}{Theorem}[section]
\newtheorem{lemma}[theorem]{Lemma}
\newtheorem{proposition}[theorem]{Proposition}
\newtheorem{corollary}[theorem]{Corollary}
\newtheorem{fact}[theorem]{Fact}

\theoremstyle{remark}
\newtheorem{remark}[theorem]{Remark}
\newtheorem{example}[theorem]{Example}
\newtheorem{warning}[theorem]{Warning}
\newtheorem{question}[theorem]{Question}
\newtheorem{guess}[theorem]{Guess}
\newtheorem{answer}[theorem]{Answer}
\newtheorem{construction}[theorem]{Construction}

\theoremstyle{definition}
\newtheorem{definition}[theorem]{Definition}
\newtheorem{notation}[theorem]{Notation}
\newtheorem{conjecture}[theorem]{Conjecture}
\newtheorem{postulate}[theorem]{Postulate}

\def\DT{{\bf DT}}
\def\GD{{\bf GD}}
\def\DB{\GD}
\def\Sch{{\bf Sch}}
\def\Null{{\bf Null}}
\def\Strings{{\bf Strings}}
\def\ND{{\bf ND}}
\def\Tables{{\bf Tables}}
\def\'{\tn{'}}
\def\disunion{\amalg}
\def\Rel{{\bf Rel}}
\def\mcRel{{\bf \mcR el}}
\def\Cech{$\check{\tn{C}}$ech }
\def\C{\check{\tn{C}}}
\def\Fin{{\bf Fin}}
\def\singleton{{\{*\}}}
\def\Sub{{\bf Sub}}
\def\card{\tn{card}}
\def\Data{{\bf DB}}
\def\DB{{\bf DB}}
\def\im{\tn{im}}
\def\'{\tn{'}}
\def\start{\note{start here}}




\newcommand{\TriRight}[7]{\xymatrix{#1\ar[dr]_{#2}\ar[rr]^{#3}&&#4\ar[dl]^{#5}\\&#6\ar@{}[u] |{\Longrightarrow}\ar@{}[u]|>>>>{#7}}}
\newcommand{\TriLeft}[7]{\xymatrix{#1\ar[dr]_{#2}\ar[rr]^{#3}&&#4\ar[dl]^{#5}\\&#6\ar@{}[u] |{\Longleftarrow}\ar@{}[u]|>>>>{#7}}}
\def\FC{{\bf FC}}

\begin{document}

\author{David I. Spivak}

\thanks{This project was supported in part by the Office of Naval Research.}

\title{The box monads}

\maketitle

\section{Notation}

Let $\{*\}$ denote the terminal category (which has one arrow).  Let $\Cat$ denote the category of small categories (morphisms are functors).  Given a category $A\in\Cat$, let $\Ob(A)$ denote the set of objects in $A$.

Let $F\taking A\to C$ and $G\taking B\to C$ be functors.  We denote by $(F\down G)$ the category whose objects are pairs $(a,b,f)$ where $a\in\Ob(A), b\in\Ob(b)$ and $f\taking F(a)\to G(b)$ is a morphism in $C$, and whose morphism sets are given by $$\Hom_{(F\down G)}((a,b,f),(a',b',f'))=\{(x\taking a\to a', y\taking b\to b')| yf=f'x\}.$$

If $F\taking A\to\Cat$ is a functor and $B$ is a category, we sometimes write $(A\down\{B\})$ to denote $(F\down i_B)$, where $i_B\taking\{*\}\to\Cat$ is the category $B$.

If $C=\Cat$ we introduce a new category $(F\Down G)$ whose objects are the same as those in $(F\down G)$ but for which morphisms sets are given by $$\Hom_{(F\Down G)}((a,b,f),(a',b',f'))=\{(x,y,\alpha)|x\taking a\to a',y\taking b\to b',\alpha\taking yf\to f'x\}.$$  We write $(F\Up G)$ to denote the category with the direction of the natural transformation reversed: same objects, but morphism sets given by $$\Hom_{(F\Up G)}((a,b,f),(a',b',f'))=\{(x,y,\alpha)|x\taking a\to a',y\taking b\to b',\alpha\taking f'x\to yf\}.$$

Again, we may write $(F\Down\{B\})$ to denote $(F\Down i_B)$.  Explicitly, the objects in $(F\Down\{B\})$ are functors $F(a)\To{f} B$, where $a\in A$, and the morphisms $(F(a)\To{f} B)\too (F(a')\To{f'} B)$ are pairs $(x,\alpha)$ giving a natural transformation diagram $$\TriRight{F(a)}{f}{x}{F(a')}{f'}{B}{\alpha}$$

Given $F\taking A\to\Cat$, let $\bar{F}\taking A\to\Cat$ denote the functor $a\mapsto F(a)\op$.  Clearly, $(F\down\{B\})$ is isomorphic to $(\bar{F}\down\{B\op\})$.  The category $(\bar{F}\Down\{B\op\})$ can be identified with the category whose objects are functors $F(a)\to B$, where $a\in A$, and the morphisms $(F(a)\To{f}B)\too(F(a')\To{f'}B)$ are pairs $(x,\alpha)$ giving a natural transformation diagram $$\TriLeft{F(a)}{f}{x}{F(a')}{f'}{B}{\alpha}.$$  Clearly, this is the category $(F\Up\{B\})=(\bar{F}\Down\{B\op\})$.

\section{Grothendieck construction}

Let $f\taking \mcS\to\Cat$ be a functor.  The Grothendieck construction for $f$ is given by $$Gr(f):=(\{*\}\Down f).$$  Explicitly, $Gr(f)$ is the category with objects $(S,s)$ where $S\in\Ob(\mcS)$ and $s\in f(S)$.  The morphism sets in $Gr(f)$ are given by $$\Hom((S,s),(S',s'))=\{(y\taking S\to S',\alpha\taking f(y)(s)\to s')\}.$$

There is a natural projection $\pi_f\taking Gr(f)\to\mcS$ given by $(S,s)\mapsto S$.  It is a split fibration; that is, for any $y\taking S\to S'$ and $s\in S$, there is a canonical choice of map $(S,s)\to (S',f(y)(s))$ lying over $y$.

\begin{lemma}\label{gro pull square}

Let $f\taking\mcS\to\Cat$ be a diagram of categories.  A functor $\sigma\taking\mcC\to\mcS$ induces a diagram $$\xymatrix{Gr(f\sigma)\ullimit\ar[r]\ar[d]_{\pi_{f\sigma}^{~}}&Gr(f)\ar[d]^{\pi_f^{~}}\\\mcC\ar[r]_\sigma&\mcS,}$$ which is a pullback square.

\end{lemma}

\begin{proof}

***

\end{proof}

\begin{definition}

Let $\pi\taking\mcT\to\mcS$ be a functor.  A {\em section} of $\pi$ is a functor $g\taking\mcS\to\mcT$ such that $\pi g=\id_\mcS$.  A {\em morphism of sections} of $\pi$ is a natural transformation over $\mcS$.

\end{definition}

Recall that a pullback square as in Lemma \ref{gro pull square} defines a functor $$g\mapsto (\id_\mcC\cross_{\id_\mcS}(g\circ\sigma))$$ from the sections of $\pi_f\taking Gr(f)\to\mcS$ to the sections of $\pi_{f\sigma}\taking Gr(f\sigma)\to\mcC$.

\begin{lemma}\label{gro nat trans}

Suppose given two functors $f,g\taking\mcS\to\Cat$; let $\Nat(f,g)$ denote the set of natural transformations from $f$ to $g$.  There is a natural isomorphism $$\Hom_\mcS(Gr(f),Gr(g))\To{\iso}\Nat(f,g).$$ 

\end{lemma}

\begin{proof}

***

\end{proof}

\begin{definition}

There is a functor $\Gamma\taking(\Cat\Up\{\Cat\})\to\Cat$ defined as follows.  On objects, $\Gamma(f\taking\mcC\to\Cat)$ is defined to be the category of sections of the functor $\pi\taking Gr(f)\to\mcC$.  To define $\Gamma$ on morphisms, suppose we are given a natural transformation diagram $$\TriLeft{\mcC}{f}{x}{\mcD}{g}{\Cat}{\alpha}.$$  By Lemma \ref{gro pull square}, there is a canonical functor $\Gamma(g)\to\Gamma(gx)$.  By Lemma \ref{gro nat trans}, there is a canonical functor $\Gamma(gx)\to\Gamma(f)$ induced by $\alpha$.  The composition gives the desired functor.

\end{definition}

\begin{lemma}

Suppose that $\sigma_i\taking C_i\to\Cat$, for $i=1,2,3$, are three functors and suppose that $f\taking C_2\to C_1$ and $g\taking C_2\to C_3$ are functors too. There is an induced functor $C_4:=C_1\amalg_{C_2}C_3\To{\sigma_4}\Cat$.  Taking Grothendieck constructions yeilds the diagram $$\xymatrix@=10pt{&Gr(\sigma_1)\ar[ddd]\ar[drr]&&\\Gr(\sigma_2)\ar[ddd]\ar[drr]_{g'}\ar[ur]^{f'}&&&Gr(\sigma_4)\ar[ddd]\\&&Gr(\sigma_3)\ar[ur]\ar[ddd]&\\&C_1\ar[drr]&&\\C_2\ar[ur]^f\ar[drr]_g&&&C_4\\&&C_3\ar[ur]&}$$  All of the vertical squares are pullback squares by Lemma \ref{gro pull square}.

The natural map $$\Gamma(\sigma_4)\too\Gamma(\sigma_1)\cross_{\Gamma(\sigma_2)}\Gamma(\sigma_3)$$ is an isomorphism.

\end{lemma}

\section{Monad}

\begin{lemma}\label{monad lift}

Suppose that $\mcC$ is a category and $T$ is a monad on $\mcC$.  If $C\in\mcC$ is a $T$-algebra, then $T$ lifts to a monad on the category $(\mcC\down C)$.

\end{lemma}

\begin{proof}

Given $f\taking X\to C$ we define $T(f)$ to be the composition $T(X)\to T(C)\to C$.  It satisfies the axioms for a monad.

\end{proof}

\begin{remark}

The material below emphasizes the role of $\Cat$, but everything said (Lemma \ref{fc down is monad} and \ref{cat is alg for fc down}) also works for $\Sets$.  In particular, replace the category $\FC$ with the category $\Fin$ of finite sets, and replace the category $\Cat$ with the category $\Sets$.

\end{remark}

\begin{theorem}\label{fc down is monad}

Let $\FC$ denote the category of finite categories.  Consider the endofunctor $d:=(\FC\Down(-))\taking\Cat\to\Cat$.  There are natural transformations $\eta\taking\id_\Cat\to d$ and $\mu\taking d^2\to d$, which gives $d$ the structure of a monad on $\Cat$.  

\end{theorem}

\begin{proof}

For $C\in\Cat$, define $\eta_C\taking C\to dC$ to be the functor given on objects by $c\mapsto (\{*\}\To{c}C)$, and given on morphisms by $$(f\taking c\to c')\mapsto\left(\TriRight{\{*\}}{c}{\id_{\{*\}}}{\{*\}}{c'}{C}{f}\right).$$  This is clearly a natural transformation.

Defining $\mu$ is more difficult.  Let $C\in\Cat$ be a category; we will define $\mu_C\taking d^2C\to dC$.

Let $X$ be a finite category and let $\sigma\taking X\to(\FC\Down C)$ be an object in $d^2C$.  We need to establish some notation for $\sigma$.  On objects $x\in X$, denote $\sigma(x)$ by $$\sigma_x\taking S_x\to C,$$ where $S_x$ is a finite category.  On morphisms $f\taking x\to y$ in $X$, denote $\sigma(f)$ by $$(\sigma_f,\sigma_f^\sharp)\taking\left(\TriRight{S_x}{\sigma_x}{\sigma_f}{S_y}{\sigma_y}{C}{\sigma_f^\sharp}\right).$$

Define $\mu(X,\sigma)\in\Cat$ as the category with objects $$\Ob(\mu(X,\sigma)):=\{(x,s_x)|x\in X,s_x\in S_x\}$$ and morphisms $$\Hom_{\mu(X,\sigma)}((x,s_x),(y,s_y))=\{(f,f^\sharp)|f\taking x\to y, (f^\sharp\taking\sigma_f(s_x)\to s_y)\in S_y\}.$$  Note that since $X$ and each $S_x$ is finite, so is $\mu(X,\sigma)$.  

There is a canonical functor $\mu(X,\sigma)\to C$ given on objects by $(x,s_x)\mapsto\sigma_x(s_x)$ and given on morphisms by sending $(f,f^\sharp)\taking(x,s_x)\to(y,s_y)$ to the composite $$\sigma_x(s_x)\To{\sigma_f^\sharp}\sigma_y\sigma_f(s_x)\To{\sigma_y(f^\sharp)}\sigma_y(s_y).$$  Thus $\mu(X,\sigma)$ is indeed in $(\FC\Down C)$.

So far, we have only established $\mu$ on objects.  On morphisms $$\TriRight{X}{\sigma}{m}{X'}{\sigma'}{(\FC\Down C)}{m^\sharp},$$ define $\mu(m,m^\sharp)\taking\mu(X,\sigma)\to\mu(X',\sigma')$ as the functor given as follows.  Send the object $(x,s_x)\in\mu(X,\sigma)$ to $(m(x),m^\sharp_x(s_x))\in\mu(X',\sigma')$.  Send $(f,f^\sharp)\taking(x,s_x)\to(y,s_y)$ to $$(m(x),(m_y^\sharp\circ f^\sharp\taking\sigma'_{mf}(m_x^\sharp(s_x))\to m_y^\sharp(s_y)))\taking\mu(X,\sigma)\to\mu(X'\sigma').$$  There is an obvious natural transformation $$\TriRight{\mu(X,\sigma)}{}{}{\mu(X',\sigma')}{}{C}{}$$ because $m^\sharp\taking\sigma'\circ m\to \sigma$ was assumed natural.  

We draw the following diagram, in case it is useful to the reader: $$\xymatrix{S_x\ar@/^3pc/[rrd]^{\sigma_x}\ar[r]^{\sigma_f}\ar[d]_{m_x^\sharp}&S_y\ar[rd]^{\sigma_y}\ar[d]_{m_y^\sharp}&\ar@{}[l]|{\Leftarrow}_{\sigma_f^\sharp}\\S'_{mx}\ar[r]_{\sigma'_{mx}}\ar@/_2pc/[rr]_{\sigma'_{mx}}&S'_{my}\ar@{}[d]|{\Uparrow}_{\sigma_{mf}^\sharp}\ar[r]_{\sigma'_{my}}\ar@{}[ur]|<<<{\Down}&C\\&&&}$$

We forgo showing that this really is a monad.

\end{proof}

\begin{corollary}

Let $\FC$ denote the category of finite categories.  Consider the endofunctor $e:=(\FC\Up(-))\taking\Cat\to\Cat$.  There are natural transformations $\eta\taking\id_\Cat\to e$ and $\mu\taking e^2\to e$, which gives $e$ the structure of a monad on $\Cat$.  

\end{corollary}

\begin{proof}

Recall the endofunctor $d\taking\Cat\to\Cat$ given by $(\FC\Up(-))$ as in Theorem \ref{fc down is monad}.  For any category $C$ there is a isomorphism (natural in $C$) $$(\FC\Down C)\To{\iso}(\FC\Up C\op)\op$$ given by sending an object $\sigma\taking S\to C$ to the object $\sigma\op\taking S\op\to C\op$ and by sending a morphism $(\sigma_x\taking S_x\to C)\too (\sigma_y\taking S_y\to C)$, denoted $$(\sigma_f,\sigma_f^\sharp)\taking\left(\TriRight{S_x}{\sigma_x}{\sigma_f}{S_y}{\sigma_y}{C}{\sigma_f^\sharp}\right),$$ to the morphism  $$(\sigma_f,\sigma_f^\sharp)\taking\left(\TriLeft{S_x\op}{\sigma_x\op}{\sigma_f}{S_y\op}{\sigma_y\op}{C\op}{\sigma_f^\sharp}\right).$$

Now we can simply apply the maps $\mu$ and $\eta$ from Theorem \ref{fc down is monad}.  For $\mu$ we have \begin{align*}(\FC\Up(\FC\Up C))&\iso(\FC\Down(\FC\Up C)\op)\op\\&\iso(\FC\Down(\FC\Down C\op))\op\\&\To{\mu}(\FC\Down C\op)\op\iso (\FC\Up C)\end{align*} and it is natural in $C$.  Writing $op\taking\Cat\to\Cat$ to denote the involution, we also have $$\xymatrix{\Cat\ar[r]^{op}\ar@/^2pc/[rrr]^{(\FC\Up-)}&\Cat\ar[r]^{(\FC\Down-)}\ar@/_1pc/[r]_{\id_\Cat}&\Cat\ar[r]^{op}&\Cat}$$ so the natural transformation $\eta\taking\id_\Cat\to(\FC\Down-)$ extends to a natural transformation $\id_\Cat\to(\FC\Up -)$.  

All the monad diagrams commute for $e$ because they do for $d$.

\end{proof}

\begin{lemma}\label{cat is alg for fc down}

The global sections functor $\Gamma\taking(\FC\Down\Cat)\to\Cat$ gives $\Cat$ the structure of a $d$-algebra.  

Therefore, one can consider $d$ as a monad on $(\Cat\Down\{\Cat\})$.

\end{lemma}

\begin{proof}

The second assertion follows from the first by Lemma \ref{monad lift}.  The first requires proof....

\end{proof}



\end{document}