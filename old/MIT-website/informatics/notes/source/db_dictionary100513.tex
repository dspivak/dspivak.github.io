\documentclass{amsart}

\usepackage{amssymb, amscd,stmaryrd,setspace,hyperref,color}

\input xy
\xyoption{all} \xyoption{poly} \xyoption{knot}\xyoption{curve}
\input{diagxy}


\newcommand{\comment}[1]{}

\newcommand{\longnote}[2][4.9in]{\fcolorbox{black}{yellow}{\parbox{#1}{\color{black} #2}}}
\newcommand{\note}[1]{\fcolorbox{black}{yellow}{\color{black} #1}}
\newcommand{\q}[1]{\begin{question}#1\end{question}}
\newcommand{\g}[1]{\begin{guess}#1\end{guess}}

\def\tn{\textnormal}
\def\mf{\mathfrak}
\def\mc{\mathcal}

\def\ZZ{{\mathbb Z}}
\def\QQ{{\mathbb Q}}
\def\RR{{\mathbb R}}
\def\CC{{\mathbb C}}
\def\AA{{\mathbb A}}
\def\PP{{\mathbb P}}
\def\NN{{\mathbb N}}

\def\Hom{\tn{Hom}}
\def\Fun{\tn{Fun}}
\def\Ob{\tn{Ob}}
\def\Op{\tn{Op}}

\def\to{\rightarrow}
\def\from{\leftarrow}
\def\cross{\times}
\def\taking{\colon}
\def\inj{\hookrightarrow}
\def\surj{\twoheadrightarrow}
\def\too{\longrightarrow}
\def\tooo{\longlongrightarrow}
\def\tto{\rightrightarrows}
\def\ttto{\equiv\!\!>}
\def\ss{\subset}
\def\superset{\supset}
\def\iso{\cong}
\def\down{\downarrow}
\def\|{{\;|\;}}
\def\m1{{-1}}
\def\op{^\tn{op}}
\def\loc{\tn{loc}}
\def\la{\langle}
\def\ra{\rangle}
\def\wt{\widetilde}
\def\wh{\widehat}
\def\we{\simeq}
\def\ol{\overline}
\def\ul{\underline}
\def\qeq{\mathop{=}^?}

\def\ullimit{\ar@{}[rd]|(.3)*+{\lrcorner}}
\def\urlimit{\ar@{}[ld]|(.3)*+{\llcorner}}
\def\lllimit{\ar@{}[ru]|(.3)*+{\urcorner}}
\def\lrlimit{\ar@{}[lu]|(.3)*+{\ulcorner}}
\def\ulhlimit{\ar@{}[rd]|(.3)*+{\diamond}}
\def\urhlimit{\ar@{}[ld]|(.3)*+{\diamond}}
\def\llhlimit{\ar@{}[ru]|(.3)*+{\diamond}}
\def\lrhlimit{\ar@{}[lu]|(.3)*+{\diamond}}
\newcommand{\clabel}[1]{\ar@{}[rd]|(.5)*+{#1}}

\newcommand{\arr}[1]{\ar@<.5ex>[#1]\ar@<-.5ex>[#1]}
\newcommand{\arrr}[1]{\ar@<.7ex>[#1]\ar@<0ex>[#1]\ar@<-.7ex>[#1]}
\newcommand{\arrrr}[1]{\ar@<.9ex>[#1]\ar@<.3ex>[#1]\ar@<-.3ex>[#1]\ar@<-.9ex>[#1]}
\newcommand{\arrrrr}[1]{\ar@<1ex>[#1]\ar@<.5ex>[#1]\ar[#1]\ar@<-.5ex>[#1]\ar@<-1ex>[#1]}

\newcommand{\To}[1]{\xrightarrow{#1}}
\newcommand{\Too}[1]{\xrightarrow{\ \ #1\ \ }}
\newcommand{\From}[1]{\xleftarrow{#1}}

\newcommand{\Adjoint}[4]{\xymatrix@1{#2 \ar@<.5ex>[r]^-{#1} & #3 \ar@<.5ex>[l]^-{#4}}}

\def\id{\tn{id}}
\def\Top{{\bf Top}}
\def\Cat{{\bf Cat}}
\def\Str{{\bf Str}}
\def\Sets{{\bf Sets}}
\def\Set{{\bf Set}}
\def\sSet{{\bf sSet}}
\def\sSets{{\bf sSets}}
\def\Grpd{{\bf Grpd}}
\def\Pre{{\bf Pre}}
\def\Shv{{\bf Shv}}
\def\Rings{{\bf Rings}}

\def\colim{\mathop{\tn{colim}}}

\def\mcA{\mc{A}}
\def\mcB{\mc{B}}
\def\mcC{\mc{C}}
\def\mcD{\mc{D}}
\def\mcE{\mc{E}}
\def\mcF{\mc{F}}
\def\mcG{\mc{G}}
\def\mcH{\mc{H}}
\def\mcI{\mc{I}}
\def\mcJ{\mc{J}}
\def\mcK{\mc{K}}
\def\mcL{\mc{L}}
\def\mcM{\mc{M}}
\def\mcN{\mc{N}}
\def\mcO{\mc{O}}
\def\mcP{\mc{P}}
\def\mcQ{\mc{Q}}
\def\mcR{\mc{R}}
\def\mcS{\mc{S}}
\def\mcT{\mc{T}}
\def\mcU{\mc{U}}
\def\mcV{\mc{V}}
\def\mcW{\mc{W}}
\def\mcX{\mc{X}}
\def\mcY{\mc{Y}}
\def\mcZ{\mc{Z}}

\newtheorem{theorem}{Theorem}[section]
\newtheorem{lemma}[theorem]{Lemma}
\newtheorem{proposition}[theorem]{Proposition}
\newtheorem{corollary}[theorem]{Corollary}
\newtheorem{fact}[theorem]{Fact}

\theoremstyle{remark}
\newtheorem{remark}[theorem]{Remark}
\newtheorem{example}[theorem]{Example}
\newtheorem{warning}[theorem]{Warning}
\newtheorem{question}[theorem]{Question}
\newtheorem{guess}[theorem]{Guess}
\newtheorem{answer}[theorem]{Answer}
\newtheorem{construction}[theorem]{Construction}

\theoremstyle{definition}
\newtheorem{definition}[theorem]{Definition}
\newtheorem{notation}[theorem]{Notation}
\newtheorem{conjecture}[theorem]{Conjecture}
\newtheorem{postulate}[theorem]{Postulate}

\def\Finm{{\bf Fin_{m}}}
\def\El{{\bf El}}
\def\Gr{{\bf Gr}}
\def\U{{\bf U}}
\def\DT{{\bf DT}}
\def\DB{{\bf DB}}
\def\Tables{{\bf Tables}}
\def\Sch{{\bf Sch}}
\def\Fin{{\bf Fin}}
\def\P{{\bf P}}
\def\SC{{\bf SC}}
\def\ND{{\bf ND}}
\def\Poset{{\bf Poset}}
\def\'{\textnormal{'}}

%%%%%

\begin{document}

\title{Categorical description of Peter Gates's Dictionary Pattern}

\author{David I. Spivak}

\thanks{This project was supported in part by a grant from the Office of Naval Research: N000140910466.}

\maketitle

\section{Introduction}

The data dictionary pattern is a broad view of what a database is.  Interestingly, this pattern itself can be expressed as a database.  A slide deck on the subject, written by Peter Gates, can be found here at: \href{http://www.uoregon.edu/~dspivak/cs/db_dic.pdf}{\tt http://www.uoregon.edu/$\sim$dspivak/cs/db\_dic.pdf}.  In this note, I will give a category theoretic description of this idea.

I will use the following idea for labeling data columns.

\begin{remark}\label{rem:1-col tables}

One can imagine each data type as a table with one column and a row for each possible value of that data type.  For example the data type char(1) can be represented as a table $T_{\tn{char(1)}}$  with one column and 26 rows.  Similarly the data type char(2) can be represented as a table $T_{\tn{char(2)}}$ with one column and $26*26=676$ rows.  

\end{remark}

\section{The dictionary pattern}

The following is a reformulation of slide 2.  What Gates calls ``Value" I am calling ``Cell".  I am also adding information about the type of a column.  Let $\DT$ denote the set of datatypes, $\U$ the disjoint union of their domains, and $\pi\taking\U\to\DT$ the function that associates each instance to its type.  While impractical, we can think of $\DT$ as a table of data types, think of $\U$ as a table of their instances, and think of $\pi$ as a foreign key with source $\U$ and target $\DT$.  

In the following diagram there are two paths from Cell to Table -- these commute.  There are two paths from Cell to $\DT$ -- these commute.  Finally, there are two paths from Column to String -- these do not commute. 

$$\xymatrix@=1.7cm{&\tn{Table}\ar[rr]^{\tn{Name}}&&T_\tn{String}\\\tn{Row}\ar[ur]^{\tn{fk\_row\_table}\;\;}&&\tn{Column}\ar[ur]_{\tn{Name}}\ar[ul]_{\;\;\tn{ fk\_column\_table}}\ar[rr]_{\;\tn{fk\_column\_dt}} &&\DT\\&\tn{Cell}\ar[ul]^{\tn{fk\_cell\_row}}\ar[ur]_{\;\;\tn{fk\_cell\_column}}\ar[rrr]_{\tn{fk\_cell\_u}}&&&\U\ar[u]_{\tn{fk\_u\_dt}}}$$

Gates's original diagram looks like this: $$\xymatrix{&\tn{Table}\ar[rr]^{\tn{Name}}&&T_\tn{String}\\\tn{Row}\ar[ur]&&\tn{Column}\ar[ur]_{\tn{Name}}\ar[ul]\\&\tn{Cell}\ar[ul]\ar[ur]\ar[rrr]_{\tn{value}}&&&\U}$$ The seven arrows above correspond to Gates's seven non-id columns. 

\begin{remark}\label{rem:naming arrows}

In category theory, every arrow has a source and target, just like a foreign key does.  Since each arrow is typically drawn from the source to the target, there is no need to include the name of the source and target of an arrow when naming the arrow.  In other words, if an arrow takes $A$ to $B$, we can call it $f\taking A\to B$, rather than something like \tn{fk\_$A$\_$B$}.  This frees us to name the arrow by its meaning rather than its syntax.  Moreover, we may also have multiple arrows from $A$ to $B$, which the above naming system may have a harder time handling.

We shall therefore adopt the convention that an arrow may have any name, such as $f\taking A\to B$.  This arrow will represent a foreign key column in table $A$ that refers to the primary key column of table $B$, and this foreign key will be named $f$.  

We also draw data columns of table $A$  (i.e. columns of $A$ that are not foreign keys) as arrows out of $A$.  The target of such an arrow will be the data type of that column.  This is in keeping with [Spivak.  A simple model of databases].   See Remark \ref{rem:1-col tables}.  The idea then is that a column of $A$ with data type char(1) can be represented as a foreign key $A\to T_{\tn{char(1)}}$.  With this convention, both keys and data are represented in the same way.

\end{remark}

The following is a redrawing of the above diagram using arrows decorated with their ``meaning" rather than their source and target.  An arrow whose meaning is ``obvious" will be left unlabeled.  I will refer to the following as the ``dictionary pattern":

\begin{align}\tag{Dictionary Pattern}\label{dictionary pattern}\xymatrix@=1.2cm{&\tn{Table}\ar[rr]^{\tn{Name}}&&T_{\tn{String}}\\\tn{Row}\ar[ur]&&\tn{Column}\ar[ur]_{\tn{Name}}\ar[ul]\ar[rr]_{\;\tn{data\_type}} &&\DT\\&\tn{Cell}\ar[ul]\ar[ur]\ar[rrr]_{\tn{value}}&&&\U.\ar[u]_\pi}\end{align}

\section{Dictionary: Referential Integrity}

We now refer to slide 3 in Gates's deck.  We make the following changes: we drop the ``Schema" table for simplicity, we change the naming system as per Remark \ref{rem:naming arrows}.

\begin{align}\xymatrix{\tn{Table}&\tn{Constraint}}\end{align}


\end{document}