\documentclass{amsart}

\usepackage{amssymb, amscd,stmaryrd,setspace,hyperref,color}

\input xy
\xyoption{all} \xyoption{poly} \xyoption{knot}\xyoption{curve}
\input{diagxy}


\newcommand{\comment}[1]{}

\newcommand{\longnote}[2][4.9in]{\fcolorbox{black}{yellow}{\parbox{#1}{\color{black} #2}}}
\newcommand{\note}[1]{\fcolorbox{black}{yellow}{\color{black} #1}}
\newcommand{\q}[1]{\begin{question}#1\end{question}}
\newcommand{\g}[1]{\begin{guess}#1\end{guess}}

\def\tn{\textnormal}
\def\mf{\mathfrak}
\def\mc{\mathcal}

\def\ZZ{{\mathbb Z}}
\def\QQ{{\mathbb Q}}
\def\RR{{\mathbb R}}
\def\CC{{\mathbb C}}
\def\AA{{\mathbb A}}
\def\PP{{\mathbb P}}
\def\NN{{\mathbb N}}

\def\Hom{\tn{Hom}}
\def\Fun{\tn{Fun}}
\def\Ob{\tn{Ob}}
\def\Op{\tn{Op}}

\def\to{\rightarrow}
\def\from{\leftarrow}
\def\cross{\times}
\def\taking{\colon}
\def\inj{\hookrightarrow}
\def\surj{\twoheadrightarrow}
\def\too{\longrightarrow}
\def\tooo{\longlongrightarrow}
\def\tto{\rightrightarrows}
\def\ttto{\equiv\!\!>}
\def\ss{\subset}
\def\superset{\supset}
\def\iso{\cong}
\def\down{\downarrow}
\def\|{{\;|\;}}
\def\m1{{-1}}
\def\op{^\tn{op}}
\def\loc{\tn{loc}}
\def\la{\langle}
\def\ra{\rangle}
\def\wt{\widetilde}
\def\wh{\widehat}
\def\we{\simeq}
\def\ol{\overline}
\def\ul{\underline}
\def\qeq{\mathop{=}^?}

\def\ullimit{\ar@{}[rd]|(.3)*+{\lrcorner}}
\def\urlimit{\ar@{}[ld]|(.3)*+{\llcorner}}
\def\lllimit{\ar@{}[ru]|(.3)*+{\urcorner}}
\def\lrlimit{\ar@{}[lu]|(.3)*+{\ulcorner}}
\def\ulhlimit{\ar@{}[rd]|(.3)*+{\diamond}}
\def\urhlimit{\ar@{}[ld]|(.3)*+{\diamond}}
\def\llhlimit{\ar@{}[ru]|(.3)*+{\diamond}}
\def\lrhlimit{\ar@{}[lu]|(.3)*+{\diamond}}
\newcommand{\clabel}[1]{\ar@{}[rd]|(.5)*+{#1}}

\newcommand{\arr}[1]{\ar@<.5ex>[#1]\ar@<-.5ex>[#1]}
\newcommand{\arrr}[1]{\ar@<.7ex>[#1]\ar@<0ex>[#1]\ar@<-.7ex>[#1]}
\newcommand{\arrrr}[1]{\ar@<.9ex>[#1]\ar@<.3ex>[#1]\ar@<-.3ex>[#1]\ar@<-.9ex>[#1]}
\newcommand{\arrrrr}[1]{\ar@<1ex>[#1]\ar@<.5ex>[#1]\ar[#1]\ar@<-.5ex>[#1]\ar@<-1ex>[#1]}

\newcommand{\To}[1]{\xrightarrow{#1}}
\newcommand{\Too}[1]{\xrightarrow{\ \ #1\ \ }}
\newcommand{\From}[1]{\xleftarrow{#1}}

\newcommand{\Adjoint}[4]{\xymatrix@1{#2 \ar@<.5ex>[r]^-{#1} & #3 \ar@<.5ex>[l]^-{#4}}}

\def\id{\tn{id}}
\def\Top{{\bf Top}}
\def\Cat{{\bf Cat}}
\def\Str{{\bf Str}}
\def\Sets{{\bf Sets}}
\def\Set{{\bf Set}}
\def\sSet{{\bf sSet}}
\def\sSets{{\bf sSets}}
\def\Grpd{{\bf Grpd}}
\def\Pre{{\bf Pre}}
\def\Shv{{\bf Shv}}
\def\Rings{{\bf Rings}}

\def\colim{\mathop{\tn{colim}}}

\def\mcA{\mc{A}}
\def\mcB{\mc{B}}
\def\mcC{\mc{C}}
\def\mcD{\mc{D}}
\def\mcE{\mc{E}}
\def\mcF{\mc{F}}
\def\mcG{\mc{G}}
\def\mcH{\mc{H}}
\def\mcI{\mc{I}}
\def\mcJ{\mc{J}}
\def\mcK{\mc{K}}
\def\mcL{\mc{L}}
\def\mcM{\mc{M}}
\def\mcN{\mc{N}}
\def\mcO{\mc{O}}
\def\mcP{\mc{P}}
\def\mcQ{\mc{Q}}
\def\mcR{\mc{R}}
\def\mcS{\mc{S}}
\def\mcT{\mc{T}}
\def\mcU{\mc{U}}
\def\mcV{\mc{V}}
\def\mcW{\mc{W}}
\def\mcX{\mc{X}}
\def\mcY{\mc{Y}}
\def\mcZ{\mc{Z}}

\newtheorem{theorem}{Theorem}[section]
\newtheorem{lemma}[theorem]{Lemma}
\newtheorem{proposition}[theorem]{Proposition}
\newtheorem{corollary}[theorem]{Corollary}
\newtheorem{fact}[theorem]{Fact}

\theoremstyle{remark}
\newtheorem{remark}[theorem]{Remark}
\newtheorem{example}[theorem]{Example}
\newtheorem{warning}[theorem]{Warning}
\newtheorem{question}[theorem]{Question}
\newtheorem{guess}[theorem]{Guess}
\newtheorem{answer}[theorem]{Answer}
\newtheorem{construction}[theorem]{Construction}

\theoremstyle{definition}
\newtheorem{definition}[theorem]{Definition}
\newtheorem{notation}[theorem]{Notation}
\newtheorem{conjecture}[theorem]{Conjecture}
\newtheorem{postulate}[theorem]{Postulate}

\def\Finm{{\bf Fin_{m}}}
\def\El{{\bf El}}
\def\Gr{{\bf Gr}}
\def\DT{{\bf DT}}
\def\DB{{\bf DB}}
\def\Tables{{\bf Tables}}
\def\Sch{{\bf Sch}}
\def\Fin{{\bf Fin}}
\def\P{{\bf P}}
\def\SC{{\bf SC}}
\def\ND{{\bf ND}}
\def\Poset{{\bf Poset}}
\def\Automata{{\bf Automata}}


\begin{document}

\title{The category of automata}

\maketitle

The following definition, restricted to objects, is \cite[Definition 1.5]{Sipser}.

\begin{definition}

An {\em automaton} consists of a 5-tuple $\mcA=(Q,\Sigma,\delta,q_0,F)$ where $Q$ and $\Sigma$ are sets, $q_0\in Q$ is an element of $Q$ and $F\ss Q$ is a subset of $Q$, and where $\delta\taking\Sigma\cross Q\to Q$ is a function.  The elements of $Q$ are called {\em states} and the elements of $\Sigma$ are called {\em letters}.

Suppose that $\mcA'=(Q',\Sigma',\delta',q_0',F')$ is another automaton.  A {\em morphism of automata}, denoted $(f,g)\taking\mcA\to\mcA'$, consists of a function $f\taking Q\to Q'$ and a function $g\taking\Sigma\to\Sigma'$ such that $f(q_0)=q_0', f(F)\ss F'$ and the diagram $$\xymatrix{Q\cross\Sigma\ar[r]^\delta\ar[d]_{f\cross g}&Q\ar[d]^f\\Q'\cross\Sigma'\ar[r]_{\delta'}&Q'}$$ commutes.  We denote the category of automata by $\Automata$.

\end{definition}

\begin{definition}

Let $\mcA=(Q,\Sigma,\delta,q_0,F)$ be an automaton.  We call the monoid $\Sigma^*$ of sequences in $\Sigma$ the {\em action monoid} for $\mcA$, and note that $\delta$ gives $Q$ the structure of a right $\Sigma^*$-set.  For $a\in\Sigma^*$ and $q\in Q$, we write $q\cdot a$ to denote the result of acting $a$ on $q$.

The {\em language} of $\mcA$, denoted $L(\mcA)$ is the subset of elements $a\in\Sigma^*$ such that $q_0\cdot a\ss F$.  Note that a morphism $(f,g)\taking\mcA\to\mcA'$ induces a morphism of sets $L(\mcA)\to L(\mcA')$, so that $L\taking\Automata\to\Sets$ is a functor.

\end{definition}

I'm not sure what the point of the following definition is, but it is natural to define.

\begin{definition}

Let $\mcA=(Q,\Sigma,\delta,q_0,F)$ be an automaton.  An element $a\in\Sigma^*$ is said to {\em preserve acceptance} if, for all $q\in F$ one has $q\cdot a\in F$.  Note that the set of letters that preserve acceptance is a submonoid of $\Sigma^*$.

\end{definition}

\bibliographystyle{amsalpha}
\begin{thebibliography}{JTT}

\bibitem [Sipser]{Sipser} Sipser, M.  {\em Introduction to the Theory of Computation} Second edition.

\end{thebibliography}

\end{document}