\documentclass{amsart}


\usepackage{amssymb, amscd,stmaryrd,setspace,hyperref,color}

\input xy
\xyoption{all} \xyoption{poly} \xyoption{knot}\xyoption{curve}

\newcommand{\comment}[1]{}

\newcommand{\longnote}[2][4.9in]{\fcolorbox{black}{yellow}{\parbox{#1}{\color{black} #2}}}
\newcommand{\note}[1]{\fcolorbox{black}{yellow}{\color{black} #1}}
\newcommand{\q}[1]{\begin{question}#1\end{question}}
\newcommand{\g}[1]{\begin{guess}#1\end{guess}}

\def\tn{\textnormal}
\def\mf{\mathfrak}
\def\mc{\mathcal}

\def\ZZ{{\mathbb Z}}
\def\QQ{{\mathbb Q}}
\def\RR{{\mathbb R}}
\def\CC{{\mathbb C}}
\def\AA{{\mathbb A}}
\def\PP{{\mathbb P}}
\def\NN{{\mathbb N}}

\def\bD{{\bf \Delta}}
\def\Str{{\bf Str}}

\def\Hom{\tn{Hom}}
\def\Fun{\tn{Fun}}
\def\Nat{\tn{Nat}}
\def\Ob{\tn{Ob}}
\def\Op{\tn{Op}}

\def\to{\rightarrow}
\def\from{\leftarrow}
\def\cross{\times}
\def\taking{\colon}
\def\inj{\hookrightarrow}
\def\surj{\twoheadrightarrow}
\def\too{\longrightarrow}
\def\fromm{\longleftarrow}
\def\down{\downarrow}
\def\Down{\Downarrow}
\def\Up{\Uparrow}
\def\tooo{\longlongrightarrow}
\def\tto{\rightrightarrows}
\def\ttto{\equiv\!\!>}
\def\ss{\subset}
\def\superset{\supset}
\def\iso{\cong}
\def\|{{\;|\;}}
\def\m1{{-1}}
\def\op{^\tn{op}}
\def\loc{\tn{loc}}
\def\la{\langle}
\def\ra{\rangle}
\def\wt{\widetilde}
\def\wh{\widehat}
\def\we{\simeq}
\def\ol{\overline}
\def\ul{\underline}
\def\qeq{\mathop{=}^?}
\def\setto{\colon\hspace{-.25cm}=}

\def\ullimit{\ar@{}[rd]|(.3)*+{\lrcorner}}
\def\urlimit{\ar@{}[ld]|(.3)*+{\llcorner}}
\def\lllimit{\ar@{}[ru]|(.3)*+{\urcorner}}
\def\lrlimit{\ar@{}[lu]|(.3)*+{\ulcorner}}
\def\ulhlimit{\ar@{}[rd]|(.3)*+{\diamond}}
\def\urhlimit{\ar@{}[ld]|(.3)*+{\diamond}}
\def\llhlimit{\ar@{}[ru]|(.3)*+{\diamond}}
\def\lrhlimit{\ar@{}[lu]|(.3)*+{\diamond}}
\newcommand{\clabel}[1]{\ar@{}[rd]|(.5)*+{#1}}

\newcommand{\arr}[1]{\ar@<.5ex>[#1]\ar@<-.5ex>[#1]}
\newcommand{\arrr}[1]{\ar@<.7ex>[#1]\ar@<0ex>[#1]\ar@<-.7ex>[#1]}
\newcommand{\arrrr}[1]{\ar@<.9ex>[#1]\ar@<.3ex>[#1]\ar@<-.3ex>[#1]\ar@<-.9ex>[#1]}
\newcommand{\arrrrr}[1]{\ar@<1ex>[#1]\ar@<.5ex>[#1]\ar[#1]\ar@<-.5ex>[#1]\ar@<-1ex>[#1]}

\newcommand{\To}[1]{\xrightarrow{#1}}
\newcommand{\Too}[1]{\xrightarrow{\ \ #1\ \ }}
\newcommand{\From}[1]{\xleftarrow{#1}}


\newcommand{\Adjoint}[4]{\xymatrix@1{#2 \ar@<.5ex>[r]^-{#1} & #3 \ar@<.5ex>[l]^-{#4}}}
\newcommand{\adjoint}[4]{\xymatrix@1{#1\colon #2\ar@<.5ex>[r]& #3\;:#4 \ar@<.5ex>[l]}}

\def\id{\tn{id}}
\def\Top{{\bf Top}}
\def\Cat{{\bf Cat}}
\def\Sets{{\bf Sets}}
\def\sSets{{\bf sSets}}
\def\Grpd{{\bf Grpd}}
\def\Pre{{\bf Pre}}
\def\Shv{{\bf Shv}}
\def\Rings{{\bf Rings}}

\def\colim{\mathop{\tn{colim}}}

\def\mcA{\mc{A}}
\def\mcB{\mc{B}}
\def\mcC{\mc{C}}
\def\mcD{\mc{D}}
\def\mcE{\mc{E}}
\def\mcF{\mc{F}}
\def\mcG{\mc{G}}
\def\mcH{\mc{H}}
\def\mcI{\mc{I}}
\def\mcJ{\mc{J}}
\def\mcK{\mc{K}}
\def\mcL{\mc{L}}
\def\mcM{\mc{M}}
\def\mcN{\mc{N}}
\def\mcO{\mc{O}}
\def\mcP{\mc{P}}
\def\mcQ{\mc{Q}}
\def\mcR{\mc{R}}
\def\mcS{\mc{S}}
\def\mcT{\mc{T}}
\def\mcU{\mc{U}}
\def\mcV{\mc{V}}
\def\mcW{\mc{W}}
\def\mcX{\mc{X}}
\def\mcY{\mc{Y}}
\def\mcZ{\mc{Z}}

\newtheorem{theorem}{Theorem}[subsection]
\newtheorem{lemma}[theorem]{Lemma}
\newtheorem{proposition}[theorem]{Proposition}
\newtheorem{corollary}[theorem]{Corollary}
\newtheorem{fact}[theorem]{Fact}

\theoremstyle{remark}
\newtheorem{remark}[theorem]{Remark}
\newtheorem{example}[theorem]{Example}
\newtheorem{warning}[theorem]{Warning}
\newtheorem{question}[theorem]{Question}
\newtheorem{guess}[theorem]{Guess}
\newtheorem{answer}[theorem]{Answer}
\newtheorem{construction}[theorem]{Construction}

\theoremstyle{definition}
\newtheorem{definition}[theorem]{Definition}
\newtheorem{notation}[theorem]{Notation}
\newtheorem{conjecture}[theorem]{Conjecture}
\newtheorem{postulate}[theorem]{Postulate}

\def\DT{{\bf DT}}
\def\GD{{\bf GD}}
\def\DB{\GD}
\def\Sch{{\bf Sch}}
\def\Null{{\bf Null}}
\def\Strings{{\bf Strings}}
\def\ND{{\bf ND}}
\def\Tables{{\bf Tables}}
\def\'{\tn{'}}
\def\disunion{\amalg}
\def\Rel{{\bf Rel}}
\def\mcRel{{\bf \mcR el}}
\def\Cech{$\check{\tn{C}}$ech }
\def\C{\check{\tn{C}}}
\def\Fin{{\bf Fin}}
\def\singleton{{\{*\}}}
\def\Sub{{\bf Sub}}
\def\card{\tn{card}}
\def\Data{{\bf DB}}
\def\DB{{\bf DB}}
\def\im{\tn{im}}
\def\'{\tn{'}}
\def\start{\note{start here}}





\def\DT{{\bf DT}}
\def\GD{{\bf GD}}
\def\Tables{{\bf Tables}}
\def\Sub{{\bf Sub}}
\def\bD{{\bf \Delta}}
\def\down{\downarrow}
\def\Map{{\bf Map}}

\begin{document}

\author{David I. Spivak}

\title{Ideas on networks etc.}

\maketitle

\tableofcontents

\begin{abstract}

General musings on CIT (Categorical Information Theory).

\end{abstract}

\section{Networks}

\begin{definition}

Let $\mcC$ denote a category with finite limits.  Let $D$ denote a semi-simplicial set and $U\taking(\Delta\downarrow D)\op\to\mcC$ a presheaf on $D$ in $\mcC$.  A {\em network of type $(\mcC,D,U)$} consists of a sequence $(X,\sigma,\mcO_X,\tau)$, where $X$ is a semi-simplicial set, $\sigma\taking X\to D$ is a morphism, $\mcO_X\taking (\Delta\downarrow X)\op\to\mcC$ is a presheaf, and $\tau\taking\mcO_X\to\sigma^*U$ is a morphism of presheaves.  We often abuse notation and write $(X,\mcO_X)$ to denote this network.

A {\em morphism of networks of type $(\mcC,D,U)$} is written $$(f,f^\sharp)\taking (X,\sigma_X,\mcO_X,\tau_X)\to(Y,\sigma_Y,\mcO_Y,\tau_Y)$$ and is given by a morphism of semi-simplicial sets $f\taking X\to Y$ such that $\sigma_X=\sigma_Y\circ f$, and a morphism of presheaves $f^\sharp\taking f^*\mcO_Y\to\mcO_X$ on $X$, such that $f^*(\tau_Y)=\tau_X\circ f^\sharp$.

\end{definition}

\begin{example}

A geometric database with data type designation $\pi\taking U\to\DT$ is a network of type $(\Sets,\C(\DT),\C(U))$.  So $\GD$ can be imbedded into this category.

\end{example}

\begin{example}

A graph is a network of type $(\{*\},S^1,U)$, where $\mcC=\{*\}$ is the terminal category, $S^1$ is the unique semi-simplicial set with one 1-simplex, one 0-simplex, and no other simplicies, and $U$ is the unique presheaf on $\mcC$ over $(\Delta\downarrow S^1)$.  

\end{example}

\begin{example}

One may be interested in a network of data types.  Each node in the network has a set of types that it ``understands," as do higher simplices, and if a higher simplex understands a data type then all its faces do too.  

Also keep in mind the following idea.  Morphisms between datatypes should maybe be retractions.  In other words, a map $A\to B$ should have a section $B\to A$.  The idea is that $A$ understands what it's saying better than $B$ does -- that's a given.  $B$ can come back and tell $A$ what it understood, by repeating what it heard.  But it's repeating it in a lower data-type.  For example $A$ may communicate to $B$ in English, but English repeats it back just in sounds.  $A$ recognizes the sounds but was actually communicating on a higher plane.

\end{example}

\begin{example}

Piano skill.  A network on the ssSet $\Delta^{10}$.  An $n$-simplex $x\taking\Delta^n\to\Delta^{10}$ corresponds to a set of fingers.  For each such simplex, put a database with schema $\Delta^1$.  Over each vertex of the schema put, as data, the set of $n$-tuples of notes playable by that set of fingers.  Over the edge, put the set of transitions which are playable.  

The only slightly weird thing here is that the above is a network in $\GD\op$.  Given an inclusion of a simplex, say $\Delta^2\ss\Delta^3$, we get a map of databases from 2-finger combos to 3-finger combos, because maps of databases are contravariant in the data.

\end{example}

\subsection{Meaning}

In a network, a node can speak to a simplex. In so doing, he in fact speaks to each subsimplex in its language.  If $x$ is a face in $y$ then the speech to $x$ is more refined than that to $y$.  That is, a simplex understands (at best) the ``lowest common denominator" of its faces.

Of course, there may be a sense in which the group ``understands" what was said better than any subsimplex does.  This is eventually what I'd like to understand.

\section{Learning}

\subsection{Changing metrics}

Jeff Hawkins writes of the phenomenon "fire together, wire together."  The idea is that when two neurons fire at the same time, a connection may be formed between them, thus decreasing the ``space" between them in the brain topology.

Here's a model for doing that mathematically.  Suppose that $(M,D_M)$ is a metric space, $x,y\in M$ are two points, and $d\geq 0$ is a number, perhaps thought of as being related to the amount of time between the firing of neurons $x$ and $y$.

We form a new metric space $N$ having the same underlying set of points as $M$ does, but with the following metric.  For $a,b\in M$, define \begin{align*}D_N(a,b)=\min\{&D_M(a,b),\\&D_M(a,x)+d+D_M(y,b),\\&D_M(a,y)+d+D_M(x,b)\}.\end{align*}

Clearly, $D_N(x,y)\leq d$, $D_N$ is symmetric, and (if $d>0$) $D_N$ satisfies $D_N(a,b)\geq 0$ iff $a\neq b$.  To show that $D_N$ satisfies the triangle inequality, one proceeds by cases.

\subsection{Query optimization}

Higher level structures in the brain ask for information from the lower level structures.  Certain queries are performed on lower level databases more often than others are.  The data should be organized to optimize the speed of these queries.  Data should be sorted for the ease of superiors, as predicted by their past behavior.

\subsection{Querying a network}

In a network of databases, one would like to query the network.  The network should be able to direct the user's question to the correct set of subsimplices who can answer the question.  That is a query should descend through the network hierarchy until it can be answered (or guessed at).  

In a democracy, each database (person) gives input as to the answer.  In general, this may not be so.

In general, the query spreads down (and up) the hierarchy, looking for paritial answers from subordinates and for clarification of the question from the superiors.

What form does an answer take?  It seems it could reside at a variety of levels (a narrative).  ``More details are needed here, no more are needed there, etc."

\question{In a network, can one vertex query another?  In whatever theory we have, how does one node use a common 1-simplex to query another node?}

\subsection{Changing the topology to make data continuous}

It is fundamental in learning to understand sameness and difference, to understand continuity and discontinuity.  When are two observations part of the same event, and when are they not?  

As data comes in to the observer, the observer decides when data belong together and when they don't.  There is an effort to produce entities (forms).  When data comes in, we use its proximity in space and time (and other dimensions) to decide what is connected and what isn't.  This forms the topology in our mind.  Two perceptions that we deem close should be connected and wired together in our mind.

Somehow, it would be nice to see a database as giving this.  Inside the space of all possible sections of $A\cross B\cross C$ there are some chosen ones, which are deemed to be ``three aspects of the same entity."  Sections of a database are entities; they are data that appear to be of the same origin.  So a database is filled by entities whose one-ness is established in some way.  For example, a person calls a company and is entered in the database: the call itself establishes the one-ness of the entity being described (the person), and we get a single section of the data bundle.

But in general, when learning, we want to wire things together in our mind if they ``go together," i.e. if doing so improves the continuity of the things we see.  We change the topology to fit the data.  Can this be made rigorous?

\subsection{Model for recognition}

Consider the big category of tables $\Tables_+$.  Imagine someone is talking to you.  You are receiving data in the form of strings (maybe facial expressions and maybe you have some context lying around). For each string (sentence, etc.), you can lift that section of the table to various other tables.  In fact, there is a diagram of (tables, lifts) for those data.  This is the context category for that data (sentence).

The main point is: I have a record, or a table of records, of type $(C,\sigma)$ and I'm interested in finding lifts of these data $$\xymatrix{&\Gamma(C',\sigma')\ar[d]^f\\R\ar[r]_\tau\ar@{--}[ru]^{\tau'}&\Gamma(C,\sigma).}$$

Let $Con(\tau)$ be the category whose objects are pairs $(f,\tau)$, liftings as above.

\section{Notes from 2008.09.27}

In this section I will just write down some ideas I had today.

The goal of the network is to ``process information."  Each node has some edges on it, both input edges and output edges.  Some are rays (i.e. edges with no other endpoint) and some are segments (connecting to another node).  Each node gets information from neighboring nodes and ``processes it" or ``interprets it."  It is generally this processed information which is passed on.

Time seems to be important in networks.  Information is ``processed" and then later ``passed on."  This is just something to keep in mind, because it isn't purely categorical.

Is there such a thing as ``false information"?  I think so.  Input information can somehow be ``mis-interpreted" and passed on into the network.  I think that it is really important to know what ``false information" is.  Characterize it.

I think the process goes something like this.  Information can only be transferred between nodes when the database schema match and some data has been shared in that schema that is agreed on by both parties.

False information is transfered when the schemas seem to match and the data seems to match, but ``false data" is sent.  This data should be detected soon, as it meshes in with the rest of the node's database.  However it is possible that the falseness in the data is sufficiently ``subtle" to prevent being discovered as false.

The goal of the network is to see patterns in the phenomena presented to the antennae.  Each node is its own network in which every exterior edge is an antenna.  By presenting good information, the antenna helps the network, and by processing information well, the network somehow helps itself.  It helps itself by settling itself better into the network, being more useful to the larger network (because then the larger network feeds it better, whatever that may mean).  Edges are modified in the process to strengthen connections to helpful nodes and erode connections to bad nodes.

\section{Notes from 2008.10.01}

When a ``phenomenon" takes place, nodes communicate their impressions of it along relevant channels.  Perhaps we can identify the phenomenon with the network communication that it inspires.  To do so is to take seriously the ``if a tree falls in the forest" Koan.\vspace{.1in}

Let $X$ be a simplicial set, and let $\mcN_X=(\Delta\downarrow X)$ denote the category of simplicies of $X$.  In this section, we refer to $\mcN_X$ as the {\em network associated to $X$}.  A {\em communication system on $\mcN_X$} is a functor (fibration?) $p\taking\mcE\to\mcN_X$.

The idea is that a communication is a section of the bundle $p$.  That is, each node $x\in\mcN_X$ has a variety of ``things it can say," given by $p^\m1(x)$.  

A {\em phenomenon} consists of a triple $(A,f,s)$, where $\mcA$ is a category, $f\taking\mcA\to\mcN_X$ is a $\mcA$-shaped diagram in $\mcN_X$, and $s\taking\mcA\to f^\m1\mcE$ is a section of the induced bundle $$f^\m1(p)\taking f^\m1(\mcE)\to\mcA.$$

\section{Notes from 2009.02.25}

\subsection{Things}

One problem I may have been having is that I may have been confusing perspectives on things.  From the perspective of a database, a thing is a bearer of the properties outlined in the schema of that database.  It is perhaps the exterior key. 

From other perspectives, a thing is the bearer of properties which do not conform to any particular model.  The type of thing that this particular one is, dictates which databases fit it and which do not.

A thing may also be thought of as a presheaf on a category of components, which are atoms for the purposes of a given discourse.  That is, a thing is the gluing together of components.  The thing may also be considered as a network of components.  Each component has properties, and it is sometimes possible to extract global properties from local ones.  For example, the weight of an object is the sum of the weights of its components -- same with volume, but not so with height.

Things can change in time.  The rate at which an aspect of a thing can change is proportional to how much force is applied to the thing and how easy it is to change that aspect of the thing.  For example, the position of a rock is changed more if more force is applied to the rock and if the rock is less heavy.  

Thus, each attribute in a database should be a metric space, where distance measures how difficult it is to modify an attribute.    

\subsubsection{}

A thing is always a thing-as-such.  In other words, to give a thing, one must tell the way in which the thing is a thing -- that is, the attributes by which it presents itself.

An attribute of an object is a certain type of measurement of an object (a point in some affine line); measurements are always unique.  For example, a child is not an attribute of his parent; however, a child is an attribute of a given parent-child relationship.  The fact that one thing is ``composed of" a substance is not an attribute of the thing, because the thing could be composed of many different substances.  And yet ``composed entirely of" is an attribute of certain things, ``composed primarily of" is another attribute of certain things, because these attributes are uniquely assigned to those things.

The point is that each thing has its own ways of being measured.  When the thing is presented, the presentation must include (at least implicitly) the sense in which the thing is a thing; i.e. the attributes by which it should be measured.  The same object may be a thing in two different senses (perhaps to two different people), and this could present confusion.  We are taking the stance that this is a case of two different things, because inherent in the thing is the way that it is a thing.

To bring this discussion down to the level of our work, we say that each thing has, first and foremost, a schema by which it should be viewed.  This means that the one attribute that makes a thing a thing is the attribute whose domain is schema.  

You can't have a partial schema; just a schema.  This suggests that the proper schema for things is a point whose universal data bundle is the set (or category?) of schemas.

Here's a question.

\begin{question} 

Suppose that I have a thing.  Then I know its schema, because this is the primary attribute of things.  Once a schema is known, it opens up further questions, because now one wants to know where the thing fits into that schema.

I believe that this is the ``has" phenomenon.  Given a thing, we get a schema called ``the schema that the thing has."  We now take that information and open up questions like ``what section of this schema represents this thing?"  

The reason I think that this is part of a more general ``has" phenomenon is that if we know that a given person has a chair, then this means that there is some kind of link between the person and the schema ``chair."  A person is an instance of the type person (i.e. a section over the person schema), but the chair which he has is not yet known to us.  We have a given person but not a given chair, just a chair.  We do not have the details of the chair filled in; what is associated to the person is not a particular chair (yet!), but just the schema for chair.

\end{question}

\begin{question}

We said that every thing has a schema.  Is this schema always simple?  Perhaps one should not consider arbitrary schemas but only simple schemas for things.  

The way to answer this question is to consider what types of thingy has non-simple schemas, and whether any thing is a thingy.

\end{question}

\subsection{Simplicial databases -- generalizations}

One way to generalize simplicial databases is to have over every simplex in the schema $X$ a category rather than a node.  For example, ``vector bundles" is a nice example of a type designation.  There is one datatype, so the category of schema is precisely the category of simplicial sets.  If $X$ is a simplicial set, the universal database on $X$ is the sheaf $\mcU_X=\Hom(-,BO)$, where $BO$ is the classifying space for real vector bundles.  A database is then a sheaf on $X$ over $\mcU_X$, which amounts to a set of vector bundles over every open set.

One can use these categories to sort data.  For example, the set of strings is ordered, i.e. it forms a category.   

Another thing one could do is to put more structure on the sheaf $\mcR$ of keys and perhaps on the map $\tau\taking\mcR\to\mcU_X$.  For example, I think you can get ringed spaces this way.  

It might be interesting to study extra structure on the type designation $\pi\taking U\to\DT$.  For example, this probably should not just be a function between sets but something like a ?fibration of categories.  We want to be able to add functionality to the data we have.  Entries in certain columns can be added together, entries in certain columns could be applied to entries of other columns.  That is, one column may contain functions of which the domain is another column, or one column may contain elements of a ring and another column may contain elements of a module over that ring, etc.  

I would like to figure out how to make this mathematically pretty.  In other words, what type of thing could we replace our map $\pi\taking U\to\DT$ of sets by, in order to house enough structure to allow elements of one column to be applied to elements of another column?

\subsection{Homotopy theory in simplicial databases}

There may be some use in taking sheaves of simplicial sets instead of sheaves of sets, when defining simplicial databases.  The reason is that we may want to keep track of the fact that two global sections restrict to ``the same" section of a subset, without making those two restrictions equal.  We could do that by putting in 1-dimensional data.  This would reduce the pain of making a mistake: identified data could be unidentified, because no information loss occurs when you add a connecting simplex.

\section{2009/03/09}

To do:

Equivalence relations, quotienting by.
New types from old: given a type designation, what can you make from it?
A schema for which the schemas of a given type are data.

\subsection{}

Given the datatype $C$=``characters", consider the terminal category $\DT=\{*\}$, and the presheaf $\Gamma$ on $(\bD\down\DT)$ given by $\Gamma(\Delta^n)=\{(c_0,\ldots,c_n)\in C^{n+1}\}$.  But note that ``English words" is not a sub-presheaf!  The heuristic reason is that we are not working with strings as themselves, but as lists of characters, and it doesn't make sense to allow some lists of characters but not others.

\subsection{Circle example}

The schema for ``circle" would have a column for radius, a column for the $x$-coordinate of the center, and a column for the $y$-coordinate of the center.  But given such data $(r,c_x,c_y)$, one can consider the set of points in $\RR^2$ that satisfy the equation $$(x-c_x)^2+(y-c_y)^2=r^2.$$  This could be achieved by  a SELECT query in any number of ways; the result is a database with two columns $x$ and $y$ and each row is a pair $(x,y)$ that satisfies the above equation.  

The question is: how do you go from one record in the ``circle" database to a database on the schema $\RR^2$?  How can we formulate this idea?

Let's try.  So given a circle datum $(r,c_x,c_y)$, we get a new database with schema $\Delta^{(\RR,\RR)}$.  Given a database of circles $K\to\RR_{\geq 0}\cross\RR^2$, we can just keep track of the key set $K$, and form a new database whose schema is $(K,K,\DB_{\RR^2})$ and whose records are a pair of signifiers for circles together with a set of points, namely the intersection of the two sets of points corresponding to the circles.  We can then transform this into a database with schema $(K,K,\NN)$, for which an entry is a pair of signifiers for circles together with the number of intersection points of these two circles.

The mapping $\Gamma(r,c_x,c_y)\to\DB_{\RR^2}$, is a case of denotation.  The sign is the domain, which gives the codes for circles.  The referent is the image of the sign, namely the set of points to which the code refers.  

The way it works is that a record (element) in $\Gamma(r,c_x,c_y)$ gives a function $\RR^2\to\RR$ by $(a,b)\mapsto(a-c_x)^2+(b-c_y)^2-r^2$.  This can be formulated as a single table or as a map of tables.  Regardless, we then select out the elements for which this function returns $0$, and the result is a new database.  

One serious goal is to make all of this precise, and find a good grammar for it.  In other words, we need to understand $\Gamma(r,c_x,c_y)$ not just as a set of triples, but as a way to locate the database on $(\RR,\RR)$ to which it refers {\em as a circle}.  We attach to the concept of circlehood all of the various relationships that enter into it: the signifiers and signifieds, together with perhaps the database of round things.  But whenever one refers to a circle, he should be able to refer explicitly to what aspects of the circle (its center-radius, or its points) he means.

\subsection{Sammy and the physics paper}

Sammy Black was reading a physics paper which discussed an idea shared by physics and math; the idea was viewed somewhat differently, however, in that the subscript of a given variable in physics notation could be obtained from the subscript in math notation by a certain isomorphism (e.g. $m=\frac{p}{2}+7$).  Sammy found this very annoying. 

If the physics paper had been a purely information-theoretic document, then a simple transformation (functor) could have turned it into a math document; i.e. one would apply this functor and have all notation be mathematical.  

Similarly, consider the theory of sign relations.  The same sign (e.g. ``You") is interpreted by the speaker differently than it is by the spoken to (who interprets it as ``Me").  The ability to make such transformations should be part of CIT.

\section{Heuristics}

We should define ``generalized affine space" and ``generalized manifold."  A tree is not a generalized affine space, for example. 

\begin{definition}

Let $\mcC$ be a topological category.  A {\em coordinate system in $\mcC$} is simply a morphism $\pi\taking U\to D$ in $\mcC$, and the {\em affine space on $\pi$, denoted $\AA^\pi$} is the space of sections $$\AA^\pi=\Map_{\mcC_{/D}}(D,U)$$

\end{definition}

\begin{example}

Let $\mcC=\Top$, let $D=\{1,\ldots,n\}$ for any $n\in\NN$, let $U=\coprod_{i\in D}\RR$, and let $\pi\taking U\to D$ denote the obvious projection.  Then $\AA^\pi=\RR^n$.  

(This works even if $n=0$.)

\end{example}

\begin{example}

The space of sections of a vector bundle is a generalized affine space.

\end{example}

As a heuristic, I am now thinking of an $n$ simplex as spanning $\RR^{n+1}$ in the usual way.  A database schema is a gluing together of these spanning guys.  Note the difference between the space spanned by $\Delta^1\amalg_{\Delta^0}\Delta^1$ and that spanned by $\Delta^2$.  Although the global sections of each is isomorphic to $\RR^3$, there is somehow a different feel.  One is $\RR^2\cross_\RR\RR^2$ and the other is $\RR^3$.  While topologically these are the same, there may be a difference in geometry.  Are geometric objects closed under fiber product?  It seems like they aren't, which agrees with the intuition we get from databases, that there is a subtle difference between the schema $\Delta^1\amalg_{\Delta^0}\Delta^1$ and the schema $\Delta^2$.

\subsection{Queries}

One might consider a query as a kind of agent that is let loose in a field of information.  It is attracted to (has an affinity for) and binds to sources whose shape matches (complementarily) with that of the query.  When the query finds a site to which it can bind, it receives the data and can carry it away.  

\end{document}