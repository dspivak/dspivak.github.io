\documentclass[12pt]{letter}

\usepackage{fullpage}

\begin{document}

\begin{center}\Large Syllabus for Differential 
Equations \normalsize\end{center}

\vspace{.7in}

\Large Basic Information:\normalsize\\

\vspace{.1in}

\begin{tabular}{ll}

Class:& Math 256: Differential Equations\\
Meeting Days:& M, Tu, W, F\\
Meeting Time:& 8-8:50 am\\
Meeting Room:& 306 Deady.\\
&\\
Final Exam:& 10:15pm Thursday, December 6, 2007.\\
&\\
My Name:&David Spivak\\
My Office:&317 Fenton Hall\\
My Office Hours:& Monday, 9-10.\\

\end{tabular}

\vspace{.2in}

\Large Term Grading Structure and Outline:\normalsize\\

\vspace{.1in}

\begin{tabular}{ll}

Homework:&15\%\\
Quizzes:&15\%\\
PAC score:&5\%\\
Midterm 1:&20\%\\
Midterm 2:&20\%\\
Final Exam:&25\%\\

\end{tabular}

\vspace{.1in}

Homework will be assigned on Wednesdays and collected on Tuesdays.  
Sometimes something happens in your life and you can't complete a homework 
on time.  For example, a death in the family, an illness, massive amounts 
of homework from another class, a party.  In this case you may ``drop" 
that homework from your grade.  You may drop exactly one homework during 
the quarter, even if you have multiple calamities, so use it wisely!

Please don't email me with math questions.  These things are best 
communicated in person.  If you need to know an answer before I'll next 
see you, go to the Braddock Tutoring Center (155 Lillis Hall), or call a 
friend from class.

Quizzes will be on Wednesdays.  You may drop 1 quiz throughout the quarter 
(with the same reasoning and description as for dropping homework).

In this course, we'll cover chapters approximately chapters 1 through 4 
and chapter 7.  There will be two midterms and a final.

Test dates:\\

Test 1: Monday, October 15.

Test 2: Monday, November 5.

Final: Thursday, December 6.

\vspace{.2in}

\Large Details and Philosophy:\normalsize

The following details and philosophy is subject to minor changes, 
as result from the evolution of our class throughout the quarter.  Please 
tell me if you have questions or suggestions.  

\large Homework\normalsize 

The purpose of doing homework is to get acquainted with subjects that you 
don't already understand.  It is not to prove that you understand 
something.  The results (answers) to homework problems are fairly 
irrelevant.  The process is far more important.  Math is like a child: 
spend quality time with it and things will turn out well.  Just finding 
the answers somehow is like just buying the kid gifts: it is no substitute 
for quality time.

This course will have a grader, and this grader will have a personality of 
his or her own.  The grade you receive on your homeworks will be a 
function of many variables, and one of them is the grader's personality 
(grading style).  This is a reality you are going to have to deal with.  
On the other hand, you are invited to engage the grader directly about 
questions you have about your grades.

Here is how I envision the homework process.  When you do your homework, 
either alone or in a small group, your goal is to access and enunciate 
your own points of confusion.  By ``accessing" your points of confusion, I 
mean finding them for yourself, and by ``enunciating" your points of 
confusion, I mean writing them down for the grader.  

If a certain problem presents you with no points of confusion, then you 
will complete it and write the answer.  If it does confuse you, you will 
attempt to resolve the confusion.  If you can resolve your confusion, 
continue with the problem until the next point of confusion; if you don't 
resolve your confusion, clearly and concisely enunciate your point of 
confusion to the grader by writing it down on your homework paper.  Then 
move on to the next problem -- don't worry about the answer.  Seriously, 
knowing the answer is not as important as you have been led to believe.

The grader will attempt to understand my philosophy and attempt to grade 
the homework in accordance with it.  That is, if you write down why you 
are confused instead of writing an answer, you'll get full credit.

\large Quizzes\normalsize 

You will have a quiz every week.  It will test basic concepts, as directly 
as possible.  My philosophy on quizzes is the same as on homework.  If you 
get confused, spend your time finding the source of the confusion.  Once 
you've done so, play with it until you begin to find yourself moving 
again.  Don't be afraid to move backwards, if needed.  If you can't find 
the answer, write down why you are confused.  It will not be full credit 
(as it is on homework) but doing so will be worth points.

\large PAC score\normalsize 

PAC stands for Personal Addition to Class.  This is purely a subjective 
grade which I will assign you at the end of the term, possibly with your 
input.

When you ask questions that get at the heart of your ignorance, you make 
me happy.  When you make others feel comfortable asking questions that get 
to the heart of their ignorance, you make me happy.  Ignorance is not a 
problem, even though the word is typically used in a very negative way.  
What often {\em is} a problem is trying to hide your ignorance: let it 
out!  Flaunt it!  Einstein once said ``If you don't know something, say it 
loud."

Of course sometimes, you may ask a question which would be better handled 
in my office hours.  In that case, I'll tell you so.  In other words, you 
don't have to worry about the boundaries of what you can ask in class -- 
ask anything.

At the end of the term, you'll in some sense get to grade my PAC score on 
an evaluation form.  The course material is as it is -- you will grade me 
on how well I contributed to the class itself.  Similarly, I will grade 
your PAC score not on how well you understood the material, but on how 
engaged you were and how willing to improve your experience and the 
experiences of those around you.

In some sense, the PAC score is free points.  Just be yourself, foster 
creativity in yourself and others, come to class, and you will get 5 
percentage points for free.  On the other hand, please do me this favor.  
If you don't choose to work towards adding to the classroom environment, 
don't come to me at the end with some sort of death-bed change-of-heart.  
Don't come to me with dishonesty and attempts to improve your grade beyond 
what you know you deserve.  My compassion for students could be seen as a 
weakness of mine; please don't exploit it for your own gain.

\large Tests \normalsize  

Tests have a three-fold purpose.  First, they encourage you to learn 
throughout the quarter.  Second, they assess your knowledge.  Third, 
learning actually takes place during the tests.  Please understand that 
you are encouraged to learn things you didn't already know even as late as 
during the tests.

The best time to discover your own sources of confusion is as they arise.  
If you do the homework in the way I've suggested, you will discover your 
points of confusion and can address them during the homework process.  
This way, you will do well on the tests.  

Instead of thinking of the exam as a test of your knowledge, think 
of it as a test of your commitment to the homework, or more precisely, to 
the process of engaging the math you are presented with throughout the 
semester.  Don't learn how to solve the problems, learn to understand the 
mathematical truths behind the problems.  This will make you much more 
effective.\\

If your goal in this class is to get an A, I'm sorry to tell you that you 
may find it to be a painful experience.  If your goal in this class is to 
see if you can actually be interested in the subject matter, you will find 
the experience enjoyable and probably get the grade you want in the 
process.  I believe this to be so.  If your personal experience begins to 
run counter to this, come talk to me and we can search together for the 
problem.

\newpage

\Large What is Differential Equations?\normalsize

Recall from calculus that the derivative measures how fast a situation is 
changing (a big derivative implies that the situation is changing fast).  
In some instances, the rate of change is dependent only on time.  For 
example, if a train is on a very strict schedule then it's speed at any 
time is dependent only on time.  For most real-life situations, however, 
the rate of change of a situation depends on the make-up of that 
situation.

For example, the bigger a black hole is, the faster it pulls stuff in to 
itself.  The faster a car is going down a hill, the more air resistance 
will slow it down.  In most examples you can think of, if a situation is 
changing then the make-up of the situation probably influences the rate at 
which it is changing.  Such processes are governed by differential 
equations.

In math it seems to be the case that the more useful something is, the 
harder it is to solve it.  Differential equations are very useful!  As a 
result, they are generally impossible to solve (at this point in 
humanity's understanding).  However, humans have found some types of 
differential equations that are actually solvable; a few such types will 
make up most of this course.

The rest of the course is the study of how solutions to a given equation, 
say $y'=f(x)$, fit together into a space.  Each point in the space is a 
solution to the equation, and if two points in the space are close to 
each other than the solutions they represent are also close to each other 
(e.g. their graphs are similar).

\Large Conclusion\normalsize

Please try to understand my philosophy, because it will govern the class 
and your part in it.  One part of my philosophy which you should 
understand is that I am interested in communicating as much knowledge to 
each of you as I can, thereby improving everyone's experience in the 
class.  My philosophy is first and foremost one of ``evolution."  
Therefore, if my stated philosophy conflicts with what is best for the 
class, then I will be happy to change it.  We'll talk.

Please also take seriously your part in this class.  Your engagement in 
the class will be the strongest indicator of your success in and your 
enjoyment of the class.

\end{document}
