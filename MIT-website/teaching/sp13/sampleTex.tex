\documentclass{book}

\usepackage{etex}
\usepackage{savesym}
\usepackage{amssymb, amsmath,amsthm,amscd}
%\savesymbol{lrcorner}
\usepackage{txfonts}
%\usepackage{marvosym}
\usepackage{wasysym}
\savesymbol{Sun}\savesymbol{Mercury}\savesymbol{Venus}\savesymbol{Earth}\savesymbol{Mars}\savesymbol{Jupiter}\savesymbol{Saturn}\savesymbol{Uranus}\savesymbol{Neptune}\savesymbol{Pluto}\savesymbol{leftmoon}\savesymbol{rightmoon}\savesymbol{fullmoon}\savesymbol{newmoon}\savesymbol{Aries}\savesymbol{Taurus}\savesymbol{Gemini}\savesymbol{Leo}\savesymbol{Libra}\savesymbol{Scorpio}\savesymbol{diameter}
\usepackage{mathabx}
%\usepackage{stmaryrd}
\usepackage{setspace}
\usepackage{chngcntr}
\usepackage[tt]{titlepic}
\usepackage{enumerate,makecell}
\usepackage{makeidx,tabularx,dashbox}
\usepackage[usenames,dvipsnames]{xcolor}
\usepackage[bookmarks=true,colorlinks=true, linkcolor=MidnightBlue, citecolor=cyan]{hyperref}
\usepackage{lmodern}
\usepackage{graphicx,float}
\usepackage{multirow}
\usepackage{geometry}
\newgeometry{left=1.6in,right=1.6in,top=1.4in,bottom=1.4in}

\usepackage{color}
\usepackage[all,poly,color,matrix,arrow]{xy}
\makeindex

%\usepackage{showkeys}

\newcommand{\comment}[1]{}

\newcommand{\longnote}[2][4.9in]{\fcolorbox{black}{yellow}{\parbox{#1}{\color{black} #2}}}
\newcommand{\shortnote}[1]{\fcolorbox{black}{yellow}{\color{black} #1}}
\newcommand{\start}[1]{\shortnote{Start here: #1.}}
\newcommand{\q}[1]{\begin{question}#1\end{question}}
\newcommand{\g}[1]{\begin{guess}#1\end{guess}}
\newcommand{\cfbox}[2]{
    \colorlet{currentcolor}{.}
    {\color{#1}
    \fbox{\color{currentcolor}#2}}
}

\def\tn{\textnormal}
\def\mf{\mathfrak}
\def\mc{\mathcal}
\newcommand{\qt}[1]{\tn{``}#1\tn{"}}

\def\ZZ{{\mathbb Z}}
\def\QQ{{\mathbb Q}}
\def\RR{{\mathbb R}}
\def\CC{{\mathbb C}}
\def\AA{{\mathbb A}}
\def\PP{{\mathbb P}}
\def\NN{{\mathbb N}}


\def\Hom{\tn{Hom}}
\def\Path{\tn{Path}}
\def\Paths{\tn{Paths}}
\def\List{\tn{List}}
\def\im{\tn{im}}
\def\Fun{\tn{Fun}}
\def\Ob{\tn{Ob}}
\def\Skel{\tn{Skel}}
\def\Op{\tn{Op}}
\def\PK{\tn{PK}}
\def\FK{\tn{FK}}
\def\SEL*{\tn{SEL*}}
\def\Res{\tn{Res}}
\def\hsp{\hspace{.3in}}
\newcommand{\hsps}[1]{{\hspace{2mm} #1\hspace{2mm}}}
\newcommand{\tin}[1]{\text{\tiny #1}}

\def\singleton{\{\smiley\}}
\newcommand{\boxtitle}[1]{\begin{center}#1\end{center}\vspace{-.1in}}
\newcommand{\singlefun}[1]{\star^{#1}}
\newcommand{\pullb}[1]{\Delta_{#1}}
\newcommand{\lpush}[1]{\Sigma_{#1}}
\newcommand{\rpush}[1]{\Pi_{#1}}
\def\lcone{^\triangleleft}
\def\rcone{^\triangleright}
\def\to{\rightarrow}
\def\from{\leftarrow}
\def\down{\downarrrow}
\def\Down{\Downarrow}
\def\cross{\times}
\def\taking{\colon}
\def\inj{\hookrightarrow}
\def\surj{\twoheadrightarrow}
\def\too{\longrightarrow}
\newcommand{\xyright}[1]{\xymatrix{~\ar[r]#1&}}
\newcommand{\xydown}[1]{\xymatrix{~\ar[d]#1\\~}}
\newcommand{\xydoown}[1]{\xymatrix{~\ar[ddd]#1\\\parbox{0in}{~}\\\parbox{0in}{~}\\~}}
\def\fromm{\longleftarrow}
\def\tooo{\longlongrightarrow}
\def\tto{\rightrightarrows}
\def\ttto{\equiv\!\!>}
\def\ss{\subseteq}
\def\superset{\supseteq}
\def\iso{\cong}
\def\down{\downarrow}
\def\|{{\;|\;}}
\def\m1{{-1}}
\def\op{^\tn{op}}
\def\loc{\tn{loc}}
\def\la{\langle}
\def\ra{\rangle}
\def\wt{\widetilde}
\def\wh{\widehat}
\def\we{\simeq}
\def\ol{\overline}
\def\ul{\underline}
\def\plpl{+\!\!+\hspace{1pt}}
\def\acts{\lefttorightarrow}
\def\vect{\overrightarrow}
\def\qeq{\mathop{=}^?}
\def\del{\partial\,}

\def\rr{\raggedright}

%\newcommand{\LMO}[1]{\bullet^{#1}}
%\newcommand{\LTO}[1]{\bullet^{\tn{#1}}}
\newcommand{\LMO}[1]{\stackrel{#1}{\bullet}}
\newcommand{\LTO}[1]{\stackrel{\tt{#1}}{\bullet}}
\newcommand{\LA}[2]{\ar[#1]^-{\tn {#2}}}
\newcommand{\LAL}[2]{\ar[#1]_-{\tn {#2}}}
\newcommand{\obox}[3]{\stackrel{#1}{\fbox{\parbox{#2}{#3}}}}
\newcommand{\labox}[2]{\obox{#1}{1.6in}{#2}}
\newcommand{\mebox}[2]{\obox{#1}{1in}{#2}}
\newcommand{\smbox}[2]{\stackrel{#1}{\fbox{#2}}}
\newcommand{\fakebox}[1]{\tn{$\ulcorner$#1$\urcorner$}}
\newcommand{\sq}[4]{\xymatrix{#1\ar[r]\ar[d]&#2\ar[d]\\#3\ar[r]&#4}}
\newcommand{\namecat}[1]{\begin{center}$#1:=$\end{center}}

\def\monOb{\blacktriangle}

\def\ullimit{\ar@{}[rd]|(.25)*+{\Large\lrcorner}}
\def\urlimit{\ar@{}[ld]|(.25)*+{\Large\llcorner}}
\def\lllimit{\ar@{}[ru]|(.25)*+{\Large\urcorner}}
\def\lrlimit{\ar@{}[lu]|(.25)*+{\Large\ulcorner}}
\def\ulhlimit{\ar@{}[rd]|(.3)*+{\diamond}}
\def\urhlimit{\ar@{}[ld]|(.3)*+{\diamond}}
\def\llhlimit{\ar@{}[ru]|(.3)*+{\diamond}}
\def\lrhlimit{\ar@{}[lu]|(.3)*+{\diamond}}
\newcommand{\clabel}[1]{\ar@{}[rd]|(.5)*+{#1}}
\newcommand{\TriRight}[7]{\xymatrix{#1\ar[dr]_{#2}\ar[rr]^{#3}&&#4\ar[dl]^{#5}\\&#6\ar@{}[u] |{\Longrightarrow}\ar@{}[u]|>>>>{#7}}}
\newcommand{\TriLeft}[7]{\xymatrix{#1\ar[dr]_{#2}\ar[rr]^{#3}&&#4\ar[dl]^{#5}\\&#6\ar@{}[u] |{\Longleftarrow}\ar@{}[u]|>>>>{#7}}}
\newcommand{\TriIso}[7]{\xymatrix{#1\ar[dr]_{#2}\ar[rr]^{#3}&&#4\ar[dl]^{#5}\\&#6\ar@{}[u] |{\Longleftrightarrow}\ar@{}[u]|>>>>{#7}}}


\newcommand{\arr}[1]{\ar@<.5ex>[#1]\ar@<-.5ex>[#1]}
\newcommand{\arrr}[1]{\ar@<.7ex>[#1]\ar@<0ex>[#1]\ar@<-.7ex>[#1]}
\newcommand{\arrrr}[1]{\ar@<.9ex>[#1]\ar@<.3ex>[#1]\ar@<-.3ex>[#1]\ar@<-.9ex>[#1]}
\newcommand{\arrrrr}[1]{\ar@<1ex>[#1]\ar@<.5ex>[#1]\ar[#1]\ar@<-.5ex>[#1]\ar@<-1ex>[#1]}

\newcommand{\To}[1]{\xrightarrow{#1}}
\newcommand{\Too}[1]{\xrightarrow{\ \ #1\ \ }}
\newcommand{\From}[1]{\xleftarrow{#1}}
\newcommand{\Fromm}[1]{\xleftarrow{\ \ #1\ \ }}

\newcommand{\Adjoint}[4]{\xymatrix@1{#2 \ar@<.5ex>[r]^-{#1} & #3 \ar@<.5ex>[l]^-{#4}}}
\newcommand{\adjoint}[4]{\xymatrix{#1\taking #2\ar@<.5ex>[r]& #3\hspace{1pt}:\hspace{-2pt} #4\ar@<.5ex>[l]}}

\def\id{\tn{id}}
\def\Top{{\bf Top}}
\def\Cat{{\bf Cat}}
\def\Oprd{{\bf Oprd}}
\def\Str{{\bf Str}}
\def\Mon{{\bf Mon}}
\def\Grp{{\bf Grp}}
\def\Grph{{\bf Grph}}
\def\Type{{\bf Type}}
\def\Supp{{\bf Supp}}
\def\Dist{{\bf Dist}}
\def\Vect{{\bf Vect}}
\def\Kls{{\bf Kls}}
\def\Prop{{\bf Prop}}
\def\FLin{{\bf FLin}}
\def\Set{{\bf Set}}
\def\Sets{{\bf Sets}}
\def\PrO{{\bf PrO}}
\def\Star{{\bf Star}}
\def\Cob{{\bf Cob}}
\def\Qry{{\bf Qry}}
\def\set{{\text \textendash}{\bf Set}}
\def\sSet{{\bf sSet}}
\def\sSets{{\bf sSets}}
\def\Grpd{{\bf Grpd}}
\def\Pre{{\bf Pre}}
\def\Shv{{\bf Shv}}
\def\Rings{{\bf Rings}}
\def\bD{{\bf \Delta}}
\def\dispInt{\parbox{.1in}{$\int$}}
\def\bhline{\Xhline{2\arrayrulewidth}}
\def\bbhline{\Xhline{2.5\arrayrulewidth}}
\def\bbbhline{\Xhline{3\arrayrulewidth}}


\def\colim{\mathop{\tn{colim}}}
\def\hocolim{\mathop{\tn{hocolim}}}

\def\mcA{\mc{A}}
\def\mcB{\mc{B}}
\def\mcC{\mc{C}}
\def\mcD{\mc{D}}
\def\mcE{\mc{E}}
\def\mcF{\mc{F}}
\def\mcG{\mc{G}}
\def\mcH{\mc{H}}
\def\mcI{\mc{I}}
\def\mcJ{\mc{J}}
\def\mcK{\mc{K}}
\def\mcL{\mc{L}}
\def\mcM{\mc{M}}
\def\mcN{\mc{N}}
\def\mcO{\mc{O}}
\def\mcP{\mc{P}}
\def\mcQ{\mc{Q}}
\def\mcR{\mc{R}}
\def\mcS{\mc{S}}
\def\mcT{\mc{T}}
\def\mcU{\mc{U}}
\def\mcV{\mc{V}}
\def\mcW{\mc{W}}
\def\mcX{\mc{X}}
\def\mcY{\mc{Y}}
\def\mcZ{\mc{Z}}

\def\undsc{\rule{2mm}{0.4pt}}
\def\Loop{{\mcL oop}}
\def\LoopSchema{{\parbox{.5in}{\fbox{\xymatrix{\LMO{s}\ar@(l,u)[]^f}}}}}

\newtheorem{theorem}[subsubsection]{Theorem}
\newtheorem{lemma}[subsubsection]{Lemma}
\newtheorem{proposition}[subsubsection]{Proposition}
\newtheorem{corollary}[subsubsection]{Corollary}
\newtheorem{fact}[subsubsection]{Fact}

\theoremstyle{remark}
\newtheorem{remark}[subsubsection]{Remark}
\newtheorem{example}[subsubsection]{Example}
\newtheorem{warning}[subsubsection]{Warning}
\newtheorem{question}[subsubsection]{Question}
\newtheorem{guess}[subsubsection]{Guess}
\newtheorem{answer}[subsubsection]{Answer}
\newtheorem{construction}[subsubsection]{Construction}
\newtheorem{rules}[subsubsection]{Rules of good practice}
\newtheorem{exc}[subsubsection]{Exercise}
\newenvironment{exercise}{\begin{exc}}{\hspace*{\fill}$\lozenge$\end{exc}}
\newtheorem{app}[subsubsection]{Application}
\newenvironment{application}{\begin{app}}{\hspace*{\fill}$\lozenge\lozenge$\end{app}}

%\newenvironment{exercise}{\addtocounter{theorem}{1}\vspace{.1in}\begin{sloppypar}\noindent{\em Exercise}\;\arabic{chapter}.\arabic{section}.\arabic{subsection}.\arabic{theorem}.}{\end{sloppypar}\vspace{.1in}}

\newenvironment{slogan}{\addtocounter{subsubsection}{1}\vspace{.1in}\begin{sloppypar}\noindent{\em Slogan}\;\arabic{chapter}.\arabic{section}.\arabic{subsection}.\arabic{subsubsection}. \begin{quote}``\slshape}{"\end{quote}\end{sloppypar}\vspace{.1in}}

%\newenvironment{application}{\addtocounter{subsubsection}{1}\vspace{.1in}\begin{sloppypar}\noindent{\em Application}\;\arabic{chapter}.\arabic{section}.\arabic{subsection}.\arabic{subsubsection}. \begin{quote}}{\end{quote}\end{sloppypar}\vspace{.1in}}
\makeatletter\let\c@figure\c@equation\makeatother

\theoremstyle{definition}
\newtheorem{definition}[subsubsection]{Definition}
\newtheorem{notation}[subsubsection]{Notation}
\newtheorem{conjecture}[subsubsection]{Conjecture}
\newtheorem{postulate}[subsubsection]{Postulate}


%\newtheorem{theorem}{Theorem}[subsection]
%\newtheorem{lemma}[theorem]{Lemma}
%\newtheorem{proposition}[theorem]{Proposition}
%\newtheorem{corollary}[theorem]{Corollary}
%\newtheorem{fact}[theorem]{Fact}
%
%\theoremstyle{remark}
%\newtheorem{remark}[theorem]{Remark}
%\newtheorem{example}[theorem]{Example}
%\newtheorem{warning}[theorem]{Warning}
%\newtheorem{question}[theorem]{Question}
%\newtheorem{guess}[theorem]{Guess}
%\newtheorem{answer}[theorem]{Answer}
%\newtheorem{construction}[theorem]{Construction}
%\newtheorem{rules}[theorem]{Rules of good practice}
%\newtheorem{exc}[theorem]{Exercise}
%%\newenvironment{exercise}{\addtocounter{theorem}{1}\vspace{.1in}\begin{sloppypar}\noindent{\em Exercise}\;\arabic{chapter}.\arabic{section}.\arabic{subsection}.\arabic{theorem}.}{\end{sloppypar}\vspace{.1in}}
%\newenvironment{exercise}{\begin{exc}}{\hspace*{\fill}$\lozenge$\end{exc}}
%
%\theoremstyle{definition}
%\newtheorem{definition}[theorem]{Definition}
%\newtheorem{notation}[theorem]{Notation}
%\newtheorem{conjecture}[theorem]{Conjecture}
%\newtheorem{postulate}[theorem]{Postulate}

\def\Finm{{\bf Fin_{m}}}
\def\Prb{{\bf Prb}}
\def\Prbs{{\wt{\bf Prb}}}
\def\El{{\bf El}}
\def\Gr{{\bf Gr}}
\def\DT{{\bf DT}}
\def\DB{{\bf DB}}
\def\Tables{{\bf Tables}}
\def\Sch{{\bf Sch}}
\def\Fin{{\bf Fin}}
\def\P{{\bf P}}
\def\SC{{\bf SC}}
\def\ND{{\bf ND}}
\def\Poset{{\bf Poset}}


\newcommand{\MainCatLarge}[1]{ 
	\stackrel{#1}{
		\parbox{4.5in}{\fbox{\parbox{4.4in}{\begin{center}\underline{{\tt Employee} manager worksIn $\simeq$ {\tt Employee} worksIn}\hsp  \underline{{\tt Department} secretary worksIn $\simeq$ {\tt Department}}\end{center}~\\\\\\
			\xymatrix@=8pt{&\LTO{Employee}\ar@<.5ex>[rrrrr]^{\tn{worksIn}}\ar@(l,u)[]+<5pt,10pt>^{\tn{manager}}\ar[dddl]_{\tn{first}}\ar[dddr]^{\tn{last}}&&&&&\LTO{Department}\ar@<.5ex>[lllll]^{\tn{secretary}}\ar[ddd]^{\tn{name}}\\\\\\\LTO{FirstNameString}&&\LTO{LastNameString}&~&~&~&\LTO{DepartmentNameString}
			}
		}}}
	}
}
%\CompileMatrices

\setcounter{secnumdepth}{3}
\setcounter{tocdepth}{1}

\def\sub{\begin{itemize}\item}
\def\sexc{\begin{enumerate}[a.)]\setlength{\itemsep}{.1cm}\setlength{\parskip}{.1cm}\item}
\def\next{\item}
\def\endsub{\end{itemize}}
\def\endsexc{\end{enumerate}}

%%%%%

\begin{document}

\chapter{Sets and functions}
\section{Ologs}
\subsection{Facts}
\subsubsection{Images}\label{sec:images}\index{olog!images}

In this section we discuss a specific kind of fact, generated by any aspect. Recall that every function has an image, meaning the subset of elements in the codomain that are ``hit" by the function. For example the function $f(x)=2*x\taking \ZZ\to\ZZ$ has as image the set of all even numbers.

Similarly the set of mothers arises as is the image of the ``has as mother" function, as shown below 
$$
\xymatrix{\obox{P}{.5in}{a person}\LAL{rd}{has}\LA{rr}{$\stackrel{f\taking P\to P}{\tn{has as mother}}$}&&\obox{P}{.5in}{a person}\\
&\obox{M=\im(f)}{.6in}{a mother}\LAL{ur}{is}\ar@{}[u]|(.6){\checkmark}
}$$

\begin{exercise}
For each of the following types, write down a function for which it is the image, or say ``not clearly an image type" 
\sexc \fakebox{a book}
\next \fakebox{a material that has been fabricated by a process of type $T$}
\next \fakebox{a bicycle owner}
\next \fakebox{a child}
\next \fakebox{a used book}
\next \fakebox{an inhabited residence}
\endsexc
\end{exercise}

%%%%%% Section %%%%%%

\section{Products and coproducts}\label{sec:prods and coprods in set}

In this section we introduce two concepts that are likely to be familiar, although perhaps not by their category-theoretic names, product and coproduct. Each is an example of a large class of ideas that exist far beyond the realm of sets.

%%%% Subsection %%%%

\subsection{Products}\label{sec:products}\index{products!of sets}

\begin{definition}

Let $X$ and $Y$ be sets. The {\em product of $X$ and $Y$}, denoted $X\cross Y$,\index{a symbol!$\cross$} is defined as the set of ordered pairs $(x,y)$ where $x\in X$ and $y\in Y$. Symbolically, $$X\cross Y=\{(x,y)\|x\in X,\;\; y\in Y\}.$$ There are two natural {\em projection functions} $\pi_1\taking X\cross Y\to X$ and $\pi_2\taking X\cross Y\to Y$.
$$\xymatrix@=15pt{&X\cross Y\ar[ddr]^{\pi_2}\ar[ddl]_{\pi_1}\\\\X&&Y}$$

\end{definition}

\begin{example}\label{ex:grid1}[Grid of dots]\index{product!as grid}

Let $X=\{1,2,3,4,5,6\}$ and $Y=\{\clubsuit,\diamondsuit,\heartsuit,\spadesuit\}$. Then we can draw $X\cross Y$ as a 6-by-4 grid of dots, and the projections as projections
\begin{align}
\parbox{2.9in}{\begin{center}\small $X\cross Y$\vspace{-.1in}\end{center}\fbox{
\xymatrix@=10pt{
\LMO{(1,\clubsuit)}&\LMO{(2,\clubsuit)}&\LMO{(3,\clubsuit)}&\LMO{(4,\clubsuit)}&\LMO{(5,\clubsuit)}&\LMO{(6,\clubsuit)}\\
\LMO{(1,\diamondsuit)}&\LMO{(2,\diamondsuit)}&\LMO{(3,\diamondsuit)}&\LMO{(4,\diamondsuit)}&\LMO{(5,\diamondsuit)}&\LMO{(6,\diamondsuit)}\\
\LMO{(1,\heartsuit)}&\LMO{(2,\heartsuit)}&\LMO{(3,\heartsuit)}&\LMO{(4,\heartsuit)}&\LMO{(5,\heartsuit)}&\LMO{(6,\heartsuit)}\\
\LMO{(1,\spadesuit)}&\LMO{(2,\spadesuit)}&\LMO{(3,\spadesuit)}&\LMO{(4,\spadesuit)}&\LMO{(5,\spadesuit)}&\LMO{(6,\spadesuit)}\\
}}}
\parbox{.9in}{
\xymatrix{~\ar[rr]^{\pi_2}&&~}
}
\parbox{.3in}{\begin{center}\small $Y$\vspace{-.1in}\end{center}\fbox{
\xymatrix@=10pt{
\LMO{\clubsuit}\\\LMO{\diamondsuit}\\\LMO{\heartsuit}\\\LMO{\spadesuit}
}}}
\\\nonumber
\parbox{1in}{\hspace{-1.95in}\xymatrix{~\ar[dd]_{\pi_1}\\\\~}}
\\\nonumber
\parbox{2.9in}{\hspace{-1.2in}\fbox{
\xymatrix@=24pt{
\LMO{1}&\LMO{2}&\LMO{3}&\LMO{4}&\LMO{5}&\LMO{6}
}}\begin{center}\hspace{-2.6in}\small$X$\end{center}}
\end{align}

\end{example}

\begin{application}
A traditional (Mendelian) way to predict the genotype of offspring based on the genotype of its parents is by the use of \href{http://en.wikipedia.org/wiki/Punnett_square}{Punnett squares}. If $F$ is the set of possible genotypes for the female parent and $M$ is the set of possible genotypes of the male parent, then $F\cross M$ is drawn as a square, called a Punnett square, in which every combination is drawn. 
\end{application}

\begin{exercise}
How many elements does the set $\{a,b,c,d\}\cross\{1,2,3\}$ have?
\end{exercise}

\begin{application}

Suppose we are conducting experiments about the mechanical properties of materials, as in Application \ref{app:force-extension}. For each material sample we will produce multiple data points in the set $\fakebox{extension}\cross\fakebox{force}\iso\RR\cross\RR$.

\end{application}

\begin{remark}

It is possible to take the product of more than two sets as well. For example, if $A,B,$ and $C$ are sets then $A\cross B\cross C$ is the set of triples, 
$$A\cross B\cross C:=\{(a,b,c)\|a\in A, b\in B, c\in C\}.$$

This kind of generality is useful in understanding multiple dimensions, e.g. what physicists mean by 10-dimensional space. It comes under the heading of {\em limits}, which we will see in Section \ref{sec:lims and colims in a cat}.

\end{remark}

\begin{example}\label{ex:R2}

Let $\RR$\index{a symbol!$\RR$} be the set of real numbers. By $\RR^2$ we mean $\RR\cross\RR$ (though see Exercise \ref{exc:two R2s}). Similarly, for any $n\in\NN$, we define $\RR^n$ to be the product of $n$ copies of $\RR$. 

According to \cite{Pen}, Aristotle seems to have conceived of space as something like $S:=\RR^3$ and of time as something like $T:=\RR$. Spacetime, had he conceived of it, would probably have been $S\cross T\iso\RR^4$. He of course did not have access to this kind of abstraction, which was probably due to Descartes. 

\end{example}

\begin{exercise}
Let $\ZZ$ denote the set of integers, and let $+\taking\ZZ\cross\ZZ\to\ZZ$ denote the addition function and $\cdot\taking\ZZ\cross\ZZ\to\ZZ$ denote the multiplication function. Which of the following diagrams commute?
\sexc $$\xymatrix{
\ZZ\cross\ZZ\cross\ZZ\ar[rr]^-{(a,b,c)\mapsto(a\cdot b,a\cdot c)}\ar[d]_{(a,b,c)\mapsto(a+b,c)}&\hsp&\ZZ\cross\ZZ\ar[d]^{(x,y)\mapsto x+y}\\
\ZZ\cross\ZZ\ar[rr]_{(x,y)\mapsto xy}&&\ZZ}
$$
\next $$
\xymatrix{
\ZZ\ar[rr]^{x\mapsto (x,0)}\ar[drr]_{\id_\ZZ}&&\ZZ\cross\ZZ\ar[d]^{(a,b)\mapsto a\cdot b}\\&&\ZZ}
$$
\next$$
\xymatrix{
\ZZ\ar[rr]^{x\mapsto (x,1)}\ar[drr]_{\id_\ZZ}&&\ZZ\cross\ZZ\ar[d]^{(a,b)\mapsto a\cdot b}\\&&\ZZ}
$$
\endsexc
\end{exercise}

%% Subsubsection %%

\subsubsection{Universal property for products}\index{products!universal property of}\index{universal property!products}

\begin{lemma}[Universal property for product]\label{lemma:up for prod}

Let $X$ and $Y$ be sets. For any set $A$ and functions $f\taking A\to X$ and $g\taking A\to Y$, there exists a unique function $A\to X\cross Y$ such that the following diagram commutes \footnote{The symbol $\forall$ is read ``for all"; the symbol $\exists$ is read ``there exists", and the symbol $\exists!$ is read ``there exists a unique". So this diagram is intended to express the idea that for any functions $f\taking A\to X$ and $g\taking A\to Y$, there exists a unique function $A\to X\cross Y$ for which the two triangles commute.}
\begin{align}\label{dia:univ prop for products}
\xymatrix@=15pt{&X\cross Y\ar[ldd]_{\pi_1}\ar[rdd]^{\pi_2}\\\\X\ar@{}[r]|{\checkmark}&&Y\ar@{}[l]|{\checkmark}\\\\&A\ar[luu]^{\forall f}\ar[ruu]_{\forall g}\ar@{-->}[uuuu]^{\exists !}}
\end{align}
We might write the unique function as $$f\cross g\taking A\to X\cross Y.$$

\end{lemma}

\begin{proof}

Suppose given $f,g$ as above. To provide a function $\ell\taking A\to X\cross Y$ is equivalent to providing an element $\ell(a)\in X\cross Y$ for each $a\in A$. We need such a function for which $\pi_1\circ \ell=f$ and $\pi_2\circ \ell=g$. An element of $X\cross Y$ is an ordered pair $(x,y)$, and we can use $\ell(a)=(x,y)$ if and only if $x=\pi_1(x,y)=f(a)$ and $y=\pi_2(x,y)=g(a)$. So it is necessary and sufficient to define $$(f\cross g)(a)=(f(a),g(a))$$ for all $a\in A$.

\end{proof}

\end{document}
